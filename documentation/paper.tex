\documentclass[11pt,twocolumn]{article}
\usepackage[]{inputenc}
\usepackage[T1]{fontenc}
\usepackage{fullpage}
\usepackage{coqdoc}
\usepackage{amsmath,amssymb}
\usepackage{url}
\usepackage{nopageno}
\title{Towards a Formalization of \\ Claude Shannon's Masters Thesis}
\author{andrewtron3000 @ github}

\date{\today} 
\begin{document}

\maketitle
\begin{abstract}

\end{abstract}
\begin{coqdoccode}
\coqdocemptyline
\end{coqdoccode}
\section{Overview}





Claude Shannon's seminal MS thesis \cite{shannon40} is considered by
many to be the most important masters thesis of the 20th century.


Shannon's thesis laid the foundation for the digital revolution by
showing how to reason about electromechanical relay circuits using
boolean algebra and propositional logic.


Shannon's thesis contains many theorems and assertions -- most of
which are not proven.  This paper sets out to prove many of those
theorems and assertions.


Interested readers can obtain a copy of Shannon's M.S. thesis in the
link provided in the reference.  Most of the theorems are numbered in
the original thesis and this paper refers to those numbers.
Occasionally an assertion is not uniquely identifiable in the thesis,
and in this case, a page number is used to identify it.


The available copy of the thesis is a scanned PDF of a typewritten
manuscript.  The quality of the scan is somewhat poor and this was
another motivation to subject the contents of the thesis to more
rigorous examination.  Despite being published in 1936, we find that
the typography is sorely lacking: no doubt a harbinger of the
degradation of typography during the 20th century \cite{qnuth07}.




\section{Postulates}





In order to define the circuit algebra, we first introduce the notion
of a circuit, which can be either closed (conducting) or open
(non-conducting).  This is postulate 4 from the thesis.  We specify
this postulate as an inductive type in Coq:


\begin{coqdoccode}
\coqdocemptyline
\coqdocnoindent
\coqdockw{Inductive} \coqdef{computational.circuit}{circuit}{\coqdocinductive{circuit}} : \coqdockw{Type} :=\coqdoceol
\coqdocnoindent
\ensuremath{|} \coqdef{computational.closed}{closed}{\coqdocconstructor{closed}} : \coqref{computational.circuit}{\coqdocinductive{circuit}}   \coqdoceol
\coqdocnoindent
\ensuremath{|} \coqdef{computational.open}{open}{\coqdocconstructor{open}} : \coqref{computational.circuit}{\coqdocinductive{circuit}}. \coqdocemptyline
\end{coqdoccode}


Next, we define the notion of ``plus'' in the circuit algebra.  Plus
is synonymous with two circuits in series, or the AND operation in
boolean algebra.  This is defined in postulates 1b, 2a and 3a.  We use
a Coq definition to formalize this notion.


\begin{coqdoccode}
\coqdocemptyline
\coqdocnoindent
\coqdockw{Definition} \coqdef{computational.plus}{plus}{\coqdocdefinition{plus}} (\coqdocvar{v1} \coqdocvar{v2} : \coqref{computational.circuit}{\coqdocinductive{circuit}}) : \coqref{computational.circuit}{\coqdocinductive{circuit}} :=\coqdoceol
\coqdocindent{1.00em}
\coqdockw{match} \coqdocvariable{v1}, \coqdocvariable{v2} \coqdockw{with}\coqdoceol
\coqdocindent{1.00em}
\ensuremath{|} \coqref{computational.open}{\coqdocconstructor{open}}, \coqref{computational.open}{\coqdocconstructor{open}} \ensuremath{\Rightarrow} \coqref{computational.open}{\coqdocconstructor{open}} \coqdoceol
\coqdocindent{1.00em}
\ensuremath{|} \coqref{computational.open}{\coqdocconstructor{open}}, \coqref{computational.closed}{\coqdocconstructor{closed}} \ensuremath{\Rightarrow} \coqref{computational.open}{\coqdocconstructor{open}} \coqdoceol
\coqdocindent{1.00em}
\ensuremath{|} \coqref{computational.closed}{\coqdocconstructor{closed}}, \coqref{computational.open}{\coqdocconstructor{open}} \ensuremath{\Rightarrow} \coqref{computational.open}{\coqdocconstructor{open}} \coqdoceol
\coqdocindent{1.00em}
\ensuremath{|} \coqref{computational.closed}{\coqdocconstructor{closed}}, \coqref{computational.closed}{\coqdocconstructor{closed}} \ensuremath{\Rightarrow} \coqref{computational.closed}{\coqdocconstructor{closed}} \coqdoceol
\coqdocindent{1.00em}
\coqdockw{end}.\coqdoceol
\coqdocemptyline
\end{coqdoccode}


Next, we define the notion of ``times'' in the circuit algebra.  Times
is synonymous with two circuits in parallel (or the OR operation in
boolean algebra) and it is defined in postulates 1a, 2b, and 3b.


\begin{coqdoccode}
\coqdocemptyline
\coqdocnoindent
\coqdockw{Definition} \coqdef{computational.times}{times}{\coqdocdefinition{times}} (\coqdocvar{v1} \coqdocvar{v2} : \coqref{computational.circuit}{\coqdocinductive{circuit}}) : \coqref{computational.circuit}{\coqdocinductive{circuit}} :=\coqdoceol
\coqdocindent{1.00em}
\coqdockw{match} \coqdocvariable{v1}, \coqdocvariable{v2} \coqdockw{with}\coqdoceol
\coqdocindent{1.00em}
\ensuremath{|} \coqref{computational.closed}{\coqdocconstructor{closed}}, \coqref{computational.closed}{\coqdocconstructor{closed}} \ensuremath{\Rightarrow} \coqref{computational.closed}{\coqdocconstructor{closed}} \coqdoceol
\coqdocindent{1.00em}
\ensuremath{|} \coqref{computational.open}{\coqdocconstructor{open}}, \coqref{computational.closed}{\coqdocconstructor{closed}} \ensuremath{\Rightarrow} \coqref{computational.closed}{\coqdocconstructor{closed}} \coqdoceol
\coqdocindent{1.00em}
\ensuremath{|} \coqref{computational.closed}{\coqdocconstructor{closed}}, \coqref{computational.open}{\coqdocconstructor{open}} \ensuremath{\Rightarrow} \coqref{computational.closed}{\coqdocconstructor{closed}} \coqdoceol
\coqdocindent{1.00em}
\ensuremath{|} \coqref{computational.open}{\coqdocconstructor{open}}, \coqref{computational.open}{\coqdocconstructor{open}} \ensuremath{\Rightarrow} \coqref{computational.open}{\coqdocconstructor{open}} \coqdoceol
\coqdocindent{1.00em}
\coqdockw{end}.\coqdoceol
\coqdocemptyline
\end{coqdoccode}


Now we define some convenient Coq notation for these plus and times
functions so that further developments can use the normal symbols for
plus and times.


\begin{coqdoccode}
\coqdocemptyline
\coqdocnoindent
\coqdockw{Notation} \coqdef{computational.::x '+' x}{"}{"}x + y" :=\coqdoceol
\coqdocindent{1.00em}
(\coqref{computational.plus}{\coqdocdefinition{plus}} \coqdocvar{x} \coqdocvar{y})\coqdoceol
\coqdocindent{2.00em}
(\coqdoctac{at} \coqdockw{level} 50,\coqdoceol
\coqdocindent{2.50em}
\coqdoctac{left} \coqdockw{associativity}).\coqdoceol
\coqdocemptyline
\coqdocnoindent
\coqdockw{Notation} \coqdef{computational.::x '*' x}{"}{"}x * y" :=\coqdoceol
\coqdocindent{1.00em}
(\coqref{computational.times}{\coqdocdefinition{times}} \coqdocvar{x} \coqdocvar{y})\coqdoceol
\coqdocindent{2.00em}
(\coqdoctac{at} \coqdockw{level} 40,\coqdoceol
\coqdocindent{2.50em}
\coqdoctac{left} \coqdockw{associativity}).\coqdoceol
\coqdocemptyline
\end{coqdoccode}


Next we will introduce the definition of negation.  As one would
expect, negation of an open circuit is closed and negation of a closed
circuit is open.


\begin{coqdoccode}
\coqdocemptyline
\coqdocnoindent
\coqdockw{Definition} \coqdef{computational.negation}{negation}{\coqdocdefinition{negation}} (\coqdocvar{v1} : \coqref{computational.circuit}{\coqdocinductive{circuit}}) : \coqref{computational.circuit}{\coqdocinductive{circuit}} :=\coqdoceol
\coqdocindent{1.00em}
\coqdockw{match} \coqdocvariable{v1} \coqdockw{with}\coqdoceol
\coqdocindent{1.00em}
\ensuremath{|} \coqref{computational.open}{\coqdocconstructor{open}} \ensuremath{\Rightarrow} \coqref{computational.closed}{\coqdocconstructor{closed}}\coqdoceol
\coqdocindent{1.00em}
\ensuremath{|} \coqref{computational.closed}{\coqdocconstructor{closed}} \ensuremath{\Rightarrow} \coqref{computational.open}{\coqdocconstructor{open}}\coqdoceol
\coqdocindent{1.00em}
\coqdockw{end}.\coqdoceol
\coqdocemptyline
\end{coqdoccode}
\section{Proofs of First Theorems}




With these postulates defined, we can begin using them to prove
theorems 1 through 5.


Throughout this development, with just a few exceptions, our goal is
to have all proofs proven with a single tactic application, following
the Chlipala discipline \cite{chlipala13a}.


To do this, we start by defining a set of custom Ltac tactics below.


\begin{coqdoccode}
\coqdocemptyline
\coqdocnoindent
\coqdockw{Ltac} \coqdocvar{reduce1} \coqdocvar{X} :=\coqdoceol
\coqdocindent{1.00em}
\coqdoctac{try} \coqdoctac{destruct} \coqdocvar{X};\coqdoceol
\coqdocindent{1.00em}
\coqdoctac{simpl};\coqdoceol
\coqdocindent{1.00em}
\coqdoctac{reflexivity}.\coqdoceol
\coqdocemptyline
\coqdocnoindent
\coqdockw{Ltac} \coqdocvar{reduce2} \coqdocvar{X} \coqdocvar{Y} :=\coqdoceol
\coqdocindent{1.00em}
\coqdoctac{try} \coqdoctac{destruct} \coqdocvar{X};\coqdoceol
\coqdocindent{1.00em}
\coqdoctac{try} \coqdoctac{destruct} \coqdocvar{Y};\coqdoceol
\coqdocindent{1.00em}
\coqdoctac{simpl};\coqdoceol
\coqdocindent{1.00em}
\coqdoctac{reflexivity}.\coqdoceol
\coqdocemptyline
\coqdocnoindent
\coqdockw{Ltac} \coqdocvar{reduce3} \coqdocvar{X} \coqdocvar{Y} \coqdocvar{Z} :=\coqdoceol
\coqdocindent{1.00em}
\coqdoctac{try} \coqdoctac{destruct} \coqdocvar{X};\coqdoceol
\coqdocindent{1.00em}
\coqdoctac{try} \coqdoctac{destruct} \coqdocvar{Y};\coqdoceol
\coqdocindent{1.00em}
\coqdoctac{try} \coqdoctac{destruct} \coqdocvar{Z};\coqdoceol
\coqdocindent{1.00em}
\coqdoctac{simpl};\coqdoceol
\coqdocindent{1.00em}
\coqdoctac{reflexivity}.\coqdoceol
\coqdocemptyline
\end{coqdoccode}


Now we state Theorem 1a (plus over circuits is commutative) and prove
it in a straightforward fashion.


\begin{coqdoccode}
\coqdocemptyline
\coqdocnoindent
\coqdockw{Theorem} \coqdef{computational.plus comm}{plus\_comm}{\coqdoclemma{plus\_comm}} : \coqdockw{\ensuremath{\forall}} (\coqdocvar{x} \coqdocvar{y} : \coqref{computational.circuit}{\coqdocinductive{circuit}}),\coqdoceol
\coqdocindent{2.00em}
\coqdocvariable{x} \coqref{computational.::x '+' x}{\coqdocnotation{+}} \coqdocvariable{y} \coqexternalref{:type scope:x '=' x}{http://coq.inria.fr/distrib/8.6/stdlib/Coq.Init.Logic}{\coqdocnotation{=}} \coqdocvariable{y} \coqref{computational.::x '+' x}{\coqdocnotation{+}} \coqdocvariable{x}.\coqdoceol
\coqdocnoindent
\coqdockw{Proof}.\coqdoceol
\coqdocindent{1.00em}
\coqdoctac{intros} \coqdocvar{X} \coqdocvar{Y}.\coqdoceol
\coqdocindent{1.00em}
\coqdocvar{reduce2} \coqdocvar{X} \coqdocvar{Y}.\coqdoceol
\coqdocnoindent
\coqdockw{Qed}.\coqdoceol
\coqdocemptyline
\end{coqdoccode}


Next we state Theorem 1b (times over circuits is commutative) and
prove it.


\begin{coqdoccode}
\coqdocemptyline
\coqdocnoindent
\coqdockw{Theorem} \coqdef{computational.times comm}{times\_comm}{\coqdoclemma{times\_comm}} : \coqdockw{\ensuremath{\forall}} (\coqdocvar{x} \coqdocvar{y} : \coqref{computational.circuit}{\coqdocinductive{circuit}}),\coqdoceol
\coqdocindent{2.00em}
\coqdocvariable{x} \coqref{computational.::x '*' x}{\coqdocnotation{*}} \coqdocvariable{y} \coqexternalref{:type scope:x '=' x}{http://coq.inria.fr/distrib/8.6/stdlib/Coq.Init.Logic}{\coqdocnotation{=}} \coqdocvariable{y} \coqref{computational.::x '*' x}{\coqdocnotation{*}} \coqdocvariable{x}.\coqdoceol
\coqdocnoindent
\coqdockw{Proof}.\coqdoceol
\coqdocindent{1.00em}
\coqdoctac{intros} \coqdocvar{X} \coqdocvar{Y}.\coqdoceol
\coqdocindent{1.00em}
\coqdocvar{reduce2} \coqdocvar{X} \coqdocvar{Y}.\coqdoceol
\coqdocnoindent
\coqdockw{Qed}.\coqdoceol
\coqdocemptyline
\end{coqdoccode}


Next we prove Theorem 2a -- that plus is associative.


\begin{coqdoccode}
\coqdocemptyline
\coqdocnoindent
\coqdockw{Theorem} \coqdef{computational.plus assoc}{plus\_assoc}{\coqdoclemma{plus\_assoc}} : \coqdockw{\ensuremath{\forall}} (\coqdocvar{x} \coqdocvar{y} \coqdocvar{z} : \coqref{computational.circuit}{\coqdocinductive{circuit}}),\coqdoceol
\coqdocindent{2.00em}
\coqdocvariable{x} \coqref{computational.::x '+' x}{\coqdocnotation{+}} \coqref{computational.::x '+' x}{\coqdocnotation{(}}\coqdocvariable{y} \coqref{computational.::x '+' x}{\coqdocnotation{+}} \coqdocvariable{z}\coqref{computational.::x '+' x}{\coqdocnotation{)}} \coqexternalref{:type scope:x '=' x}{http://coq.inria.fr/distrib/8.6/stdlib/Coq.Init.Logic}{\coqdocnotation{=}} \coqref{computational.::x '+' x}{\coqdocnotation{(}}\coqdocvariable{x} \coqref{computational.::x '+' x}{\coqdocnotation{+}} \coqdocvariable{y}\coqref{computational.::x '+' x}{\coqdocnotation{)}} \coqref{computational.::x '+' x}{\coqdocnotation{+}} \coqdocvariable{z}.\coqdoceol
\coqdocnoindent
\coqdockw{Proof}.\coqdoceol
\coqdocindent{1.00em}
\coqdoctac{intros} \coqdocvar{X} \coqdocvar{Y} \coqdocvar{Z}.\coqdoceol
\coqdocindent{1.00em}
\coqdocvar{reduce3} \coqdocvar{X} \coqdocvar{Y} \coqdocvar{Z}.\coqdoceol
\coqdocnoindent
\coqdockw{Qed}.\coqdoceol
\coqdocemptyline
\end{coqdoccode}


Next we prove Theorem 2b -- that times is also associative.


\begin{coqdoccode}
\coqdocemptyline
\coqdocnoindent
\coqdockw{Theorem} \coqdef{computational.times assoc}{times\_assoc}{\coqdoclemma{times\_assoc}} : \coqdockw{\ensuremath{\forall}} (\coqdocvar{x} \coqdocvar{y} \coqdocvar{z} : \coqref{computational.circuit}{\coqdocinductive{circuit}}),\coqdoceol
\coqdocindent{2.00em}
\coqdocvariable{x} \coqref{computational.::x '*' x}{\coqdocnotation{*}} \coqref{computational.::x '*' x}{\coqdocnotation{(}}\coqdocvariable{y} \coqref{computational.::x '*' x}{\coqdocnotation{*}} \coqdocvariable{z}\coqref{computational.::x '*' x}{\coqdocnotation{)}} \coqexternalref{:type scope:x '=' x}{http://coq.inria.fr/distrib/8.6/stdlib/Coq.Init.Logic}{\coqdocnotation{=}} \coqref{computational.::x '*' x}{\coqdocnotation{(}}\coqdocvariable{x} \coqref{computational.::x '*' x}{\coqdocnotation{*}} \coqdocvariable{y}\coqref{computational.::x '*' x}{\coqdocnotation{)}} \coqref{computational.::x '*' x}{\coqdocnotation{*}} \coqdocvariable{z}.\coqdoceol
\coqdocnoindent
\coqdockw{Proof}.\coqdoceol
\coqdocindent{1.00em}
\coqdoctac{intros} \coqdocvar{X} \coqdocvar{Y} \coqdocvar{Z}.\coqdoceol
\coqdocindent{1.00em}
\coqdocvar{reduce3} \coqdocvar{X} \coqdocvar{Y} \coqdocvar{Z}.\coqdoceol
\coqdocnoindent
\coqdockw{Qed}.\coqdoceol
\coqdocemptyline
\end{coqdoccode}


Next, we prove Theorem 3a -- that times is distributive.


\begin{coqdoccode}
\coqdocemptyline
\coqdocnoindent
\coqdockw{Theorem} \coqdef{computational.times dist}{times\_dist}{\coqdoclemma{times\_dist}} : \coqdockw{\ensuremath{\forall}} (\coqdocvar{x} \coqdocvar{y} \coqdocvar{z} : \coqref{computational.circuit}{\coqdocinductive{circuit}}),\coqdoceol
\coqdocindent{2.00em}
\coqdocvariable{x} \coqref{computational.::x '*' x}{\coqdocnotation{*}} \coqref{computational.::x '*' x}{\coqdocnotation{(}}\coqdocvariable{y} \coqref{computational.::x '+' x}{\coqdocnotation{+}} \coqdocvariable{z}\coqref{computational.::x '*' x}{\coqdocnotation{)}} \coqexternalref{:type scope:x '=' x}{http://coq.inria.fr/distrib/8.6/stdlib/Coq.Init.Logic}{\coqdocnotation{=}} \coqref{computational.::x '+' x}{\coqdocnotation{(}}\coqdocvariable{x} \coqref{computational.::x '*' x}{\coqdocnotation{*}} \coqdocvariable{y}\coqref{computational.::x '+' x}{\coqdocnotation{)}} \coqref{computational.::x '+' x}{\coqdocnotation{+}} \coqref{computational.::x '+' x}{\coqdocnotation{(}}\coqdocvariable{x} \coqref{computational.::x '*' x}{\coqdocnotation{*}} \coqdocvariable{z}\coqref{computational.::x '+' x}{\coqdocnotation{)}}.\coqdoceol
\coqdocnoindent
\coqdockw{Proof}.\coqdoceol
\coqdocindent{1.00em}
\coqdoctac{intros} \coqdocvar{X} \coqdocvar{Y} \coqdocvar{Z}.\coqdoceol
\coqdocindent{1.00em}
\coqdocvar{reduce3} \coqdocvar{X} \coqdocvar{Y} \coqdocvar{Z}.\coqdoceol
\coqdocnoindent
\coqdockw{Qed}.\coqdoceol
\coqdocemptyline
\end{coqdoccode}


Next, we prove Theorem 3b, that plus is also distributive.


\begin{coqdoccode}
\coqdocemptyline
\coqdocnoindent
\coqdockw{Theorem} \coqdef{computational.plus dist}{plus\_dist}{\coqdoclemma{plus\_dist}} : \coqdockw{\ensuremath{\forall}} (\coqdocvar{x} \coqdocvar{y} \coqdocvar{z} : \coqref{computational.circuit}{\coqdocinductive{circuit}}),\coqdoceol
\coqdocindent{2.00em}
\coqdocvariable{x} \coqref{computational.::x '+' x}{\coqdocnotation{+}} \coqref{computational.::x '+' x}{\coqdocnotation{(}}\coqdocvariable{y} \coqref{computational.::x '*' x}{\coqdocnotation{*}} \coqdocvariable{z}\coqref{computational.::x '+' x}{\coqdocnotation{)}} \coqexternalref{:type scope:x '=' x}{http://coq.inria.fr/distrib/8.6/stdlib/Coq.Init.Logic}{\coqdocnotation{=}} \coqref{computational.::x '*' x}{\coqdocnotation{(}}\coqdocvariable{x} \coqref{computational.::x '+' x}{\coqdocnotation{+}} \coqdocvariable{y}\coqref{computational.::x '*' x}{\coqdocnotation{)}} \coqref{computational.::x '*' x}{\coqdocnotation{*}} \coqref{computational.::x '*' x}{\coqdocnotation{(}}\coqdocvariable{x} \coqref{computational.::x '+' x}{\coqdocnotation{+}} \coqdocvariable{z}\coqref{computational.::x '*' x}{\coqdocnotation{)}}.\coqdoceol
\coqdocnoindent
\coqdockw{Proof}.\coqdoceol
\coqdocindent{1.00em}
\coqdoctac{intros} \coqdocvar{X} \coqdocvar{Y} \coqdocvar{Z}.\coqdoceol
\coqdocindent{1.00em}
\coqdocvar{reduce3} \coqdocvar{X} \coqdocvar{Y} \coqdocvar{Z}.\coqdoceol
\coqdocnoindent
\coqdockw{Qed}.\coqdoceol
\coqdocemptyline
\end{coqdoccode}


Now we get to Theorem 4a which is a theorem about how times works when
the first argument is the value open.


\begin{coqdoccode}
\coqdocemptyline
\coqdocnoindent
\coqdockw{Theorem} \coqdef{computational.open times x}{open\_times\_x}{\coqdoclemma{open\_times\_x}} : \coqdockw{\ensuremath{\forall}} (\coqdocvar{x} : \coqref{computational.circuit}{\coqdocinductive{circuit}}),\coqdoceol
\coqdocindent{2.00em}
\coqref{computational.open}{\coqdocconstructor{open}} \coqref{computational.::x '*' x}{\coqdocnotation{*}} \coqdocvariable{x} \coqexternalref{:type scope:x '=' x}{http://coq.inria.fr/distrib/8.6/stdlib/Coq.Init.Logic}{\coqdocnotation{=}} \coqdocvariable{x}.\coqdoceol
\coqdocnoindent
\coqdockw{Proof}.\coqdoceol
\coqdocindent{1.00em}
\coqdoctac{intros} \coqdocvar{X}.\coqdoceol
\coqdocindent{1.00em}
\coqdocvar{reduce1} \coqdocvar{X}.\coqdoceol
\coqdocnoindent
\coqdockw{Qed}.\coqdoceol
\coqdocemptyline
\end{coqdoccode}


And we prove Theorem 4b which is a theorem about how plus works when
the first argument is the value closed.


\begin{coqdoccode}
\coqdocemptyline
\coqdocnoindent
\coqdockw{Theorem} \coqdef{computational.closed plus x}{closed\_plus\_x}{\coqdoclemma{closed\_plus\_x}} : \coqdockw{\ensuremath{\forall}} (\coqdocvar{x}: \coqref{computational.circuit}{\coqdocinductive{circuit}}),\coqdoceol
\coqdocindent{2.00em}
\coqref{computational.closed}{\coqdocconstructor{closed}} \coqref{computational.::x '+' x}{\coqdocnotation{+}} \coqdocvariable{x} \coqexternalref{:type scope:x '=' x}{http://coq.inria.fr/distrib/8.6/stdlib/Coq.Init.Logic}{\coqdocnotation{=}} \coqdocvariable{x}.\coqdoceol
\coqdocnoindent
\coqdockw{Proof}.\coqdoceol
\coqdocindent{1.00em}
\coqdoctac{intros} \coqdocvar{X}.\coqdoceol
\coqdocindent{1.00em}
\coqdocvar{reduce1} \coqdocvar{X}.\coqdoceol
\coqdocnoindent
\coqdockw{Qed}.\coqdoceol
\coqdocemptyline
\end{coqdoccode}


Next, Theorem 5a asserts a relationship of plus when the first
argument is the value open.


\begin{coqdoccode}
\coqdocemptyline
\coqdocnoindent
\coqdockw{Theorem} \coqdef{computational.open plus x}{open\_plus\_x}{\coqdoclemma{open\_plus\_x}} : \coqdockw{\ensuremath{\forall}} (\coqdocvar{x}: \coqref{computational.circuit}{\coqdocinductive{circuit}}),\coqdoceol
\coqdocindent{2.00em}
\coqref{computational.open}{\coqdocconstructor{open}} \coqref{computational.::x '+' x}{\coqdocnotation{+}} \coqdocvariable{x} \coqexternalref{:type scope:x '=' x}{http://coq.inria.fr/distrib/8.6/stdlib/Coq.Init.Logic}{\coqdocnotation{=}} \coqref{computational.open}{\coqdocconstructor{open}}.\coqdoceol
\coqdocnoindent
\coqdockw{Proof}.\coqdoceol
\coqdocindent{1.00em}
\coqdoctac{intros} \coqdocvar{X}.\coqdoceol
\coqdocindent{1.00em}
\coqdocvar{reduce1} \coqdocvar{X}.\coqdoceol
\coqdocnoindent
\coqdockw{Qed}.\coqdoceol
\coqdocemptyline
\end{coqdoccode}


And Theorem 5b asserts a relationship of times when the first argument
is closed.


\begin{coqdoccode}
\coqdocemptyline
\coqdocnoindent
\coqdockw{Theorem} \coqdef{computational.closed times x}{closed\_times\_x}{\coqdoclemma{closed\_times\_x}} : \coqdockw{\ensuremath{\forall}} (\coqdocvar{x} : \coqref{computational.circuit}{\coqdocinductive{circuit}}),\coqdoceol
\coqdocindent{2.00em}
\coqref{computational.closed}{\coqdocconstructor{closed}} \coqref{computational.::x '*' x}{\coqdocnotation{*}} \coqdocvariable{x} \coqexternalref{:type scope:x '=' x}{http://coq.inria.fr/distrib/8.6/stdlib/Coq.Init.Logic}{\coqdocnotation{=}} \coqref{computational.closed}{\coqdocconstructor{closed}}.\coqdoceol
\coqdocnoindent
\coqdockw{Proof}.\coqdoceol
\coqdocindent{1.00em}
\coqdoctac{intros} \coqdocvar{X}.\coqdoceol
\coqdocindent{1.00em}
\coqdocvar{reduce1} \coqdocvar{X}.\coqdoceol
\coqdocnoindent
\coqdockw{Qed}.\coqdoceol
\coqdocemptyline
\end{coqdoccode}
\section{Negation Theorems}




Theorem 6a asserts the behavior when a circuit and its negative are
connected in series.  As you might expect, this always results in an
open circuit.


\begin{coqdoccode}
\coqdocemptyline
\coqdocnoindent
\coqdockw{Theorem} \coqdef{computational.plus neg}{plus\_neg}{\coqdoclemma{plus\_neg}} : \coqdockw{\ensuremath{\forall}} (\coqdocvar{x} : \coqref{computational.circuit}{\coqdocinductive{circuit}}),\coqdoceol
\coqdocindent{2.00em}
\coqdocvariable{x} \coqref{computational.::x '+' x}{\coqdocnotation{+}} \coqref{computational.::x '+' x}{\coqdocnotation{(}}\coqref{computational.negation}{\coqdocdefinition{negation}} \coqdocvariable{x}\coqref{computational.::x '+' x}{\coqdocnotation{)}} \coqexternalref{:type scope:x '=' x}{http://coq.inria.fr/distrib/8.6/stdlib/Coq.Init.Logic}{\coqdocnotation{=}} \coqref{computational.open}{\coqdocconstructor{open}}.\coqdoceol
\coqdocnoindent
\coqdockw{Proof}.\coqdoceol
\coqdocindent{1.00em}
\coqdoctac{intros} \coqdocvar{X}.\coqdoceol
\coqdocindent{1.00em}
\coqdocvar{reduce1} \coqdocvar{X}.\coqdoceol
\coqdocnoindent
\coqdockw{Qed}.\coqdoceol
\coqdocemptyline
\end{coqdoccode}


And Theorem 6b specifies what happens when you connect a circuit and
its negative in parallel.  As you would expect, the circuit is always
closed in this case.


\begin{coqdoccode}
\coqdocemptyline
\coqdocnoindent
\coqdockw{Theorem} \coqdef{computational.times neg}{times\_neg}{\coqdoclemma{times\_neg}} : \coqdockw{\ensuremath{\forall}} (\coqdocvar{x} : \coqref{computational.circuit}{\coqdocinductive{circuit}}),\coqdoceol
\coqdocindent{2.00em}
\coqdocvariable{x} \coqref{computational.::x '*' x}{\coqdocnotation{*}} \coqref{computational.::x '*' x}{\coqdocnotation{(}}\coqref{computational.negation}{\coqdocdefinition{negation}} \coqdocvariable{x}\coqref{computational.::x '*' x}{\coqdocnotation{)}} \coqexternalref{:type scope:x '=' x}{http://coq.inria.fr/distrib/8.6/stdlib/Coq.Init.Logic}{\coqdocnotation{=}} \coqref{computational.closed}{\coqdocconstructor{closed}}.\coqdoceol
\coqdocnoindent
\coqdockw{Proof}.\coqdoceol
\coqdocindent{1.00em}
\coqdoctac{intros} \coqdocvar{X}.\coqdoceol
\coqdocindent{1.00em}
\coqdocvar{reduce1} \coqdocvar{X}.\coqdoceol
\coqdocnoindent
\coqdockw{Qed}.\coqdoceol
\coqdocemptyline
\end{coqdoccode}


Theorems 7a and 7b specify what happens when you negate the specific
values of open or closed.  These are quite simple and are formalized
below.


\begin{coqdoccode}
\coqdocemptyline
\coqdocnoindent
\coqdockw{Theorem} \coqdef{computational.closed neg}{closed\_neg}{\coqdoclemma{closed\_neg}} :\coqdoceol
\coqdocindent{1.00em}
\coqref{computational.negation}{\coqdocdefinition{negation}} \coqref{computational.closed}{\coqdocconstructor{closed}} \coqexternalref{:type scope:x '=' x}{http://coq.inria.fr/distrib/8.6/stdlib/Coq.Init.Logic}{\coqdocnotation{=}} \coqref{computational.open}{\coqdocconstructor{open}}.\coqdoceol
\coqdocnoindent
\coqdockw{Proof}.\coqdoceol
\coqdocindent{1.00em}
\coqdocvar{reduce1} \coqdocvar{X}.\coqdoceol
\coqdocnoindent
\coqdockw{Qed}.\coqdoceol
\coqdocemptyline
\coqdocnoindent
\coqdockw{Theorem} \coqdef{computational.open neg}{open\_neg}{\coqdoclemma{open\_neg}} :\coqdoceol
\coqdocindent{1.00em}
\coqref{computational.negation}{\coqdocdefinition{negation}} \coqref{computational.open}{\coqdocconstructor{open}} \coqexternalref{:type scope:x '=' x}{http://coq.inria.fr/distrib/8.6/stdlib/Coq.Init.Logic}{\coqdocnotation{=}} \coqref{computational.closed}{\coqdocconstructor{closed}}.\coqdoceol
\coqdocnoindent
\coqdockw{Proof}.\coqdoceol
\coqdocindent{1.00em}
\coqdocvar{reduce1} \coqdocvar{X}.\coqdoceol
\coqdocnoindent
\coqdockw{Qed}.\coqdoceol
\coqdocemptyline
\end{coqdoccode}


Theorem 8 specifies what happens when you take the negative of the
negative of a circuit.  As expected one gets the original circuit
back.


\begin{coqdoccode}
\coqdocemptyline
\coqdocnoindent
\coqdockw{Theorem} \coqdef{computational.double neg}{double\_neg}{\coqdoclemma{double\_neg}} : \coqdockw{\ensuremath{\forall}} (\coqdocvar{x} : \coqref{computational.circuit}{\coqdocinductive{circuit}}),\coqdoceol
\coqdocindent{2.00em}
\coqref{computational.negation}{\coqdocdefinition{negation}} (\coqref{computational.negation}{\coqdocdefinition{negation}} \coqdocvariable{x}) \coqexternalref{:type scope:x '=' x}{http://coq.inria.fr/distrib/8.6/stdlib/Coq.Init.Logic}{\coqdocnotation{=}} \coqdocvariable{x}.\coqdoceol
\coqdocnoindent
\coqdockw{Proof}.\coqdoceol
\coqdocindent{1.00em}
\coqdoctac{intros} \coqdocvar{X}.\coqdoceol
\coqdocindent{1.00em}
\coqdocvar{reduce1} \coqdocvar{X}.\coqdoceol
\coqdocnoindent
\coqdockw{Qed}.\coqdoceol
\coqdocemptyline
\end{coqdoccode}
\section{Equivalence to Calculus of Propositions}




Claude then describes how the algebra defined above is equivalent to
propositional logic.  He does this by showing an equivalence
between the algebra above and E.V. Huntington's formulation of
symbolic logic.  This formulation has 6 postulates and postulates 1,
2, 3, and 4 are clearly met without proof.  Postulates 5 and 6 of
E.V. Huntington's formulation are proved below.


\begin{coqdoccode}
\coqdocemptyline
\coqdocnoindent
\coqdockw{Theorem} \coqdef{computational.plus same}{plus\_same}{\coqdoclemma{plus\_same}} : \coqdockw{\ensuremath{\forall}} (\coqdocvar{x} \coqdocvar{y} : \coqref{computational.circuit}{\coqdocinductive{circuit}}),\coqdoceol
\coqdocindent{2.00em}
\coqdocvariable{x} \coqexternalref{:type scope:x '=' x}{http://coq.inria.fr/distrib/8.6/stdlib/Coq.Init.Logic}{\coqdocnotation{=}} \coqdocvariable{y} \coqexternalref{:type scope:x '->' x}{http://coq.inria.fr/distrib/8.6/stdlib/Coq.Init.Logic}{\coqdocnotation{\ensuremath{\rightarrow}}}\coqdoceol
\coqdocindent{2.00em}
\coqdocvariable{x} \coqref{computational.::x '+' x}{\coqdocnotation{+}} \coqdocvariable{y} \coqexternalref{:type scope:x '=' x}{http://coq.inria.fr/distrib/8.6/stdlib/Coq.Init.Logic}{\coqdocnotation{=}} \coqdocvariable{x}.\coqdoceol
\coqdocnoindent
\coqdockw{Proof}.\coqdoceol
\coqdocindent{1.00em}
\coqdoctac{intros} \coqdocvar{X} \coqdocvar{Y}.\coqdoceol
\coqdocindent{1.00em}
\coqdoctac{intros} \coqdocvar{H}.\coqdoceol
\coqdocindent{1.00em}
\coqdoctac{rewrite} \ensuremath{\rightarrow} \coqdocvar{H}.\coqdoceol
\coqdocindent{1.00em}
\coqdocvar{reduce1} \coqdocvar{Y}.\coqdoceol
\coqdocnoindent
\coqdockw{Qed}.\coqdoceol
\coqdocemptyline
\coqdocnoindent
\coqdockw{Theorem} \coqdef{computational.dist neg}{dist\_neg}{\coqdoclemma{dist\_neg}} : \coqdockw{\ensuremath{\forall}} (\coqdocvar{x} \coqdocvar{y} : \coqref{computational.circuit}{\coqdocinductive{circuit}}),\coqdoceol
\coqdocindent{2.00em}
\coqref{computational.::x '+' x}{\coqdocnotation{(}}\coqdocvariable{x} \coqref{computational.::x '*' x}{\coqdocnotation{*}} \coqdocvariable{y}\coqref{computational.::x '+' x}{\coqdocnotation{)}} \coqref{computational.::x '+' x}{\coqdocnotation{+}} \coqref{computational.::x '+' x}{\coqdocnotation{(}}\coqdocvariable{x} \coqref{computational.::x '*' x}{\coqdocnotation{*}} \coqref{computational.::x '*' x}{\coqdocnotation{(}}\coqref{computational.negation}{\coqdocdefinition{negation}} \coqdocvariable{y}\coqref{computational.::x '*' x}{\coqdocnotation{)}}\coqref{computational.::x '+' x}{\coqdocnotation{)}} \coqexternalref{:type scope:x '=' x}{http://coq.inria.fr/distrib/8.6/stdlib/Coq.Init.Logic}{\coqdocnotation{=}} \coqdocvariable{x}.\coqdoceol
\coqdocnoindent
\coqdockw{Proof}.\coqdoceol
\coqdocindent{1.00em}
\coqdoctac{intros} \coqdocvar{X} \coqdocvar{Y}.\coqdoceol
\coqdocindent{1.00em}
\coqdocvar{reduce2} \coqdocvar{X} \coqdocvar{Y}.\coqdoceol
\coqdocnoindent
\coqdockw{Qed}.\coqdoceol
\coqdocemptyline
\end{coqdoccode}


We can, for completeness also prove the definition mentioned in
proposition 6.


\begin{coqdoccode}
\coqdocemptyline
\coqdocnoindent
\coqdockw{Theorem} \coqdef{computational.dist neg defn}{dist\_neg\_defn}{\coqdoclemma{dist\_neg\_defn}} : \coqdockw{\ensuremath{\forall}} (\coqdocvar{x} \coqdocvar{y} : \coqref{computational.circuit}{\coqdocinductive{circuit}}),\coqdoceol
\coqdocindent{2.00em}
\coqexternalref{:type scope:x '=' x}{http://coq.inria.fr/distrib/8.6/stdlib/Coq.Init.Logic}{\coqdocnotation{(}}\coqdocvariable{x} \coqref{computational.::x '*' x}{\coqdocnotation{*}} \coqdocvariable{y}\coqexternalref{:type scope:x '=' x}{http://coq.inria.fr/distrib/8.6/stdlib/Coq.Init.Logic}{\coqdocnotation{)}} \coqexternalref{:type scope:x '=' x}{http://coq.inria.fr/distrib/8.6/stdlib/Coq.Init.Logic}{\coqdocnotation{=}} \coqref{computational.negation}{\coqdocdefinition{negation}} (\coqref{computational.::x '+' x}{\coqdocnotation{(}}\coqref{computational.negation}{\coqdocdefinition{negation}} \coqdocvariable{x}\coqref{computational.::x '+' x}{\coqdocnotation{)}} \coqref{computational.::x '+' x}{\coqdocnotation{+}}\coqdoceol
\coqdocindent{12.00em}
\coqref{computational.::x '+' x}{\coqdocnotation{(}}\coqref{computational.negation}{\coqdocdefinition{negation}} \coqdocvariable{y}\coqref{computational.::x '+' x}{\coqdocnotation{)}}).\coqdoceol
\coqdocnoindent
\coqdockw{Proof}.\coqdoceol
\coqdocindent{1.00em}
\coqdoctac{intros} \coqdocvar{X} \coqdocvar{Y}.\coqdoceol
\coqdocindent{1.00em}
\coqdocvar{reduce2} \coqdocvar{X} \coqdocvar{Y}.\coqdoceol
\coqdocnoindent
\coqdockw{Qed}.\coqdoceol
\coqdocemptyline
\end{coqdoccode}
\section{A Proof of De Morgans Law}




Once this equivalence between the circuit algebra and propositional
logic is shown, it is possible to bring over powerful theorems from
propositional logic into our new algebra.  We will begin by proving De
Morgan's theorem.  This is Theorem 9.


While the thesis asserts these theorems for an arbitrary number of
variables, we will only illustrate proofs for two and three variables.


\begin{coqdoccode}
\coqdocemptyline
\coqdocnoindent
\coqdockw{Theorem} \coqdef{computational.demorgan 9a 2}{demorgan\_9a\_2}{\coqdoclemma{demorgan\_9a\_2}} : \coqdockw{\ensuremath{\forall}} (\coqdocvar{x} \coqdocvar{y}: \coqref{computational.circuit}{\coqdocinductive{circuit}}),\coqdoceol
\coqdocindent{2.00em}
\coqref{computational.negation}{\coqdocdefinition{negation}} (\coqdocvariable{x} \coqref{computational.::x '+' x}{\coqdocnotation{+}} \coqdocvariable{y}) \coqexternalref{:type scope:x '=' x}{http://coq.inria.fr/distrib/8.6/stdlib/Coq.Init.Logic}{\coqdocnotation{=}}\coqdoceol
\coqdocindent{2.00em}
\coqexternalref{:type scope:x '=' x}{http://coq.inria.fr/distrib/8.6/stdlib/Coq.Init.Logic}{\coqdocnotation{(}} \coqref{computational.::x '*' x}{\coqdocnotation{(}}\coqref{computational.negation}{\coqdocdefinition{negation}} \coqdocvariable{x}\coqref{computational.::x '*' x}{\coqdocnotation{)}} \coqref{computational.::x '*' x}{\coqdocnotation{*}}\coqdoceol
\coqdocindent{3.00em}
\coqref{computational.::x '*' x}{\coqdocnotation{(}}\coqref{computational.negation}{\coqdocdefinition{negation}} \coqdocvariable{y}\coqref{computational.::x '*' x}{\coqdocnotation{)}} \coqexternalref{:type scope:x '=' x}{http://coq.inria.fr/distrib/8.6/stdlib/Coq.Init.Logic}{\coqdocnotation{)}}.\coqdoceol
\coqdocnoindent
\coqdockw{Proof}.\coqdoceol
\coqdocindent{1.00em}
\coqdoctac{intros} \coqdocvar{X} \coqdocvar{Y}.\coqdoceol
\coqdocindent{1.00em}
\coqdocvar{reduce2} \coqdocvar{X} \coqdocvar{Y}.\coqdoceol
\coqdocnoindent
\coqdockw{Qed}.\coqdoceol
\coqdocemptyline
\coqdocnoindent
\coqdockw{Theorem} \coqdef{computational.demorgan 9a 3}{demorgan\_9a\_3}{\coqdoclemma{demorgan\_9a\_3}} : \coqdockw{\ensuremath{\forall}} (\coqdocvar{x} \coqdocvar{y} \coqdocvar{z}: \coqref{computational.circuit}{\coqdocinductive{circuit}}),\coqdoceol
\coqdocindent{2.00em}
\coqref{computational.negation}{\coqdocdefinition{negation}} (\coqdocvariable{x} \coqref{computational.::x '+' x}{\coqdocnotation{+}} \coqdocvariable{y} \coqref{computational.::x '+' x}{\coqdocnotation{+}} \coqdocvariable{z}) \coqexternalref{:type scope:x '=' x}{http://coq.inria.fr/distrib/8.6/stdlib/Coq.Init.Logic}{\coqdocnotation{=}}\coqdoceol
\coqdocindent{2.00em}
\coqexternalref{:type scope:x '=' x}{http://coq.inria.fr/distrib/8.6/stdlib/Coq.Init.Logic}{\coqdocnotation{(}} \coqref{computational.::x '*' x}{\coqdocnotation{(}}\coqref{computational.negation}{\coqdocdefinition{negation}} \coqdocvariable{x}\coqref{computational.::x '*' x}{\coqdocnotation{)}} \coqref{computational.::x '*' x}{\coqdocnotation{*}}\coqdoceol
\coqdocindent{3.00em}
\coqref{computational.::x '*' x}{\coqdocnotation{(}}\coqref{computational.negation}{\coqdocdefinition{negation}} \coqdocvariable{y}\coqref{computational.::x '*' x}{\coqdocnotation{)}} \coqref{computational.::x '*' x}{\coqdocnotation{*}}\coqdoceol
\coqdocindent{3.00em}
\coqref{computational.::x '*' x}{\coqdocnotation{(}}\coqref{computational.negation}{\coqdocdefinition{negation}} \coqdocvariable{z}\coqref{computational.::x '*' x}{\coqdocnotation{)}} \coqexternalref{:type scope:x '=' x}{http://coq.inria.fr/distrib/8.6/stdlib/Coq.Init.Logic}{\coqdocnotation{)}}.\coqdoceol
\coqdocnoindent
\coqdockw{Proof}.\coqdoceol
\coqdocindent{1.00em}
\coqdoctac{intros} \coqdocvar{X} \coqdocvar{Y} \coqdocvar{Z}.\coqdoceol
\coqdocindent{1.00em}
\coqdocvar{reduce3} \coqdocvar{X} \coqdocvar{Y} \coqdocvar{Z}.\coqdoceol
\coqdocnoindent
\coqdockw{Qed}.\coqdoceol
\coqdocemptyline
\end{coqdoccode}


And we will prove De Morgan's theorem over times over two and three
variables.  This is Theorem 9b.


\begin{coqdoccode}
\coqdocemptyline
\coqdocnoindent
\coqdockw{Theorem} \coqdef{computational.demorgan 9b 2}{demorgan\_9b\_2}{\coqdoclemma{demorgan\_9b\_2}} : \coqdockw{\ensuremath{\forall}} (\coqdocvar{x} \coqdocvar{y}: \coqref{computational.circuit}{\coqdocinductive{circuit}}),\coqdoceol
\coqdocindent{2.00em}
\coqref{computational.negation}{\coqdocdefinition{negation}} (\coqdocvariable{x} \coqref{computational.::x '*' x}{\coqdocnotation{*}} \coqdocvariable{y}) \coqexternalref{:type scope:x '=' x}{http://coq.inria.fr/distrib/8.6/stdlib/Coq.Init.Logic}{\coqdocnotation{=}}\coqdoceol
\coqdocindent{2.00em}
\coqexternalref{:type scope:x '=' x}{http://coq.inria.fr/distrib/8.6/stdlib/Coq.Init.Logic}{\coqdocnotation{(}} \coqref{computational.::x '+' x}{\coqdocnotation{(}}\coqref{computational.negation}{\coqdocdefinition{negation}} \coqdocvariable{x}\coqref{computational.::x '+' x}{\coqdocnotation{)}} \coqref{computational.::x '+' x}{\coqdocnotation{+}}\coqdoceol
\coqdocindent{3.00em}
\coqref{computational.::x '+' x}{\coqdocnotation{(}}\coqref{computational.negation}{\coqdocdefinition{negation}} \coqdocvariable{y}\coqref{computational.::x '+' x}{\coqdocnotation{)}} \coqexternalref{:type scope:x '=' x}{http://coq.inria.fr/distrib/8.6/stdlib/Coq.Init.Logic}{\coqdocnotation{)}}.\coqdoceol
\coqdocnoindent
\coqdockw{Proof}.\coqdoceol
\coqdocindent{1.00em}
\coqdoctac{intros} \coqdocvar{X} \coqdocvar{Y}.\coqdoceol
\coqdocindent{1.00em}
\coqdocvar{reduce2} \coqdocvar{X} \coqdocvar{Y}.\coqdoceol
\coqdocnoindent
\coqdockw{Qed}.\coqdoceol
\coqdocemptyline
\coqdocnoindent
\coqdockw{Theorem} \coqdef{computational.demorgan 9b 3}{demorgan\_9b\_3}{\coqdoclemma{demorgan\_9b\_3}} : \coqdockw{\ensuremath{\forall}} (\coqdocvar{x} \coqdocvar{y} \coqdocvar{z}: \coqref{computational.circuit}{\coqdocinductive{circuit}}),\coqdoceol
\coqdocindent{2.00em}
\coqref{computational.negation}{\coqdocdefinition{negation}} (\coqdocvariable{x} \coqref{computational.::x '*' x}{\coqdocnotation{*}} \coqdocvariable{y} \coqref{computational.::x '*' x}{\coqdocnotation{*}} \coqdocvariable{z}) \coqexternalref{:type scope:x '=' x}{http://coq.inria.fr/distrib/8.6/stdlib/Coq.Init.Logic}{\coqdocnotation{=}}\coqdoceol
\coqdocindent{2.00em}
\coqexternalref{:type scope:x '=' x}{http://coq.inria.fr/distrib/8.6/stdlib/Coq.Init.Logic}{\coqdocnotation{(}} \coqref{computational.::x '+' x}{\coqdocnotation{(}}\coqref{computational.negation}{\coqdocdefinition{negation}} \coqdocvariable{x}\coqref{computational.::x '+' x}{\coqdocnotation{)}} \coqref{computational.::x '+' x}{\coqdocnotation{+}}\coqdoceol
\coqdocindent{3.00em}
\coqref{computational.::x '+' x}{\coqdocnotation{(}}\coqref{computational.negation}{\coqdocdefinition{negation}} \coqdocvariable{y}\coqref{computational.::x '+' x}{\coqdocnotation{)}} \coqref{computational.::x '+' x}{\coqdocnotation{+}}\coqdoceol
\coqdocindent{3.00em}
\coqref{computational.::x '+' x}{\coqdocnotation{(}}\coqref{computational.negation}{\coqdocdefinition{negation}} \coqdocvariable{z}\coqref{computational.::x '+' x}{\coqdocnotation{)}} \coqexternalref{:type scope:x '=' x}{http://coq.inria.fr/distrib/8.6/stdlib/Coq.Init.Logic}{\coqdocnotation{)}}.\coqdoceol
\coqdocnoindent
\coqdockw{Proof}.\coqdoceol
\coqdocindent{1.00em}
\coqdoctac{intros} \coqdocvar{X} \coqdocvar{Y} \coqdocvar{Z}.\coqdoceol
\coqdocindent{1.00em}
\coqdocvar{reduce3} \coqdocvar{X} \coqdocvar{Y} \coqdocvar{Z}.\coqdoceol
\coqdocnoindent
\coqdockw{Qed}.\coqdoceol
\coqdocemptyline
\end{coqdoccode}
\section{Onward to Taylor Series}




Claude then starts the discussion of how to specify arbitrary
functions in the circuit algebra.  He starts by illustrating the
capability to expand an arbitrary function into a Taylor series
expansion.


In order to complete these proofs we introduce more Ltac tactic
machinery.  At this point it will be important for us to be able to
leverage many of the above theorems in subsequent proofs.  We
encapsulate these theorems into a new set of Ltac tactics.  The
tactics are shown below.


\begin{coqdoccode}
\coqdocemptyline
\coqdocnoindent
\coqdockw{Ltac} \coqdocvar{wham} :=\coqdoceol
\coqdocindent{1.00em}
\coqdoctac{try} \coqdoctac{repeat} ( (\coqdoctac{rewrite} \ensuremath{\rightarrow} \coqref{computational.closed times x}{\coqdoclemma{closed\_times\_x}};\coqdoceol
\coqdocindent{8.00em}
\coqdoctac{rewrite} \ensuremath{\rightarrow} \coqref{computational.closed neg}{\coqdoclemma{closed\_neg}};\coqdoceol
\coqdocindent{8.00em}
\coqdoctac{rewrite} \ensuremath{\rightarrow} \coqref{computational.open times x}{\coqdoclemma{open\_times\_x}}) ||\coqdoceol
\coqdocindent{7.50em}
(\coqdoctac{rewrite} \ensuremath{\rightarrow} \coqref{computational.open neg}{\coqdoclemma{open\_neg}};\coqdoceol
\coqdocindent{8.00em}
\coqdoctac{rewrite} \ensuremath{\rightarrow} \coqref{computational.closed times x}{\coqdoclemma{closed\_times\_x}};\coqdoceol
\coqdocindent{8.00em}
\coqdoctac{rewrite} \ensuremath{\rightarrow} \coqref{computational.open times x}{\coqdoclemma{open\_times\_x}}) ||\coqdoceol
\coqdocindent{7.50em}
(\coqdoctac{rewrite} \ensuremath{\rightarrow} \coqref{computational.open neg}{\coqdoclemma{open\_neg}}) ||\coqdoceol
\coqdocindent{7.50em}
(\coqdoctac{rewrite} \ensuremath{\rightarrow} \coqref{computational.closed neg}{\coqdoclemma{closed\_neg}}) ).\coqdoceol
\coqdocemptyline
\coqdocnoindent
\coqdockw{Ltac} \coqdocvar{open\_plus\_bam} :=\coqdoceol
\coqdocindent{1.00em}
\coqdoctac{try} ( (\coqdoctac{rewrite} \ensuremath{\rightarrow} \coqref{computational.open plus x}{\coqdoclemma{open\_plus\_x}}) ||\coqdoceol
\coqdocindent{4.00em}
(\coqdoctac{rewrite} \coqref{computational.plus comm}{\coqdoclemma{plus\_comm}};\coqdoceol
\coqdocindent{4.50em}
\coqdoctac{rewrite} \coqref{computational.open plus x}{\coqdoclemma{open\_plus\_x}}) ).\coqdoceol
\coqdocemptyline
\coqdocnoindent
\coqdockw{Ltac} \coqdocvar{closed\_plus\_bam} :=\coqdoceol
\coqdocindent{1.00em}
\coqdoctac{try} ( (\coqdoctac{rewrite} \ensuremath{\rightarrow} \coqref{computational.closed plus x}{\coqdoclemma{closed\_plus\_x}}) ||\coqdoceol
\coqdocindent{4.00em}
(\coqdoctac{rewrite} \coqref{computational.plus comm}{\coqdoclemma{plus\_comm}};\coqdoceol
\coqdocindent{4.50em}
\coqdoctac{rewrite} \coqref{computational.closed plus x}{\coqdoclemma{closed\_plus\_x}}) ).\coqdoceol
\coqdocemptyline
\coqdocnoindent
\coqdockw{Ltac} \coqdocvar{open\_times\_bam} :=\coqdoceol
\coqdocindent{1.00em}
\coqdoctac{try} ( (\coqdoctac{rewrite} \ensuremath{\rightarrow} \coqref{computational.open times x}{\coqdoclemma{open\_times\_x}}) ||\coqdoceol
\coqdocindent{4.00em}
(\coqdoctac{rewrite} \coqref{computational.times comm}{\coqdoclemma{times\_comm}};\coqdoceol
\coqdocindent{4.50em}
\coqdoctac{rewrite} \coqref{computational.open times x}{\coqdoclemma{open\_times\_x}}) ).\coqdoceol
\coqdocemptyline
\coqdocnoindent
\coqdockw{Ltac} \coqdocvar{closed\_times\_bam} :=\coqdoceol
\coqdocindent{1.00em}
\coqdoctac{try} ( (\coqdoctac{rewrite} \ensuremath{\rightarrow} \coqref{computational.closed times x}{\coqdoclemma{closed\_times\_x}}) ||\coqdoceol
\coqdocindent{4.00em}
(\coqdoctac{rewrite} \coqref{computational.times comm}{\coqdoclemma{times\_comm}};\coqdoceol
\coqdocindent{4.50em}
\coqdoctac{rewrite} \coqref{computational.closed times x}{\coqdoclemma{closed\_times\_x}}) ).\coqdoceol
\coqdocemptyline
\coqdocnoindent
\coqdockw{Ltac} \coqdocvar{bam} :=\coqdoceol
\coqdocindent{1.00em}
\coqdoctac{try} \coqdoctac{repeat} (\coqdocvar{closed\_plus\_bam} ||\coqdoceol
\coqdocindent{7.00em}
\coqdocvar{open\_plus\_bam} ||\coqdoceol
\coqdocindent{7.00em}
\coqdocvar{closed\_times\_bam} ||\coqdoceol
\coqdocindent{7.00em}
\coqdocvar{open\_times\_bam}).\coqdoceol
\coqdocemptyline
\coqdocnoindent
\coqdockw{Ltac} \coqdocvar{wham\_bam\_1} \coqdocvar{X} :=\coqdoceol
\coqdocindent{1.00em}
\coqdoctac{try} (\coqdoctac{destruct} \coqdocvar{X};\coqdoceol
\coqdocindent{3.50em}
\coqdocvar{wham}; \coqdocvar{bam};\coqdoceol
\coqdocindent{3.50em}
\coqdoctac{reflexivity}).\coqdoceol
\coqdocemptyline
\coqdocnoindent
\coqdockw{Ltac} \coqdocvar{wham\_bam\_2} \coqdocvar{X} \coqdocvar{Y} :=\coqdoceol
\coqdocindent{1.00em}
\coqdoctac{try} (\coqdoctac{destruct} \coqdocvar{X}, \coqdocvar{Y};\coqdoceol
\coqdocindent{3.50em}
\coqdocvar{wham}; \coqdocvar{bam};\coqdoceol
\coqdocindent{3.50em}
\coqdoctac{reflexivity}).\coqdoceol
\coqdocemptyline
\coqdocnoindent
\coqdockw{Ltac} \coqdocvar{wham\_bam\_3} \coqdocvar{X} \coqdocvar{Y} \coqdocvar{Z} :=\coqdoceol
\coqdocindent{1.00em}
\coqdoctac{try} (\coqdoctac{destruct} \coqdocvar{X}, \coqdocvar{Y}, \coqdocvar{Z};\coqdoceol
\coqdocindent{3.50em}
\coqdocvar{wham}; \coqdocvar{bam};\coqdoceol
\coqdocindent{3.50em}
\coqdoctac{reflexivity}).\coqdoceol
\coqdocemptyline
\coqdocnoindent
\coqdockw{Ltac} \coqdocvar{wham\_bam\_4} \coqdocvar{W} \coqdocvar{X} \coqdocvar{Y} \coqdocvar{Z} :=\coqdoceol
\coqdocindent{1.00em}
\coqdoctac{try} (\coqdoctac{destruct} \coqdocvar{W}, \coqdocvar{X}, \coqdocvar{Y}, \coqdocvar{Z};\coqdoceol
\coqdocindent{3.50em}
\coqdocvar{wham}; \coqdocvar{bam};\coqdoceol
\coqdocindent{3.50em}
\coqdoctac{reflexivity}).\coqdoceol
\coqdocemptyline
\coqdocnoindent
\coqdockw{Ltac} \coqdocvar{wham\_bam\_5} \coqdocvar{V} \coqdocvar{W} \coqdocvar{X} \coqdocvar{Y} \coqdocvar{Z} :=\coqdoceol
\coqdocindent{1.00em}
\coqdoctac{try} (\coqdoctac{destruct} \coqdocvar{V}, \coqdocvar{W}, \coqdocvar{X}, \coqdocvar{Y}, \coqdocvar{Z};\coqdoceol
\coqdocindent{3.50em}
\coqdocvar{wham}; \coqdocvar{bam};\coqdoceol
\coqdocindent{3.50em}
\coqdoctac{reflexivity}).\coqdoceol
\coqdocemptyline
\coqdocnoindent
\coqdockw{Ltac} \coqdocvar{wham\_bam\_6} \coqdocvar{S} \coqdocvar{V} \coqdocvar{W} \coqdocvar{X} \coqdocvar{Y} \coqdocvar{Z} :=\coqdoceol
\coqdocindent{1.00em}
\coqdoctac{try} (\coqdoctac{destruct} \coqdocvar{S}, \coqdocvar{V}, \coqdocvar{W}, \coqdocvar{X}, \coqdocvar{Y}, \coqdocvar{Z};\coqdoceol
\coqdocindent{3.50em}
\coqdocvar{wham}; \coqdocvar{bam};\coqdoceol
\coqdocindent{3.50em}
\coqdoctac{reflexivity}).\coqdoceol
\coqdocemptyline
\end{coqdoccode}


And now we have the appropriate machinery in place to be able to prove
the Taylor series expansion on two variables shown in Theorems 10a and
10b, here called taylorA and taylorB respectively.


\begin{coqdoccode}
\coqdocemptyline
\coqdocnoindent
\coqdockw{Theorem} \coqdef{computational.taylorA}{taylorA}{\coqdoclemma{taylorA}} :\coqdoceol
\coqdocindent{1.00em}
\coqdockw{\ensuremath{\forall}} (\coqdocvar{x}  \coqdocvar{y}: \coqref{computational.circuit}{\coqdocinductive{circuit}}),\coqdoceol
\coqdocindent{2.00em}
\coqdockw{\ensuremath{\forall}} (\coqdocvar{f} : \coqref{computational.circuit}{\coqdocinductive{circuit}} \coqexternalref{:type scope:x '->' x}{http://coq.inria.fr/distrib/8.6/stdlib/Coq.Init.Logic}{\coqdocnotation{\ensuremath{\rightarrow}}} \coqref{computational.circuit}{\coqdocinductive{circuit}} \coqexternalref{:type scope:x '->' x}{http://coq.inria.fr/distrib/8.6/stdlib/Coq.Init.Logic}{\coqdocnotation{\ensuremath{\rightarrow}}} \coqref{computational.circuit}{\coqdocinductive{circuit}}),\coqdoceol
\coqdocindent{3.00em}
\coqdocvariable{f} \coqdocvariable{x} \coqdocvariable{y} \coqexternalref{:type scope:x '=' x}{http://coq.inria.fr/distrib/8.6/stdlib/Coq.Init.Logic}{\coqdocnotation{=}}\coqdoceol
\coqdocindent{3.00em}
\coqexternalref{:type scope:x '=' x}{http://coq.inria.fr/distrib/8.6/stdlib/Coq.Init.Logic}{\coqdocnotation{(}} \coqref{computational.::x '+' x}{\coqdocnotation{(}}\coqdocvariable{x} \coqref{computational.::x '*' x}{\coqdocnotation{*}} \coqref{computational.::x '*' x}{\coqdocnotation{(}}\coqdocvariable{f} \coqref{computational.open}{\coqdocconstructor{open}} \coqdocvariable{y}\coqref{computational.::x '*' x}{\coqdocnotation{)}}\coqref{computational.::x '+' x}{\coqdocnotation{)}} \coqref{computational.::x '+' x}{\coqdocnotation{+}}\coqdoceol
\coqdocindent{4.00em}
\coqref{computational.::x '+' x}{\coqdocnotation{(}}\coqref{computational.::x '*' x}{\coqdocnotation{(}}\coqref{computational.negation}{\coqdocdefinition{negation}} \coqdocvariable{x}\coqref{computational.::x '*' x}{\coqdocnotation{)}} \coqref{computational.::x '*' x}{\coqdocnotation{*}} \coqref{computational.::x '*' x}{\coqdocnotation{(}}\coqdocvariable{f} \coqref{computational.closed}{\coqdocconstructor{closed}} \coqdocvariable{y}\coqref{computational.::x '*' x}{\coqdocnotation{)}}\coqref{computational.::x '+' x}{\coqdocnotation{)}} \coqexternalref{:type scope:x '=' x}{http://coq.inria.fr/distrib/8.6/stdlib/Coq.Init.Logic}{\coqdocnotation{)}}.\coqdoceol
\coqdocnoindent
\coqdockw{Proof}.\coqdoceol
\coqdocindent{1.00em}
\coqdoctac{intros} \coqdocvar{X} \coqdocvar{Y}.\coqdoceol
\coqdocindent{1.00em}
\coqdoctac{intros} \coqdocvar{F}.\coqdoceol
\coqdocindent{1.00em}
\coqdocvar{wham\_bam\_2} \coqdocvar{X} \coqdocvar{Y}.\coqdoceol
\coqdocnoindent
\coqdockw{Qed}.\coqdoceol
\coqdocemptyline
\coqdocnoindent
\coqdockw{Theorem} \coqdef{computational.taylorB}{taylorB}{\coqdoclemma{taylorB}} :\coqdoceol
\coqdocindent{1.00em}
\coqdockw{\ensuremath{\forall}} (\coqdocvar{x}  \coqdocvar{y}: \coqref{computational.circuit}{\coqdocinductive{circuit}}),\coqdoceol
\coqdocindent{1.00em}
\coqdockw{\ensuremath{\forall}} (\coqdocvar{f} : \coqref{computational.circuit}{\coqdocinductive{circuit}} \coqexternalref{:type scope:x '->' x}{http://coq.inria.fr/distrib/8.6/stdlib/Coq.Init.Logic}{\coqdocnotation{\ensuremath{\rightarrow}}} \coqref{computational.circuit}{\coqdocinductive{circuit}} \coqexternalref{:type scope:x '->' x}{http://coq.inria.fr/distrib/8.6/stdlib/Coq.Init.Logic}{\coqdocnotation{\ensuremath{\rightarrow}}} \coqref{computational.circuit}{\coqdocinductive{circuit}}),\coqdoceol
\coqdocindent{2.00em}
\coqdocvariable{f} \coqdocvariable{x} \coqdocvariable{y} \coqexternalref{:type scope:x '=' x}{http://coq.inria.fr/distrib/8.6/stdlib/Coq.Init.Logic}{\coqdocnotation{=}}\coqdoceol
\coqdocindent{2.00em}
\coqexternalref{:type scope:x '=' x}{http://coq.inria.fr/distrib/8.6/stdlib/Coq.Init.Logic}{\coqdocnotation{(}} \coqref{computational.::x '*' x}{\coqdocnotation{(}}\coqref{computational.::x '+' x}{\coqdocnotation{(}}\coqdocvariable{f} \coqref{computational.closed}{\coqdocconstructor{closed}} \coqdocvariable{y}\coqref{computational.::x '+' x}{\coqdocnotation{)}} \coqref{computational.::x '+' x}{\coqdocnotation{+}} \coqdocvariable{x}\coqref{computational.::x '*' x}{\coqdocnotation{)}} \coqref{computational.::x '*' x}{\coqdocnotation{*}}\coqdoceol
\coqdocindent{3.00em}
\coqref{computational.::x '*' x}{\coqdocnotation{(}}\coqref{computational.::x '+' x}{\coqdocnotation{(}}\coqdocvariable{f} \coqref{computational.open}{\coqdocconstructor{open}} \coqdocvariable{y}\coqref{computational.::x '+' x}{\coqdocnotation{)}} \coqref{computational.::x '+' x}{\coqdocnotation{+}} \coqref{computational.::x '+' x}{\coqdocnotation{(}}\coqref{computational.negation}{\coqdocdefinition{negation}} \coqdocvariable{x}\coqref{computational.::x '+' x}{\coqdocnotation{)}}\coqref{computational.::x '*' x}{\coqdocnotation{)}} \coqexternalref{:type scope:x '=' x}{http://coq.inria.fr/distrib/8.6/stdlib/Coq.Init.Logic}{\coqdocnotation{)}}.\coqdoceol
\coqdocnoindent
\coqdockw{Proof}.\coqdoceol
\coqdocindent{1.00em}
\coqdoctac{intros} \coqdocvar{X} \coqdocvar{Y}.\coqdoceol
\coqdocindent{1.00em}
\coqdoctac{intros} \coqdocvar{F}.\coqdoceol
\coqdocindent{1.00em}
\coqdocvar{wham\_bam\_2} \coqdocvar{X} \coqdocvar{Y}.\coqdoceol
\coqdocnoindent
\coqdockw{Qed}.\coqdoceol
\coqdocemptyline
\end{coqdoccode}


We continue with the expansion of the Taylor series to the second
variable as described in Theorems 11a and 11b.


\begin{coqdoccode}
\coqdocemptyline
\coqdocnoindent
\coqdockw{Theorem} \coqdef{computational.taylor11a}{taylor11a}{\coqdoclemma{taylor11a}} : \coqdockw{\ensuremath{\forall}} (\coqdocvar{x} \coqdocvar{y}: \coqref{computational.circuit}{\coqdocinductive{circuit}}),\coqdoceol
\coqdocindent{2.00em}
\coqdockw{\ensuremath{\forall}} (\coqdocvar{f} : \coqref{computational.circuit}{\coqdocinductive{circuit}} \coqexternalref{:type scope:x '->' x}{http://coq.inria.fr/distrib/8.6/stdlib/Coq.Init.Logic}{\coqdocnotation{\ensuremath{\rightarrow}}} \coqref{computational.circuit}{\coqdocinductive{circuit}} \coqexternalref{:type scope:x '->' x}{http://coq.inria.fr/distrib/8.6/stdlib/Coq.Init.Logic}{\coqdocnotation{\ensuremath{\rightarrow}}} \coqref{computational.circuit}{\coqdocinductive{circuit}}),\coqdoceol
\coqdocindent{3.00em}
\coqdocvariable{f} \coqdocvariable{x} \coqdocvariable{y} \coqexternalref{:type scope:x '=' x}{http://coq.inria.fr/distrib/8.6/stdlib/Coq.Init.Logic}{\coqdocnotation{=}}\coqdoceol
\coqdocindent{3.00em}
\coqref{computational.::x '+' x}{\coqdocnotation{(}} \coqref{computational.::x '*' x}{\coqdocnotation{(}}\coqdocvariable{x} \coqref{computational.::x '*' x}{\coqdocnotation{*}} \coqdocvariable{y}\coqref{computational.::x '*' x}{\coqdocnotation{)}} \coqref{computational.::x '*' x}{\coqdocnotation{*}}\coqdoceol
\coqdocindent{4.00em}
\coqref{computational.::x '*' x}{\coqdocnotation{(}}\coqdocvariable{f} \coqref{computational.open}{\coqdocconstructor{open}} \coqref{computational.open}{\coqdocconstructor{open}}\coqref{computational.::x '*' x}{\coqdocnotation{)}} \coqref{computational.::x '+' x}{\coqdocnotation{)}} \coqref{computational.::x '+' x}{\coqdocnotation{+}}\coqdoceol
\coqdocindent{3.00em}
\coqref{computational.::x '+' x}{\coqdocnotation{(}} \coqref{computational.::x '*' x}{\coqdocnotation{(}}\coqdocvariable{x} \coqref{computational.::x '*' x}{\coqdocnotation{*}} \coqref{computational.::x '*' x}{\coqdocnotation{(}}\coqref{computational.negation}{\coqdocdefinition{negation}} \coqdocvariable{y}\coqref{computational.::x '*' x}{\coqdocnotation{))}} \coqref{computational.::x '*' x}{\coqdocnotation{*}}\coqdoceol
\coqdocindent{4.00em}
\coqref{computational.::x '*' x}{\coqdocnotation{(}}\coqdocvariable{f} \coqref{computational.open}{\coqdocconstructor{open}} \coqref{computational.closed}{\coqdocconstructor{closed}}\coqref{computational.::x '*' x}{\coqdocnotation{)}} \coqref{computational.::x '+' x}{\coqdocnotation{)}} \coqref{computational.::x '+' x}{\coqdocnotation{+}}\coqdoceol
\coqdocindent{3.00em}
\coqref{computational.::x '+' x}{\coqdocnotation{(}} \coqref{computational.::x '*' x}{\coqdocnotation{((}}\coqref{computational.negation}{\coqdocdefinition{negation}} \coqdocvariable{x}\coqref{computational.::x '*' x}{\coqdocnotation{)}} \coqref{computational.::x '*' x}{\coqdocnotation{*}} \coqdocvariable{y}\coqref{computational.::x '*' x}{\coqdocnotation{)}} \coqref{computational.::x '*' x}{\coqdocnotation{*}}\coqdoceol
\coqdocindent{4.00em}
\coqref{computational.::x '*' x}{\coqdocnotation{(}}\coqdocvariable{f} \coqref{computational.closed}{\coqdocconstructor{closed}} \coqref{computational.open}{\coqdocconstructor{open}}\coqref{computational.::x '*' x}{\coqdocnotation{)}} \coqref{computational.::x '+' x}{\coqdocnotation{)}} \coqref{computational.::x '+' x}{\coqdocnotation{+}}\coqdoceol
\coqdocindent{3.00em}
\coqref{computational.::x '+' x}{\coqdocnotation{(}} \coqref{computational.::x '*' x}{\coqdocnotation{((}}\coqref{computational.negation}{\coqdocdefinition{negation}} \coqdocvariable{x}\coqref{computational.::x '*' x}{\coqdocnotation{)}} \coqref{computational.::x '*' x}{\coqdocnotation{*}}\coqdoceol
\coqdocindent{4.50em}
\coqref{computational.::x '*' x}{\coqdocnotation{(}}\coqref{computational.negation}{\coqdocdefinition{negation}} \coqdocvariable{y}\coqref{computational.::x '*' x}{\coqdocnotation{))}} \coqref{computational.::x '*' x}{\coqdocnotation{*}}\coqdoceol
\coqdocindent{4.00em}
\coqref{computational.::x '*' x}{\coqdocnotation{(}}\coqdocvariable{f} \coqref{computational.closed}{\coqdocconstructor{closed}} \coqref{computational.closed}{\coqdocconstructor{closed}}\coqref{computational.::x '*' x}{\coqdocnotation{)}} \coqref{computational.::x '+' x}{\coqdocnotation{)}}.\coqdoceol
\coqdocnoindent
\coqdockw{Proof}.\coqdoceol
\coqdocindent{1.00em}
\coqdoctac{intros} \coqdocvar{X} \coqdocvar{Y}.\coqdoceol
\coqdocindent{1.00em}
\coqdoctac{intros} \coqdocvar{F}.\coqdoceol
\coqdocindent{1.00em}
\coqdocvar{wham\_bam\_2} \coqdocvar{X} \coqdocvar{Y}.\coqdoceol
\coqdocnoindent
\coqdockw{Qed}.\coqdoceol
\coqdocemptyline
\coqdocnoindent
\coqdockw{Theorem} \coqdef{computational.taylor11b}{taylor11b}{\coqdoclemma{taylor11b}} : \coqdockw{\ensuremath{\forall}} (\coqdocvar{x} \coqdocvar{y}: \coqref{computational.circuit}{\coqdocinductive{circuit}}),\coqdoceol
\coqdocindent{2.00em}
\coqdockw{\ensuremath{\forall}} (\coqdocvar{f} : \coqref{computational.circuit}{\coqdocinductive{circuit}} \coqexternalref{:type scope:x '->' x}{http://coq.inria.fr/distrib/8.6/stdlib/Coq.Init.Logic}{\coqdocnotation{\ensuremath{\rightarrow}}} \coqref{computational.circuit}{\coqdocinductive{circuit}} \coqexternalref{:type scope:x '->' x}{http://coq.inria.fr/distrib/8.6/stdlib/Coq.Init.Logic}{\coqdocnotation{\ensuremath{\rightarrow}}} \coqref{computational.circuit}{\coqdocinductive{circuit}}),\coqdoceol
\coqdocindent{3.00em}
\coqdocvariable{f} \coqdocvariable{x} \coqdocvariable{y} \coqexternalref{:type scope:x '=' x}{http://coq.inria.fr/distrib/8.6/stdlib/Coq.Init.Logic}{\coqdocnotation{=}}\coqdoceol
\coqdocindent{3.00em}
\coqref{computational.::x '*' x}{\coqdocnotation{(}} \coqref{computational.::x '+' x}{\coqdocnotation{(}}\coqdocvariable{x} \coqref{computational.::x '+' x}{\coqdocnotation{+}} \coqdocvariable{y}\coqref{computational.::x '+' x}{\coqdocnotation{)}} \coqref{computational.::x '+' x}{\coqdocnotation{+}}\coqdoceol
\coqdocindent{4.00em}
\coqref{computational.::x '+' x}{\coqdocnotation{(}}\coqdocvariable{f} \coqref{computational.closed}{\coqdocconstructor{closed}} \coqref{computational.closed}{\coqdocconstructor{closed}}\coqref{computational.::x '+' x}{\coqdocnotation{)}} \coqref{computational.::x '*' x}{\coqdocnotation{)}} \coqref{computational.::x '*' x}{\coqdocnotation{*}}\coqdoceol
\coqdocindent{3.00em}
\coqref{computational.::x '*' x}{\coqdocnotation{(}} \coqref{computational.::x '+' x}{\coqdocnotation{(}}\coqdocvariable{x} \coqref{computational.::x '+' x}{\coqdocnotation{+}} \coqref{computational.::x '+' x}{\coqdocnotation{(}}\coqref{computational.negation}{\coqdocdefinition{negation}} \coqdocvariable{y}\coqref{computational.::x '+' x}{\coqdocnotation{))}} \coqref{computational.::x '+' x}{\coqdocnotation{+}}\coqdoceol
\coqdocindent{4.00em}
\coqref{computational.::x '+' x}{\coqdocnotation{(}}\coqdocvariable{f} \coqref{computational.closed}{\coqdocconstructor{closed}} \coqref{computational.open}{\coqdocconstructor{open}}\coqref{computational.::x '+' x}{\coqdocnotation{)}} \coqref{computational.::x '*' x}{\coqdocnotation{)}} \coqref{computational.::x '*' x}{\coqdocnotation{*}}\coqdoceol
\coqdocindent{3.00em}
\coqref{computational.::x '*' x}{\coqdocnotation{(}} \coqref{computational.::x '+' x}{\coqdocnotation{((}}\coqref{computational.negation}{\coqdocdefinition{negation}} \coqdocvariable{x}\coqref{computational.::x '+' x}{\coqdocnotation{)}} \coqref{computational.::x '+' x}{\coqdocnotation{+}} \coqdocvariable{y}\coqref{computational.::x '+' x}{\coqdocnotation{)}} \coqref{computational.::x '+' x}{\coqdocnotation{+}}\coqdoceol
\coqdocindent{4.00em}
\coqref{computational.::x '+' x}{\coqdocnotation{(}}\coqdocvariable{f} \coqref{computational.open}{\coqdocconstructor{open}} \coqref{computational.closed}{\coqdocconstructor{closed}}\coqref{computational.::x '+' x}{\coqdocnotation{)}} \coqref{computational.::x '*' x}{\coqdocnotation{)}} \coqref{computational.::x '*' x}{\coqdocnotation{*}}\coqdoceol
\coqdocindent{3.00em}
\coqref{computational.::x '*' x}{\coqdocnotation{(}} \coqref{computational.::x '+' x}{\coqdocnotation{((}}\coqref{computational.negation}{\coqdocdefinition{negation}} \coqdocvariable{x}\coqref{computational.::x '+' x}{\coqdocnotation{)}} \coqref{computational.::x '+' x}{\coqdocnotation{+}} \coqref{computational.::x '+' x}{\coqdocnotation{(}}\coqref{computational.negation}{\coqdocdefinition{negation}} \coqdocvariable{y}\coqref{computational.::x '+' x}{\coqdocnotation{))}} \coqref{computational.::x '+' x}{\coqdocnotation{+}}\coqdoceol
\coqdocindent{4.00em}
\coqref{computational.::x '+' x}{\coqdocnotation{(}}\coqdocvariable{f} \coqref{computational.open}{\coqdocconstructor{open}} \coqref{computational.open}{\coqdocconstructor{open}}\coqref{computational.::x '+' x}{\coqdocnotation{)}} \coqref{computational.::x '*' x}{\coqdocnotation{)}}.\coqdoceol
\coqdocnoindent
\coqdockw{Proof}.\coqdoceol
\coqdocindent{1.00em}
\coqdoctac{intros} \coqdocvar{X} \coqdocvar{Y}.\coqdoceol
\coqdocindent{1.00em}
\coqdoctac{intros} \coqdocvar{F}.\coqdoceol
\coqdocindent{1.00em}
\coqdocvar{wham\_bam\_2} \coqdocvar{X} \coqdocvar{Y}.\coqdoceol
\coqdocnoindent
\coqdockw{Qed}.\coqdoceol
\coqdocemptyline
\end{coqdoccode}


We skip the proofs of Theorem 12a and 12b as we have shown their
validity in the two variable case above.  We also leave the proof of
Theorem 13 to future work.


\begin{coqdoccode}
\coqdocemptyline
\coqdocemptyline
\end{coqdoccode}


At the end of the first paragraph on page 14, the thesis illustrates
an example of finding the negative of a particular function using the
generalization described in Theorem 13.  We prove it here, but do not
use the power of Theorem 13.  Instead we can use our simple custom
tactic with good results.


\begin{coqdoccode}
\coqdocemptyline
\coqdocnoindent
\coqdockw{Theorem} \coqdef{computational.example1}{example1}{\coqdoclemma{example1}} : \coqdockw{\ensuremath{\forall}} (\coqdocvar{w} \coqdocvar{x} \coqdocvar{y} \coqdocvar{z}: \coqref{computational.circuit}{\coqdocinductive{circuit}}),\coqdoceol
\coqdocindent{2.00em}
\coqref{computational.negation}{\coqdocdefinition{negation}} ( \coqdocvariable{x} \coqref{computational.::x '+' x}{\coqdocnotation{+}}\coqdoceol
\coqdocindent{7.50em}
\coqref{computational.::x '+' x}{\coqdocnotation{(}} \coqdocvariable{y} \coqref{computational.::x '*' x}{\coqdocnotation{*}}\coqdoceol
\coqdocindent{8.50em}
\coqref{computational.::x '*' x}{\coqdocnotation{(}}\coqdocvariable{z} \coqref{computational.::x '+' x}{\coqdocnotation{+}} \coqdocvariable{w} \coqref{computational.::x '*' x}{\coqdocnotation{*}} \coqref{computational.::x '*' x}{\coqdocnotation{(}}\coqref{computational.negation}{\coqdocdefinition{negation}} \coqdocvariable{x}\coqref{computational.::x '*' x}{\coqdocnotation{))}}\coqref{computational.::x '+' x}{\coqdocnotation{)}} ) \coqexternalref{:type scope:x '=' x}{http://coq.inria.fr/distrib/8.6/stdlib/Coq.Init.Logic}{\coqdocnotation{=}}\coqdoceol
\coqdocindent{2.00em}
\coqref{computational.::x '*' x}{\coqdocnotation{(}}\coqref{computational.negation}{\coqdocdefinition{negation}} \coqdocvariable{x}\coqref{computational.::x '*' x}{\coqdocnotation{)}} \coqref{computational.::x '*' x}{\coqdocnotation{*}}\coqdoceol
\coqdocindent{2.00em}
\coqref{computational.::x '*' x}{\coqdocnotation{(}} \coqref{computational.::x '+' x}{\coqdocnotation{(}}\coqref{computational.negation}{\coqdocdefinition{negation}} \coqdocvariable{y}\coqref{computational.::x '+' x}{\coqdocnotation{)}} \coqref{computational.::x '+' x}{\coqdocnotation{+}}\coqdoceol
\coqdocindent{3.00em}
\coqref{computational.::x '*' x}{\coqdocnotation{(}}\coqref{computational.negation}{\coqdocdefinition{negation}} \coqdocvariable{z}\coqref{computational.::x '*' x}{\coqdocnotation{)}} \coqref{computational.::x '*' x}{\coqdocnotation{*}}\coqdoceol
\coqdocindent{3.00em}
\coqref{computational.::x '*' x}{\coqdocnotation{(}}\coqref{computational.::x '+' x}{\coqdocnotation{(}}\coqref{computational.negation}{\coqdocdefinition{negation}} \coqdocvariable{w}\coqref{computational.::x '+' x}{\coqdocnotation{)}} \coqref{computational.::x '+' x}{\coqdocnotation{+}} \coqdocvariable{x}\coqref{computational.::x '*' x}{\coqdocnotation{)}} \coqref{computational.::x '*' x}{\coqdocnotation{)}}.\coqdoceol
\coqdocnoindent
\coqdockw{Proof}.\coqdoceol
\coqdocindent{1.00em}
\coqdoctac{intros} \coqdocvar{W} \coqdocvar{X} \coqdocvar{Y} \coqdocvar{Z}.\coqdoceol
\coqdocindent{1.00em}
\coqdocvar{wham\_bam\_4} \coqdocvar{W} \coqdocvar{X} \coqdocvar{Y} \coqdocvar{Z}.\coqdoceol
\coqdocnoindent
\coqdockw{Qed}.\coqdoceol
\coqdocemptyline
\end{coqdoccode}
\section{Simplification Theorems}




Next Claude presents Theorems 14-18, useful for simplifying
expressions.


\begin{coqdoccode}
\coqdocnoindent
\coqdockw{Theorem} \coqdef{computational.x plus x is x}{x\_plus\_x\_is\_x}{\coqdoclemma{x\_plus\_x\_is\_x}} : \coqdockw{\ensuremath{\forall}} (\coqdocvar{x}: \coqref{computational.circuit}{\coqdocinductive{circuit}}),\coqdoceol
\coqdocindent{2.00em}
\coqdocvariable{x} \coqref{computational.::x '+' x}{\coqdocnotation{+}} \coqdocvariable{x} \coqexternalref{:type scope:x '=' x}{http://coq.inria.fr/distrib/8.6/stdlib/Coq.Init.Logic}{\coqdocnotation{=}} \coqdocvariable{x}.\coqdoceol
\coqdocnoindent
\coqdockw{Proof}.\coqdoceol
\coqdocindent{1.00em}
\coqdoctac{intros} \coqdocvar{X}.\coqdoceol
\coqdocindent{1.00em}
\coqdocvar{wham\_bam\_1} \coqdocvar{X}.\coqdoceol
\coqdocnoindent
\coqdockw{Qed}.\coqdoceol
\coqdocemptyline
\coqdocnoindent
\coqdockw{Theorem} \coqdef{computational.x times x is x}{x\_times\_x\_is\_x}{\coqdoclemma{x\_times\_x\_is\_x}} : \coqdockw{\ensuremath{\forall}} (\coqdocvar{x}: \coqref{computational.circuit}{\coqdocinductive{circuit}}),\coqdoceol
\coqdocindent{2.00em}
\coqdocvariable{x} \coqref{computational.::x '*' x}{\coqdocnotation{*}} \coqdocvariable{x} \coqexternalref{:type scope:x '=' x}{http://coq.inria.fr/distrib/8.6/stdlib/Coq.Init.Logic}{\coqdocnotation{=}} \coqdocvariable{x}.\coqdoceol
\coqdocnoindent
\coqdockw{Proof}.\coqdoceol
\coqdocindent{1.00em}
\coqdoctac{intros} \coqdocvar{X}.\coqdoceol
\coqdocindent{1.00em}
\coqdocvar{wham\_bam\_1} \coqdocvar{X}.\coqdoceol
\coqdocnoindent
\coqdockw{Qed}.\coqdoceol
\coqdocemptyline
\coqdocnoindent
\coqdockw{Theorem} \coqdef{computational.x plus xy}{x\_plus\_xy}{\coqdoclemma{x\_plus\_xy}} : \coqdockw{\ensuremath{\forall}} (\coqdocvar{x} \coqdocvar{y}: \coqref{computational.circuit}{\coqdocinductive{circuit}}),\coqdoceol
\coqdocindent{2.00em}
\coqexternalref{:type scope:x '=' x}{http://coq.inria.fr/distrib/8.6/stdlib/Coq.Init.Logic}{\coqdocnotation{(}}\coqdocvariable{x} \coqref{computational.::x '+' x}{\coqdocnotation{+}} \coqref{computational.::x '+' x}{\coqdocnotation{(}}\coqdocvariable{x} \coqref{computational.::x '*' x}{\coqdocnotation{*}} \coqdocvariable{y}\coqref{computational.::x '+' x}{\coqdocnotation{)}}\coqexternalref{:type scope:x '=' x}{http://coq.inria.fr/distrib/8.6/stdlib/Coq.Init.Logic}{\coqdocnotation{)}} \coqexternalref{:type scope:x '=' x}{http://coq.inria.fr/distrib/8.6/stdlib/Coq.Init.Logic}{\coqdocnotation{=}} \coqdocvariable{x}.\coqdoceol
\coqdocnoindent
\coqdockw{Proof}.\coqdoceol
\coqdocindent{1.00em}
\coqdoctac{intros} \coqdocvar{X} \coqdocvar{Y}.\coqdoceol
\coqdocindent{1.00em}
\coqdocvar{wham\_bam\_2} \coqdocvar{X} \coqdocvar{Y}.\coqdoceol
\coqdocnoindent
\coqdockw{Qed}.\coqdoceol
\coqdocemptyline
\coqdocnoindent
\coqdockw{Theorem} \coqdef{computational.x x plus y}{x\_x\_plus\_y}{\coqdoclemma{x\_x\_plus\_y}} : \coqdockw{\ensuremath{\forall}} (\coqdocvar{x} \coqdocvar{y}: \coqref{computational.circuit}{\coqdocinductive{circuit}}),\coqdoceol
\coqdocindent{2.00em}
\coqdocvariable{x} \coqref{computational.::x '*' x}{\coqdocnotation{*}} \coqref{computational.::x '*' x}{\coqdocnotation{(}}\coqdocvariable{x} \coqref{computational.::x '+' x}{\coqdocnotation{+}} \coqdocvariable{y}\coqref{computational.::x '*' x}{\coqdocnotation{)}} \coqexternalref{:type scope:x '=' x}{http://coq.inria.fr/distrib/8.6/stdlib/Coq.Init.Logic}{\coqdocnotation{=}} \coqdocvariable{x}.\coqdoceol
\coqdocnoindent
\coqdockw{Proof}.\coqdoceol
\coqdocindent{1.00em}
\coqdoctac{intros} \coqdocvar{X} \coqdocvar{Y}.\coqdoceol
\coqdocindent{1.00em}
\coqdocvar{wham\_bam\_2} \coqdocvar{X} \coqdocvar{Y}.\coqdoceol
\coqdocnoindent
\coqdockw{Qed}.\coqdoceol
\coqdocemptyline
\coqdocnoindent
\coqdockw{Theorem} \coqdef{computational.theorem16a}{theorem16a}{\coqdoclemma{theorem16a}} : \coqdockw{\ensuremath{\forall}} (\coqdocvar{x} \coqdocvar{y} \coqdocvar{z}: \coqref{computational.circuit}{\coqdocinductive{circuit}}),\coqdoceol
\coqdocindent{2.00em}
\coqref{computational.::x '+' x}{\coqdocnotation{(}}\coqdocvariable{x} \coqref{computational.::x '*' x}{\coqdocnotation{*}} \coqdocvariable{y}\coqref{computational.::x '+' x}{\coqdocnotation{)}} \coqref{computational.::x '+' x}{\coqdocnotation{+}} \coqref{computational.::x '*' x}{\coqdocnotation{(}}\coqref{computational.negation}{\coqdocdefinition{negation}} \coqdocvariable{x}\coqref{computational.::x '*' x}{\coqdocnotation{)}} \coqref{computational.::x '*' x}{\coqdocnotation{*}} \coqdocvariable{z} \coqexternalref{:type scope:x '=' x}{http://coq.inria.fr/distrib/8.6/stdlib/Coq.Init.Logic}{\coqdocnotation{=}}\coqdoceol
\coqdocindent{2.00em}
\coqref{computational.::x '+' x}{\coqdocnotation{(}}\coqdocvariable{x} \coqref{computational.::x '*' x}{\coqdocnotation{*}} \coqdocvariable{y}\coqref{computational.::x '+' x}{\coqdocnotation{)}} \coqref{computational.::x '+' x}{\coqdocnotation{+}} \coqref{computational.::x '+' x}{\coqdocnotation{(}}\coqref{computational.::x '*' x}{\coqdocnotation{(}}\coqref{computational.negation}{\coqdocdefinition{negation}} \coqdocvariable{x}\coqref{computational.::x '*' x}{\coqdocnotation{)}} \coqref{computational.::x '*' x}{\coqdocnotation{*}} \coqdocvariable{z}\coqref{computational.::x '+' x}{\coqdocnotation{)}} \coqref{computational.::x '+' x}{\coqdocnotation{+}} \coqref{computational.::x '+' x}{\coqdocnotation{(}}\coqdocvariable{y} \coqref{computational.::x '*' x}{\coqdocnotation{*}} \coqdocvariable{z}\coqref{computational.::x '+' x}{\coqdocnotation{)}}.\coqdoceol
\coqdocnoindent
\coqdockw{Proof}.\coqdoceol
\coqdocindent{1.00em}
\coqdoctac{intros} \coqdocvar{X} \coqdocvar{Y} \coqdocvar{Z}.\coqdoceol
\coqdocindent{1.00em}
\coqdocvar{wham\_bam\_3} \coqdocvar{X} \coqdocvar{Y} \coqdocvar{Z}.\coqdoceol
\coqdocnoindent
\coqdockw{Qed}.\coqdoceol
\coqdocemptyline
\coqdocnoindent
\coqdockw{Theorem} \coqdef{computational.theorem16b}{theorem16b}{\coqdoclemma{theorem16b}} : \coqdockw{\ensuremath{\forall}} (\coqdocvar{x} \coqdocvar{y} \coqdocvar{z}: \coqref{computational.circuit}{\coqdocinductive{circuit}}),\coqdoceol
\coqdocindent{2.00em}
\coqref{computational.::x '*' x}{\coqdocnotation{(}}\coqdocvariable{x} \coqref{computational.::x '+' x}{\coqdocnotation{+}} \coqdocvariable{y}\coqref{computational.::x '*' x}{\coqdocnotation{)}} \coqref{computational.::x '*' x}{\coqdocnotation{*}} \coqref{computational.::x '*' x}{\coqdocnotation{(}}\coqref{computational.::x '+' x}{\coqdocnotation{(}}\coqref{computational.negation}{\coqdocdefinition{negation}} \coqdocvariable{x}\coqref{computational.::x '+' x}{\coqdocnotation{)}} \coqref{computational.::x '+' x}{\coqdocnotation{+}} \coqdocvariable{z}\coqref{computational.::x '*' x}{\coqdocnotation{)}} \coqexternalref{:type scope:x '=' x}{http://coq.inria.fr/distrib/8.6/stdlib/Coq.Init.Logic}{\coqdocnotation{=}}\coqdoceol
\coqdocindent{2.00em}
\coqref{computational.::x '*' x}{\coqdocnotation{(}}\coqdocvariable{x} \coqref{computational.::x '+' x}{\coqdocnotation{+}} \coqdocvariable{y}\coqref{computational.::x '*' x}{\coqdocnotation{)}} \coqref{computational.::x '*' x}{\coqdocnotation{*}} \coqref{computational.::x '*' x}{\coqdocnotation{(}}\coqref{computational.::x '+' x}{\coqdocnotation{(}}\coqref{computational.negation}{\coqdocdefinition{negation}} \coqdocvariable{x}\coqref{computational.::x '+' x}{\coqdocnotation{)}} \coqref{computational.::x '+' x}{\coqdocnotation{+}} \coqdocvariable{z}\coqref{computational.::x '*' x}{\coqdocnotation{)}} \coqref{computational.::x '*' x}{\coqdocnotation{*}} \coqref{computational.::x '*' x}{\coqdocnotation{(}}\coqdocvariable{y} \coqref{computational.::x '+' x}{\coqdocnotation{+}} \coqdocvariable{z}\coqref{computational.::x '*' x}{\coqdocnotation{)}}.\coqdoceol
\coqdocnoindent
\coqdockw{Proof}.\coqdoceol
\coqdocindent{1.00em}
\coqdoctac{intros} \coqdocvar{X} \coqdocvar{Y} \coqdocvar{Z}.\coqdoceol
\coqdocindent{1.00em}
\coqdocvar{wham\_bam\_3} \coqdocvar{X} \coqdocvar{Y} \coqdocvar{Z}.\coqdoceol
\coqdocnoindent
\coqdockw{Qed}.\coqdoceol
\coqdocemptyline
\coqdocnoindent
\coqdockw{Theorem} \coqdef{computational.theorem17a}{theorem17a}{\coqdoclemma{theorem17a}} : \coqdockw{\ensuremath{\forall}} (\coqdocvar{x}: \coqref{computational.circuit}{\coqdocinductive{circuit}}),\coqdoceol
\coqdocindent{2.00em}
\coqdockw{\ensuremath{\forall}} (\coqdocvar{f}: \coqref{computational.circuit}{\coqdocinductive{circuit}} \coqexternalref{:type scope:x '->' x}{http://coq.inria.fr/distrib/8.6/stdlib/Coq.Init.Logic}{\coqdocnotation{\ensuremath{\rightarrow}}} \coqref{computational.circuit}{\coqdocinductive{circuit}}),\coqdoceol
\coqdocindent{3.00em}
\coqdocvariable{x} \coqref{computational.::x '*' x}{\coqdocnotation{*}} \coqref{computational.::x '*' x}{\coqdocnotation{(}}\coqdocvariable{f} \coqdocvariable{x}\coqref{computational.::x '*' x}{\coqdocnotation{)}} \coqexternalref{:type scope:x '=' x}{http://coq.inria.fr/distrib/8.6/stdlib/Coq.Init.Logic}{\coqdocnotation{=}} \coqdocvariable{x} \coqref{computational.::x '*' x}{\coqdocnotation{*}} \coqref{computational.::x '*' x}{\coqdocnotation{(}}\coqdocvariable{f} \coqref{computational.open}{\coqdocconstructor{open}}\coqref{computational.::x '*' x}{\coqdocnotation{)}}.\coqdoceol
\coqdocnoindent
\coqdockw{Proof}.\coqdoceol
\coqdocindent{1.00em}
\coqdoctac{intros} \coqdocvar{X}.\coqdoceol
\coqdocindent{1.00em}
\coqdoctac{intros} \coqdocvar{F}.\coqdoceol
\coqdocindent{1.00em}
\coqdocvar{wham\_bam\_1} \coqdocvar{X}.\coqdoceol
\coqdocnoindent
\coqdockw{Qed}.\coqdoceol
\coqdocemptyline
\coqdocnoindent
\coqdockw{Theorem} \coqdef{computational.theorem17b}{theorem17b}{\coqdoclemma{theorem17b}} : \coqdockw{\ensuremath{\forall}} (\coqdocvar{x}: \coqref{computational.circuit}{\coqdocinductive{circuit}}),\coqdoceol
\coqdocindent{2.00em}
\coqdockw{\ensuremath{\forall}} (\coqdocvar{f}: \coqref{computational.circuit}{\coqdocinductive{circuit}} \coqexternalref{:type scope:x '->' x}{http://coq.inria.fr/distrib/8.6/stdlib/Coq.Init.Logic}{\coqdocnotation{\ensuremath{\rightarrow}}} \coqref{computational.circuit}{\coqdocinductive{circuit}}),\coqdoceol
\coqdocindent{3.00em}
\coqdocvariable{x} \coqref{computational.::x '+' x}{\coqdocnotation{+}} \coqref{computational.::x '+' x}{\coqdocnotation{(}}\coqdocvariable{f} \coqdocvariable{x}\coqref{computational.::x '+' x}{\coqdocnotation{)}} \coqexternalref{:type scope:x '=' x}{http://coq.inria.fr/distrib/8.6/stdlib/Coq.Init.Logic}{\coqdocnotation{=}} \coqdocvariable{x} \coqref{computational.::x '+' x}{\coqdocnotation{+}} \coqref{computational.::x '+' x}{\coqdocnotation{(}}\coqdocvariable{f} \coqref{computational.closed}{\coqdocconstructor{closed}}\coqref{computational.::x '+' x}{\coqdocnotation{)}}.\coqdoceol
\coqdocnoindent
\coqdockw{Proof}.\coqdoceol
\coqdocindent{1.00em}
\coqdoctac{intros} \coqdocvar{X}.\coqdoceol
\coqdocindent{1.00em}
\coqdoctac{intros} \coqdocvar{F}.\coqdoceol
\coqdocindent{1.00em}
\coqdocvar{wham\_bam\_1} \coqdocvar{X}.\coqdoceol
\coqdocnoindent
\coqdockw{Qed}.\coqdoceol
\coqdocemptyline
\coqdocnoindent
\coqdockw{Theorem} \coqdef{computational.theorem18a}{theorem18a}{\coqdoclemma{theorem18a}} : \coqdockw{\ensuremath{\forall}} (\coqdocvar{x}: \coqref{computational.circuit}{\coqdocinductive{circuit}}),\coqdoceol
\coqdocindent{2.00em}
\coqdockw{\ensuremath{\forall}} (\coqdocvar{f}: \coqref{computational.circuit}{\coqdocinductive{circuit}} \coqexternalref{:type scope:x '->' x}{http://coq.inria.fr/distrib/8.6/stdlib/Coq.Init.Logic}{\coqdocnotation{\ensuremath{\rightarrow}}} \coqref{computational.circuit}{\coqdocinductive{circuit}}),\coqdoceol
\coqdocindent{3.00em}
\coqref{computational.::x '*' x}{\coqdocnotation{(}}\coqref{computational.negation}{\coqdocdefinition{negation}} \coqdocvariable{x}\coqref{computational.::x '*' x}{\coqdocnotation{)}} \coqref{computational.::x '*' x}{\coqdocnotation{*}} \coqref{computational.::x '*' x}{\coqdocnotation{(}}\coqdocvariable{f} \coqdocvariable{x}\coqref{computational.::x '*' x}{\coqdocnotation{)}} \coqexternalref{:type scope:x '=' x}{http://coq.inria.fr/distrib/8.6/stdlib/Coq.Init.Logic}{\coqdocnotation{=}}\coqdoceol
\coqdocindent{3.00em}
\coqref{computational.::x '*' x}{\coqdocnotation{(}}\coqref{computational.negation}{\coqdocdefinition{negation}} \coqdocvariable{x}\coqref{computational.::x '*' x}{\coqdocnotation{)}} \coqref{computational.::x '*' x}{\coqdocnotation{*}} \coqref{computational.::x '*' x}{\coqdocnotation{(}}\coqdocvariable{f} \coqref{computational.closed}{\coqdocconstructor{closed}}\coqref{computational.::x '*' x}{\coqdocnotation{)}}.\coqdoceol
\coqdocnoindent
\coqdockw{Proof}.\coqdoceol
\coqdocindent{1.00em}
\coqdoctac{intros} \coqdocvar{X}.\coqdoceol
\coqdocindent{1.00em}
\coqdoctac{intros} \coqdocvar{F}.\coqdoceol
\coqdocindent{1.00em}
\coqdocvar{wham\_bam\_1} \coqdocvar{X}.\coqdoceol
\coqdocnoindent
\coqdockw{Qed}.\coqdoceol
\coqdocemptyline
\coqdocnoindent
\coqdockw{Theorem} \coqdef{computational.theorem18b}{theorem18b}{\coqdoclemma{theorem18b}} : \coqdockw{\ensuremath{\forall}} (\coqdocvar{x}: \coqref{computational.circuit}{\coqdocinductive{circuit}}),\coqdoceol
\coqdocindent{2.00em}
\coqdockw{\ensuremath{\forall}} (\coqdocvar{f}: \coqref{computational.circuit}{\coqdocinductive{circuit}} \coqexternalref{:type scope:x '->' x}{http://coq.inria.fr/distrib/8.6/stdlib/Coq.Init.Logic}{\coqdocnotation{\ensuremath{\rightarrow}}} \coqref{computational.circuit}{\coqdocinductive{circuit}}),\coqdoceol
\coqdocindent{3.00em}
\coqref{computational.::x '+' x}{\coqdocnotation{(}}\coqref{computational.negation}{\coqdocdefinition{negation}} \coqdocvariable{x}\coqref{computational.::x '+' x}{\coqdocnotation{)}} \coqref{computational.::x '+' x}{\coqdocnotation{+}} \coqref{computational.::x '+' x}{\coqdocnotation{(}}\coqdocvariable{f} \coqdocvariable{x}\coqref{computational.::x '+' x}{\coqdocnotation{)}} \coqexternalref{:type scope:x '=' x}{http://coq.inria.fr/distrib/8.6/stdlib/Coq.Init.Logic}{\coqdocnotation{=}}\coqdoceol
\coqdocindent{3.00em}
\coqref{computational.::x '+' x}{\coqdocnotation{(}}\coqref{computational.negation}{\coqdocdefinition{negation}} \coqdocvariable{x}\coqref{computational.::x '+' x}{\coqdocnotation{)}} \coqref{computational.::x '+' x}{\coqdocnotation{+}} \coqref{computational.::x '+' x}{\coqdocnotation{(}}\coqdocvariable{f} \coqref{computational.open}{\coqdocconstructor{open}}\coqref{computational.::x '+' x}{\coqdocnotation{)}}.\coqdoceol
\coqdocnoindent
\coqdockw{Proof}.\coqdoceol
\coqdocindent{1.00em}
\coqdoctac{intros} \coqdocvar{X}.\coqdoceol
\coqdocindent{1.00em}
\coqdoctac{intros} \coqdocvar{F}.\coqdoceol
\coqdocindent{1.00em}
\coqdocvar{wham\_bam\_1} \coqdocvar{X}.\coqdoceol
\coqdocnoindent
\coqdockw{Qed}.\coqdoceol
\coqdocemptyline
\end{coqdoccode}
\section{Series Parallel Example}




Figure 5 shows an example of an expression that represents a fairly
complex series parallel circuit.  The figure is first rendered into a
hindrance equation.  The equation is then manipulated into a simpler
form.  We prove that the transformation between Figure 5 and Figure 6
is correct.


\begin{coqdoccode}
\coqdocemptyline
\coqdocnoindent
\coqdockw{Theorem} \coqdef{computational.figure5}{figure5}{\coqdoclemma{figure5}} : \coqdockw{\ensuremath{\forall}} (\coqdocvar{s} \coqdocvar{v} \coqdocvar{w} \coqdocvar{x} \coqdocvar{y} \coqdocvar{z}: \coqref{computational.circuit}{\coqdocinductive{circuit}}),\coqdoceol
\coqdocindent{2.00em}
\coqdocvariable{w} \coqref{computational.::x '+' x}{\coqdocnotation{+}} \coqref{computational.::x '+' x}{\coqdocnotation{(}}\coqref{computational.::x '*' x}{\coqdocnotation{(}}\coqref{computational.negation}{\coqdocdefinition{negation}} \coqdocvariable{w}\coqref{computational.::x '*' x}{\coqdocnotation{)}} \coqref{computational.::x '*' x}{\coqdocnotation{*}} \coqref{computational.::x '*' x}{\coqdocnotation{(}}\coqdocvariable{x} \coqref{computational.::x '+' x}{\coqdocnotation{+}} \coqdocvariable{y}\coqref{computational.::x '*' x}{\coqdocnotation{)}}\coqref{computational.::x '+' x}{\coqdocnotation{)}} \coqref{computational.::x '+' x}{\coqdocnotation{+}}\coqdoceol
\coqdocindent{2.00em}
\coqref{computational.::x '*' x}{\coqdocnotation{(}}\coqdocvariable{x} \coqref{computational.::x '+' x}{\coqdocnotation{+}} \coqdocvariable{z}\coqref{computational.::x '*' x}{\coqdocnotation{)}} \coqref{computational.::x '*' x}{\coqdocnotation{*}} \coqref{computational.::x '*' x}{\coqdocnotation{(}}\coqdocvariable{s} \coqref{computational.::x '+' x}{\coqdocnotation{+}} \coqref{computational.::x '+' x}{\coqdocnotation{(}}\coqref{computational.negation}{\coqdocdefinition{negation}} \coqdocvariable{w}\coqref{computational.::x '+' x}{\coqdocnotation{)}} \coqref{computational.::x '+' x}{\coqdocnotation{+}} \coqdocvariable{z}\coqref{computational.::x '*' x}{\coqdocnotation{)}} \coqref{computational.::x '*' x}{\coqdocnotation{*}}\coqdoceol
\coqdocindent{2.00em}
\coqref{computational.::x '*' x}{\coqdocnotation{(}}\coqref{computational.::x '+' x}{\coqdocnotation{(}}\coqref{computational.negation}{\coqdocdefinition{negation}} \coqdocvariable{z}\coqref{computational.::x '+' x}{\coqdocnotation{)}} \coqref{computational.::x '+' x}{\coqdocnotation{+}} \coqdocvariable{y} \coqref{computational.::x '+' x}{\coqdocnotation{+}} \coqref{computational.::x '*' x}{\coqdocnotation{(}}\coqref{computational.negation}{\coqdocdefinition{negation}} \coqdocvariable{s}\coqref{computational.::x '*' x}{\coqdocnotation{)}} \coqref{computational.::x '*' x}{\coqdocnotation{*}} \coqdocvariable{v}\coqref{computational.::x '*' x}{\coqdocnotation{)}} \coqexternalref{:type scope:x '=' x}{http://coq.inria.fr/distrib/8.6/stdlib/Coq.Init.Logic}{\coqdocnotation{=}}\coqdoceol
\coqdocindent{2.00em}
\coqdocvariable{w} \coqref{computational.::x '+' x}{\coqdocnotation{+}} \coqdocvariable{x} \coqref{computational.::x '+' x}{\coqdocnotation{+}} \coqdocvariable{y} \coqref{computational.::x '+' x}{\coqdocnotation{+}} \coqdocvariable{z} \coqref{computational.::x '*' x}{\coqdocnotation{*}} \coqref{computational.::x '*' x}{\coqdocnotation{(}}\coqref{computational.negation}{\coqdocdefinition{negation}} \coqdocvariable{s}\coqref{computational.::x '*' x}{\coqdocnotation{)}} \coqref{computational.::x '*' x}{\coqdocnotation{*}} \coqdocvariable{v}.\coqdoceol
\coqdocnoindent
\coqdockw{Proof}.\coqdoceol
\coqdocindent{1.00em}
\coqdoctac{intros} \coqdocvar{S} \coqdocvar{V} \coqdocvar{W} \coqdocvar{X} \coqdocvar{Y} \coqdocvar{Z}.\coqdoceol
\coqdocindent{1.00em}
\coqdocvar{wham\_bam\_6} \coqdocvar{S} \coqdocvar{V} \coqdocvar{W} \coqdocvar{X} \coqdocvar{Y} \coqdocvar{Z}.\coqdoceol
\coqdocnoindent
\coqdockw{Qed}.\coqdoceol
\coqdocemptyline
\end{coqdoccode}
\section{Multi-Terminal Networks}




In this section, we discuss Section III of the thesis.  Section III
begins by illustrating two types of non-serial and non-parallel
networks: the delta and wye circuit configurations.


We tackle the equivalence of Figure 8, the delta to wye transformation
first.  The path from a to b in the delta configuration is r in
parallel with the both s and t in series.  This should be equivalent
to the path from a to b in the wye configuration, where the path is (r
in parallel with t) in series with (r in parallel with s).  The next
proof is a proof of this equivalence.  Then we provide proofs of the
equivalence of paths from b to c and from c to a.


\begin{coqdoccode}
\coqdocemptyline
\coqdocnoindent
\coqdockw{Theorem} \coqdef{computational.figure8 a to b}{figure8\_a\_to\_b}{\coqdoclemma{figure8\_a\_to\_b}} : \coqdockw{\ensuremath{\forall}} (\coqdocvar{r} \coqdocvar{s} \coqdocvar{t} : \coqref{computational.circuit}{\coqdocinductive{circuit}}),\coqdoceol
\coqdocindent{2.00em}
\coqdocvariable{r} \coqref{computational.::x '*' x}{\coqdocnotation{*}} \coqref{computational.::x '*' x}{\coqdocnotation{(}}\coqdocvariable{s} \coqref{computational.::x '+' x}{\coqdocnotation{+}} \coqdocvariable{t}\coqref{computational.::x '*' x}{\coqdocnotation{)}} \coqexternalref{:type scope:x '=' x}{http://coq.inria.fr/distrib/8.6/stdlib/Coq.Init.Logic}{\coqdocnotation{=}} \coqref{computational.::x '+' x}{\coqdocnotation{(}}\coqdocvariable{r} \coqref{computational.::x '*' x}{\coqdocnotation{*}} \coqdocvariable{t}\coqref{computational.::x '+' x}{\coqdocnotation{)}} \coqref{computational.::x '+' x}{\coqdocnotation{+}} \coqref{computational.::x '+' x}{\coqdocnotation{(}}\coqdocvariable{r} \coqref{computational.::x '*' x}{\coqdocnotation{*}} \coqdocvariable{s}\coqref{computational.::x '+' x}{\coqdocnotation{)}}.\coqdoceol
\coqdocnoindent
\coqdockw{Proof}.\coqdoceol
\coqdocindent{1.00em}
\coqdoctac{intros} \coqdocvar{R} \coqdocvar{S} \coqdocvar{T}.\coqdoceol
\coqdocindent{1.00em}
\coqdocvar{wham\_bam\_3} \coqdocvar{R} \coqdocvar{S} \coqdocvar{T}.\coqdoceol
\coqdocnoindent
\coqdockw{Qed}.\coqdoceol
\coqdocemptyline
\coqdocnoindent
\coqdockw{Theorem} \coqdef{computational.figure8 b to c}{figure8\_b\_to\_c}{\coqdoclemma{figure8\_b\_to\_c}} : \coqdockw{\ensuremath{\forall}} (\coqdocvar{r} \coqdocvar{s} \coqdocvar{t} : \coqref{computational.circuit}{\coqdocinductive{circuit}}),\coqdoceol
\coqdocindent{2.00em}
\coqdocvariable{s} \coqref{computational.::x '*' x}{\coqdocnotation{*}} \coqref{computational.::x '*' x}{\coqdocnotation{(}}\coqdocvariable{t} \coqref{computational.::x '+' x}{\coqdocnotation{+}} \coqdocvariable{r}\coqref{computational.::x '*' x}{\coqdocnotation{)}} \coqexternalref{:type scope:x '=' x}{http://coq.inria.fr/distrib/8.6/stdlib/Coq.Init.Logic}{\coqdocnotation{=}} \coqref{computational.::x '+' x}{\coqdocnotation{(}}\coqdocvariable{r} \coqref{computational.::x '*' x}{\coqdocnotation{*}} \coqdocvariable{s}\coqref{computational.::x '+' x}{\coqdocnotation{)}} \coqref{computational.::x '+' x}{\coqdocnotation{+}} \coqref{computational.::x '+' x}{\coqdocnotation{(}}\coqdocvariable{s} \coqref{computational.::x '*' x}{\coqdocnotation{*}} \coqdocvariable{t}\coqref{computational.::x '+' x}{\coqdocnotation{)}}.\coqdoceol
\coqdocnoindent
\coqdockw{Proof}.\coqdoceol
\coqdocindent{1.00em}
\coqdoctac{intros} \coqdocvar{R} \coqdocvar{S} \coqdocvar{T}.\coqdoceol
\coqdocindent{1.00em}
\coqdocvar{wham\_bam\_3} \coqdocvar{R} \coqdocvar{S} \coqdocvar{T}.\coqdoceol
\coqdocnoindent
\coqdockw{Qed}.\coqdoceol
\coqdocemptyline
\coqdocnoindent
\coqdockw{Theorem} \coqdef{computational.figure8 c to a}{figure8\_c\_to\_a}{\coqdoclemma{figure8\_c\_to\_a}} : \coqdockw{\ensuremath{\forall}} (\coqdocvar{r} \coqdocvar{s} \coqdocvar{t} : \coqref{computational.circuit}{\coqdocinductive{circuit}}),\coqdoceol
\coqdocindent{2.00em}
\coqdocvariable{t} \coqref{computational.::x '*' x}{\coqdocnotation{*}} \coqref{computational.::x '*' x}{\coqdocnotation{(}}\coqdocvariable{r} \coqref{computational.::x '+' x}{\coqdocnotation{+}} \coqdocvariable{s}\coqref{computational.::x '*' x}{\coqdocnotation{)}} \coqexternalref{:type scope:x '=' x}{http://coq.inria.fr/distrib/8.6/stdlib/Coq.Init.Logic}{\coqdocnotation{=}} \coqref{computational.::x '+' x}{\coqdocnotation{(}}\coqdocvariable{s} \coqref{computational.::x '*' x}{\coqdocnotation{*}} \coqdocvariable{t}\coqref{computational.::x '+' x}{\coqdocnotation{)}} \coqref{computational.::x '+' x}{\coqdocnotation{+}} \coqref{computational.::x '+' x}{\coqdocnotation{(}}\coqdocvariable{r} \coqref{computational.::x '*' x}{\coqdocnotation{*}} \coqdocvariable{t}\coqref{computational.::x '+' x}{\coqdocnotation{)}}.\coqdoceol
\coqdocnoindent
\coqdockw{Proof}.\coqdoceol
\coqdocindent{1.00em}
\coqdoctac{intros} \coqdocvar{R} \coqdocvar{S} \coqdocvar{T}.\coqdoceol
\coqdocindent{1.00em}
\coqdocvar{wham\_bam\_3} \coqdocvar{R} \coqdocvar{S} \coqdocvar{T}.\coqdoceol
\coqdocnoindent
\coqdockw{Qed}.\coqdoceol
\coqdocemptyline
\end{coqdoccode}


Next we tackle Figure 9, the wye to delta transformation.  We prove
this by proving each path independently as well.


\begin{coqdoccode}
\coqdocemptyline
\coqdocnoindent
\coqdockw{Theorem} \coqdef{computational.figure9 a to b}{figure9\_a\_to\_b}{\coqdoclemma{figure9\_a\_to\_b}} : \coqdockw{\ensuremath{\forall}} (\coqdocvar{r} \coqdocvar{s} \coqdocvar{t} : \coqref{computational.circuit}{\coqdocinductive{circuit}}),\coqdoceol
\coqdocindent{2.00em}
\coqdocvariable{r} \coqref{computational.::x '+' x}{\coqdocnotation{+}} \coqdocvariable{s} \coqexternalref{:type scope:x '=' x}{http://coq.inria.fr/distrib/8.6/stdlib/Coq.Init.Logic}{\coqdocnotation{=}} \coqref{computational.::x '*' x}{\coqdocnotation{(}}\coqdocvariable{r} \coqref{computational.::x '+' x}{\coqdocnotation{+}} \coqdocvariable{s}\coqref{computational.::x '*' x}{\coqdocnotation{)}} \coqref{computational.::x '*' x}{\coqdocnotation{*}} \coqref{computational.::x '*' x}{\coqdocnotation{(}} \coqref{computational.::x '+' x}{\coqdocnotation{(}}\coqdocvariable{t} \coqref{computational.::x '+' x}{\coqdocnotation{+}} \coqdocvariable{s}\coqref{computational.::x '+' x}{\coqdocnotation{)}} \coqref{computational.::x '+' x}{\coqdocnotation{+}} \coqref{computational.::x '+' x}{\coqdocnotation{(}}\coqdocvariable{r} \coqref{computational.::x '+' x}{\coqdocnotation{+}} \coqdocvariable{t}\coqref{computational.::x '+' x}{\coqdocnotation{)}} \coqref{computational.::x '*' x}{\coqdocnotation{)}}.\coqdoceol
\coqdocnoindent
\coqdockw{Proof}.\coqdoceol
\coqdocindent{1.00em}
\coqdoctac{intros} \coqdocvar{R} \coqdocvar{S} \coqdocvar{T}.\coqdoceol
\coqdocindent{1.00em}
\coqdocvar{wham\_bam\_3} \coqdocvar{R} \coqdocvar{S} \coqdocvar{T}.\coqdoceol
\coqdocnoindent
\coqdockw{Qed}.\coqdoceol
\coqdocemptyline
\coqdocnoindent
\coqdockw{Theorem} \coqdef{computational.figure9 b to c}{figure9\_b\_to\_c}{\coqdoclemma{figure9\_b\_to\_c}} : \coqdockw{\ensuremath{\forall}} (\coqdocvar{r} \coqdocvar{s} \coqdocvar{t} : \coqref{computational.circuit}{\coqdocinductive{circuit}}),\coqdoceol
\coqdocindent{2.00em}
\coqdocvariable{s} \coqref{computational.::x '+' x}{\coqdocnotation{+}} \coqdocvariable{t} \coqexternalref{:type scope:x '=' x}{http://coq.inria.fr/distrib/8.6/stdlib/Coq.Init.Logic}{\coqdocnotation{=}} \coqref{computational.::x '*' x}{\coqdocnotation{(}}\coqdocvariable{t} \coqref{computational.::x '+' x}{\coqdocnotation{+}} \coqdocvariable{s}\coqref{computational.::x '*' x}{\coqdocnotation{)}} \coqref{computational.::x '*' x}{\coqdocnotation{*}} \coqref{computational.::x '*' x}{\coqdocnotation{(}} \coqref{computational.::x '+' x}{\coqdocnotation{(}}\coqdocvariable{r} \coqref{computational.::x '+' x}{\coqdocnotation{+}} \coqdocvariable{s}\coqref{computational.::x '+' x}{\coqdocnotation{)}} \coqref{computational.::x '+' x}{\coqdocnotation{+}} \coqref{computational.::x '+' x}{\coqdocnotation{(}}\coqdocvariable{r} \coqref{computational.::x '+' x}{\coqdocnotation{+}} \coqdocvariable{t}\coqref{computational.::x '+' x}{\coqdocnotation{)}} \coqref{computational.::x '*' x}{\coqdocnotation{)}}.\coqdoceol
\coqdocnoindent
\coqdockw{Proof}.\coqdoceol
\coqdocindent{1.00em}
\coqdoctac{intros} \coqdocvar{R} \coqdocvar{S} \coqdocvar{T}.\coqdoceol
\coqdocindent{1.00em}
\coqdocvar{wham\_bam\_3} \coqdocvar{R} \coqdocvar{S} \coqdocvar{T}.\coqdoceol
\coqdocnoindent
\coqdockw{Qed}.\coqdoceol
\coqdocemptyline
\coqdocnoindent
\coqdockw{Theorem} \coqdef{computational.figure9 c to a}{figure9\_c\_to\_a}{\coqdoclemma{figure9\_c\_to\_a}} : \coqdockw{\ensuremath{\forall}} (\coqdocvar{r} \coqdocvar{s} \coqdocvar{t} : \coqref{computational.circuit}{\coqdocinductive{circuit}}),\coqdoceol
\coqdocindent{2.00em}
\coqdocvariable{t} \coqref{computational.::x '+' x}{\coqdocnotation{+}} \coqdocvariable{r} \coqexternalref{:type scope:x '=' x}{http://coq.inria.fr/distrib/8.6/stdlib/Coq.Init.Logic}{\coqdocnotation{=}} \coqref{computational.::x '*' x}{\coqdocnotation{(}}\coqdocvariable{r} \coqref{computational.::x '+' x}{\coqdocnotation{+}} \coqdocvariable{t}\coqref{computational.::x '*' x}{\coqdocnotation{)}} \coqref{computational.::x '*' x}{\coqdocnotation{*}} \coqref{computational.::x '*' x}{\coqdocnotation{(}} \coqref{computational.::x '+' x}{\coqdocnotation{(}}\coqdocvariable{r} \coqref{computational.::x '+' x}{\coqdocnotation{+}} \coqdocvariable{s}\coqref{computational.::x '+' x}{\coqdocnotation{)}} \coqref{computational.::x '+' x}{\coqdocnotation{+}} \coqref{computational.::x '+' x}{\coqdocnotation{(}}\coqdocvariable{t} \coqref{computational.::x '+' x}{\coqdocnotation{+}} \coqdocvariable{s}\coqref{computational.::x '+' x}{\coqdocnotation{)}} \coqref{computational.::x '*' x}{\coqdocnotation{)}}.\coqdoceol
\coqdocnoindent
\coqdockw{Proof}.\coqdoceol
\coqdocindent{1.00em}
\coqdoctac{intros} \coqdocvar{R} \coqdocvar{S} \coqdocvar{T}.\coqdoceol
\coqdocindent{1.00em}
\coqdocvar{wham\_bam\_3} \coqdocvar{R} \coqdocvar{S} \coqdocvar{T}.\coqdoceol
\coqdocnoindent
\coqdockw{Qed}.\coqdoceol
\coqdocemptyline
\end{coqdoccode}
\section{More Complex Transformations}




Figure 10 illustrates the transformation of a 5 point star to a fully
connected graph.  We prove the equivalence of the path from a to b in
Figure 10 and leave the proof of the other paths to future work.


\begin{coqdoccode}
\coqdocemptyline
\coqdocnoindent
\coqdockw{Theorem} \coqdef{computational.figure10 a to b}{figure10\_a\_to\_b}{\coqdoclemma{figure10\_a\_to\_b}} : \coqdockw{\ensuremath{\forall}} (\coqdocvar{r} \coqdocvar{s} \coqdocvar{t} \coqdocvar{u} \coqdocvar{v} : \coqref{computational.circuit}{\coqdocinductive{circuit}}),\coqdoceol
\coqdocindent{2.00em}
\coqdocvariable{r} \coqref{computational.::x '+' x}{\coqdocnotation{+}} \coqdocvariable{s} \coqexternalref{:type scope:x '=' x}{http://coq.inria.fr/distrib/8.6/stdlib/Coq.Init.Logic}{\coqdocnotation{=}} \coqref{computational.::x '*' x}{\coqdocnotation{(}}\coqdocvariable{r} \coqref{computational.::x '+' x}{\coqdocnotation{+}} \coqdocvariable{s}\coqref{computational.::x '*' x}{\coqdocnotation{)}} \coqref{computational.::x '*' x}{\coqdocnotation{*}}\coqdoceol
\coqdocindent{6.00em}
\coqref{computational.::x '*' x}{\coqdocnotation{(}} \coqref{computational.::x '+' x}{\coqdocnotation{(}}\coqdocvariable{t} \coqref{computational.::x '+' x}{\coqdocnotation{+}} \coqdocvariable{r}\coqref{computational.::x '+' x}{\coqdocnotation{)}} \coqref{computational.::x '+' x}{\coqdocnotation{+}} \coqref{computational.::x '+' x}{\coqdocnotation{(}}\coqdocvariable{s} \coqref{computational.::x '+' x}{\coqdocnotation{+}} \coqdocvariable{t}\coqref{computational.::x '+' x}{\coqdocnotation{)}} \coqref{computational.::x '*' x}{\coqdocnotation{)}} \coqref{computational.::x '*' x}{\coqdocnotation{*}} \coqdoceol
\coqdocindent{6.00em}
\coqref{computational.::x '*' x}{\coqdocnotation{(}} \coqref{computational.::x '+' x}{\coqdocnotation{(}}\coqdocvariable{r} \coqref{computational.::x '+' x}{\coqdocnotation{+}} \coqdocvariable{u}\coqref{computational.::x '+' x}{\coqdocnotation{)}} \coqref{computational.::x '+' x}{\coqdocnotation{+}} \coqref{computational.::x '+' x}{\coqdocnotation{(}}\coqdocvariable{s} \coqref{computational.::x '+' x}{\coqdocnotation{+}} \coqdocvariable{u}\coqref{computational.::x '+' x}{\coqdocnotation{)}} \coqref{computational.::x '*' x}{\coqdocnotation{)}} \coqref{computational.::x '*' x}{\coqdocnotation{*}} \coqdoceol
\coqdocindent{6.00em}
\coqref{computational.::x '*' x}{\coqdocnotation{(}} \coqref{computational.::x '+' x}{\coqdocnotation{(}}\coqdocvariable{v} \coqref{computational.::x '+' x}{\coqdocnotation{+}} \coqdocvariable{r}\coqref{computational.::x '+' x}{\coqdocnotation{)}} \coqref{computational.::x '+' x}{\coqdocnotation{+}} \coqref{computational.::x '+' x}{\coqdocnotation{(}}\coqdocvariable{v} \coqref{computational.::x '+' x}{\coqdocnotation{+}} \coqdocvariable{s}\coqref{computational.::x '+' x}{\coqdocnotation{)}} \coqref{computational.::x '*' x}{\coqdocnotation{)}} \coqref{computational.::x '*' x}{\coqdocnotation{*}} \coqdoceol
\coqdocindent{6.00em}
\coqref{computational.::x '*' x}{\coqdocnotation{(}} \coqref{computational.::x '+' x}{\coqdocnotation{(}}\coqdocvariable{t} \coqref{computational.::x '+' x}{\coqdocnotation{+}} \coqdocvariable{r}\coqref{computational.::x '+' x}{\coqdocnotation{)}} \coqref{computational.::x '+' x}{\coqdocnotation{+}} \coqref{computational.::x '+' x}{\coqdocnotation{(}}\coqdocvariable{t} \coqref{computational.::x '+' x}{\coqdocnotation{+}} \coqdocvariable{u}\coqref{computational.::x '+' x}{\coqdocnotation{)}} \coqref{computational.::x '+' x}{\coqdocnotation{+}} \coqref{computational.::x '+' x}{\coqdocnotation{(}}\coqdocvariable{s} \coqref{computational.::x '+' x}{\coqdocnotation{+}} \coqdocvariable{u}\coqref{computational.::x '+' x}{\coqdocnotation{)}} \coqref{computational.::x '*' x}{\coqdocnotation{)}} \coqref{computational.::x '*' x}{\coqdocnotation{*}} \coqdoceol
\coqdocindent{6.00em}
\coqref{computational.::x '*' x}{\coqdocnotation{(}} \coqref{computational.::x '+' x}{\coqdocnotation{(}}\coqdocvariable{r} \coqref{computational.::x '+' x}{\coqdocnotation{+}} \coqdocvariable{u}\coqref{computational.::x '+' x}{\coqdocnotation{)}} \coqref{computational.::x '+' x}{\coqdocnotation{+}} \coqref{computational.::x '+' x}{\coqdocnotation{(}}\coqdocvariable{v} \coqref{computational.::x '+' x}{\coqdocnotation{+}} \coqdocvariable{u}\coqref{computational.::x '+' x}{\coqdocnotation{)}} \coqref{computational.::x '+' x}{\coqdocnotation{+}} \coqref{computational.::x '+' x}{\coqdocnotation{(}}\coqdocvariable{v} \coqref{computational.::x '+' x}{\coqdocnotation{+}} \coqdocvariable{s}\coqref{computational.::x '+' x}{\coqdocnotation{)}} \coqref{computational.::x '*' x}{\coqdocnotation{)}} \coqref{computational.::x '*' x}{\coqdocnotation{*}} \coqdoceol
\coqdocindent{6.00em}
\coqref{computational.::x '*' x}{\coqdocnotation{(}} \coqref{computational.::x '+' x}{\coqdocnotation{(}}\coqdocvariable{t} \coqref{computational.::x '+' x}{\coqdocnotation{+}} \coqdocvariable{r}\coqref{computational.::x '+' x}{\coqdocnotation{)}} \coqref{computational.::x '+' x}{\coqdocnotation{+}} \coqref{computational.::x '+' x}{\coqdocnotation{(}}\coqdocvariable{t} \coqref{computational.::x '+' x}{\coqdocnotation{+}} \coqdocvariable{v}\coqref{computational.::x '+' x}{\coqdocnotation{)}} \coqref{computational.::x '+' x}{\coqdocnotation{+}} \coqref{computational.::x '+' x}{\coqdocnotation{(}}\coqdocvariable{v} \coqref{computational.::x '+' x}{\coqdocnotation{+}} \coqdocvariable{s}\coqref{computational.::x '+' x}{\coqdocnotation{)}} \coqref{computational.::x '*' x}{\coqdocnotation{)}} \coqref{computational.::x '*' x}{\coqdocnotation{*}} \coqdoceol
\coqdocindent{6.00em}
\coqref{computational.::x '*' x}{\coqdocnotation{(}} \coqref{computational.::x '+' x}{\coqdocnotation{(}}\coqdocvariable{t} \coqref{computational.::x '+' x}{\coqdocnotation{+}} \coqdocvariable{r}\coqref{computational.::x '+' x}{\coqdocnotation{)}} \coqref{computational.::x '+' x}{\coqdocnotation{+}} \coqref{computational.::x '+' x}{\coqdocnotation{(}}\coqdocvariable{t} \coqref{computational.::x '+' x}{\coqdocnotation{+}} \coqdocvariable{u}\coqref{computational.::x '+' x}{\coqdocnotation{)}} \coqref{computational.::x '+' x}{\coqdocnotation{+}} \coqref{computational.::x '+' x}{\coqdocnotation{(}}\coqdocvariable{v} \coqref{computational.::x '+' x}{\coqdocnotation{+}} \coqdocvariable{u}\coqref{computational.::x '+' x}{\coqdocnotation{)}} \coqref{computational.::x '+' x}{\coqdocnotation{+}}\coqdoceol
\coqdocindent{7.00em}
\coqref{computational.::x '+' x}{\coqdocnotation{(}}\coqdocvariable{v} \coqref{computational.::x '+' x}{\coqdocnotation{+}} \coqdocvariable{s}\coqref{computational.::x '+' x}{\coqdocnotation{)}} \coqref{computational.::x '*' x}{\coqdocnotation{)}}.\coqdoceol
\coqdocnoindent
\coqdockw{Proof}.\coqdoceol
\coqdocindent{1.00em}
\coqdoctac{intros} \coqdocvar{R} \coqdocvar{S} \coqdocvar{T} \coqdocvar{U} \coqdocvar{V}.\coqdoceol
\coqdocindent{1.00em}
\coqdocvar{wham\_bam\_5} \coqdocvar{R} \coqdocvar{S} \coqdocvar{T} \coqdocvar{U} \coqdocvar{V}.\coqdoceol
\coqdocnoindent
\coqdockw{Qed}.\coqdoceol
\coqdocemptyline
\end{coqdoccode}


Figure 11 presents a relatively simple non-series and non-parallel
network.  We first prove that Figure 11 and Figure 12 are equivalent.
The thesis mentions that this can be done by using the star to mesh
transformations, but we do not need such power.  We can use the same
tactics we've used in previous proofs.


For convenience we create definitions of the networks in Figure 11 and
Figure 12.  This can be done by simply creating definitions that cover
every possible path from a to b as in Figure 13.


\begin{coqdoccode}
\coqdocemptyline
\coqdocnoindent
\coqdockw{Definition} \coqdef{computational.figure11}{figure11}{\coqdocdefinition{figure11}} (\coqdocvar{r} \coqdocvar{s} \coqdocvar{t} \coqdocvar{u} \coqdocvar{v} : \coqref{computational.circuit}{\coqdocinductive{circuit}}) :\coqdoceol
\coqdocindent{1.00em}
\coqref{computational.circuit}{\coqdocinductive{circuit}} :=\coqdoceol
\coqdocindent{1.00em}
\coqref{computational.::x '*' x}{\coqdocnotation{(}}\coqdocvariable{r} \coqref{computational.::x '+' x}{\coqdocnotation{+}} \coqdocvariable{s}\coqref{computational.::x '*' x}{\coqdocnotation{)}} \coqref{computational.::x '*' x}{\coqdocnotation{*}}     \coqdoceol
\coqdocindent{1.00em}
\coqref{computational.::x '*' x}{\coqdocnotation{(}}\coqdocvariable{u} \coqref{computational.::x '+' x}{\coqdocnotation{+}} \coqdocvariable{v}\coqref{computational.::x '*' x}{\coqdocnotation{)}} \coqref{computational.::x '*' x}{\coqdocnotation{*}}     \coqdoceol
\coqdocindent{1.00em}
\coqref{computational.::x '*' x}{\coqdocnotation{(}}\coqdocvariable{r} \coqref{computational.::x '+' x}{\coqdocnotation{+}} \coqdocvariable{t} \coqref{computational.::x '+' x}{\coqdocnotation{+}} \coqdocvariable{v}\coqref{computational.::x '*' x}{\coqdocnotation{)}} \coqref{computational.::x '*' x}{\coqdocnotation{*}} \coqdoceol
\coqdocindent{1.00em}
\coqref{computational.::x '*' x}{\coqdocnotation{(}}\coqdocvariable{u} \coqref{computational.::x '+' x}{\coqdocnotation{+}} \coqdocvariable{t} \coqref{computational.::x '+' x}{\coqdocnotation{+}} \coqdocvariable{s}\coqref{computational.::x '*' x}{\coqdocnotation{)}}.\coqdoceol
\coqdocemptyline
\coqdocnoindent
\coqdockw{Definition} \coqdef{computational.figure12}{figure12}{\coqdocdefinition{figure12}} (\coqdocvar{r} \coqdocvar{s} \coqdocvar{t} \coqdocvar{u} \coqdocvar{v} : \coqref{computational.circuit}{\coqdocinductive{circuit}}) :\coqdoceol
\coqdocindent{1.00em}
\coqref{computational.circuit}{\coqdocinductive{circuit}} :=\coqdoceol
\coqdocindent{1.00em}
\coqref{computational.::x '*' x}{\coqdocnotation{(}}\coqdocvariable{r} \coqref{computational.::x '+' x}{\coqdocnotation{+}} \coqdocvariable{s}\coqref{computational.::x '*' x}{\coqdocnotation{)}} \coqref{computational.::x '*' x}{\coqdocnotation{*}} \coqdoceol
\coqdocindent{1.00em}
\coqref{computational.::x '*' x}{\coqdocnotation{(}} \coqref{computational.::x '+' x}{\coqdocnotation{(}}\coqref{computational.::x '*' x}{\coqdocnotation{(}}\coqdocvariable{r} \coqref{computational.::x '+' x}{\coqdocnotation{+}} \coqdocvariable{t}\coqref{computational.::x '*' x}{\coqdocnotation{)}} \coqref{computational.::x '*' x}{\coqdocnotation{*}} \coqdocvariable{u}\coqref{computational.::x '+' x}{\coqdocnotation{)}} \coqref{computational.::x '+' x}{\coqdocnotation{+}} \coqdoceol
\coqdocindent{2.00em}
\coqref{computational.::x '+' x}{\coqdocnotation{(}}\coqref{computational.::x '*' x}{\coqdocnotation{(}}\coqdocvariable{t} \coqref{computational.::x '+' x}{\coqdocnotation{+}} \coqdocvariable{s}\coqref{computational.::x '*' x}{\coqdocnotation{)}} \coqref{computational.::x '*' x}{\coqdocnotation{*}} \coqdocvariable{v}\coqref{computational.::x '+' x}{\coqdocnotation{)}} \coqref{computational.::x '*' x}{\coqdocnotation{)}}. \coqdocemptyline
\end{coqdoccode}


Once we have defined these figures, we can prove their equivalence.


\begin{coqdoccode}
\coqdocemptyline
\coqdocnoindent
\coqdockw{Theorem} \coqdef{computational.figure 11 12 equiv}{figure\_11\_12\_equiv}{\coqdoclemma{figure\_11\_12\_equiv}} :\coqdoceol
\coqdocindent{1.00em}
\coqdockw{\ensuremath{\forall}} (\coqdocvar{r} \coqdocvar{s} \coqdocvar{t} \coqdocvar{u} \coqdocvar{v} : \coqref{computational.circuit}{\coqdocinductive{circuit}}),\coqdoceol
\coqdocindent{2.00em}
\coqref{computational.figure11}{\coqdocdefinition{figure11}} \coqdocvariable{r} \coqdocvariable{s} \coqdocvariable{t} \coqdocvariable{u} \coqdocvariable{v} \coqexternalref{:type scope:x '=' x}{http://coq.inria.fr/distrib/8.6/stdlib/Coq.Init.Logic}{\coqdocnotation{=}}\coqdoceol
\coqdocindent{2.00em}
\coqref{computational.figure12}{\coqdocdefinition{figure12}} \coqdocvariable{r} \coqdocvariable{s} \coqdocvariable{t} \coqdocvariable{u} \coqdocvariable{v}.\coqdoceol
\coqdocnoindent
\coqdockw{Proof}.\coqdoceol
\coqdocindent{1.00em}
\coqdoctac{intros} \coqdocvar{R} \coqdocvar{S} \coqdocvar{T} \coqdocvar{U} \coqdocvar{V}.\coqdoceol
\coqdocindent{1.00em}
\coqdocvar{wham\_bam\_5} \coqdocvar{R} \coqdocvar{S} \coqdocvar{T} \coqdocvar{U} \coqdocvar{V}.\coqdoceol
\coqdocnoindent
\coqdockw{Qed}.\coqdoceol
\coqdocemptyline
\end{coqdoccode}


The thesis also mentions in this section that Figure 11 can be
simplified.  We prove this assertion here.


\begin{coqdoccode}
\coqdocemptyline
\coqdocnoindent
\coqdockw{Theorem} \coqdef{computational.figure 11 simpler}{figure\_11\_simpler}{\coqdoclemma{figure\_11\_simpler}} :\coqdoceol
\coqdocindent{1.00em}
\coqdockw{\ensuremath{\forall}} (\coqdocvar{r} \coqdocvar{s} \coqdocvar{t} \coqdocvar{u} \coqdocvar{v} : \coqref{computational.circuit}{\coqdocinductive{circuit}}),\coqdoceol
\coqdocindent{2.00em}
\coqref{computational.figure11}{\coqdocdefinition{figure11}} \coqdocvariable{r} \coqdocvariable{s} \coqdocvariable{t} \coqdocvariable{u} \coqdocvariable{v} \coqexternalref{:type scope:x '=' x}{http://coq.inria.fr/distrib/8.6/stdlib/Coq.Init.Logic}{\coqdocnotation{=}}\coqdoceol
\coqdocindent{2.00em}
\coqref{computational.::x '+' x}{\coqdocnotation{(}}\coqdocvariable{r} \coqref{computational.::x '*' x}{\coqdocnotation{*}} \coqdocvariable{u}\coqref{computational.::x '+' x}{\coqdocnotation{)}} \coqref{computational.::x '+' x}{\coqdocnotation{+}} \coqref{computational.::x '+' x}{\coqdocnotation{(}}\coqdocvariable{s} \coqref{computational.::x '*' x}{\coqdocnotation{*}} \coqdocvariable{v}\coqref{computational.::x '+' x}{\coqdocnotation{)}} \coqref{computational.::x '+' x}{\coqdocnotation{+}}\coqdoceol
\coqdocindent{2.00em}
\coqref{computational.::x '+' x}{\coqdocnotation{(}}\coqdocvariable{r} \coqref{computational.::x '*' x}{\coqdocnotation{*}} \coqdocvariable{t} \coqref{computational.::x '*' x}{\coqdocnotation{*}} \coqdocvariable{v}\coqref{computational.::x '+' x}{\coqdocnotation{)}} \coqref{computational.::x '+' x}{\coqdocnotation{+}} \coqref{computational.::x '+' x}{\coqdocnotation{(}}\coqdocvariable{s} \coqref{computational.::x '*' x}{\coqdocnotation{*}} \coqdocvariable{t} \coqref{computational.::x '*' x}{\coqdocnotation{*}} \coqdocvariable{u}\coqref{computational.::x '+' x}{\coqdocnotation{)}}.\coqdoceol
\coqdocnoindent
\coqdockw{Proof}.\coqdoceol
\coqdocindent{1.00em}
\coqdoctac{intros} \coqdocvar{R} \coqdocvar{S} \coqdocvar{T} \coqdocvar{U} \coqdocvar{V}.\coqdoceol
\coqdocindent{1.00em}
\coqdocvar{wham\_bam\_5} \coqdocvar{R} \coqdocvar{S} \coqdocvar{T} \coqdocvar{U} \coqdocvar{V}.\coqdoceol
\coqdocnoindent
\coqdockw{Qed}.\coqdoceol
\coqdocemptyline
\end{coqdoccode}
\section{Simultaneous Equations}




We leave most of the formalization of simultaneous equations to future
work, but prove the implication on page 25.  To do this we create
several new Ltac tactics.


The following Ltac tactic uses the demorgan\_9a\_2 theorem on products
of negations.


\begin{coqdoccode}
\coqdocemptyline
\coqdocnoindent
\coqdockw{Ltac} \coqdocvar{demorgan\_2} :=\coqdoceol
\coqdocindent{1.00em}
\coqdockw{match} \coqdockw{goal} \coqdockw{with}\coqdoceol
\coqdocindent{1.00em}
\ensuremath{|} [ \ensuremath{\vdash} ( \coqref{computational.::x '*' x}{\coqdocnotation{(}}\coqref{computational.negation}{\coqdocdefinition{negation}} \coqdocvar{\_}\coqref{computational.::x '*' x}{\coqdocnotation{)}} \coqref{computational.::x '*' x}{\coqdocnotation{*}}\coqdoceol
\coqdocindent{5.50em}
\coqref{computational.::x '*' x}{\coqdocnotation{(}}\coqref{computational.negation}{\coqdocdefinition{negation}} \coqdocvar{\_}\coqref{computational.::x '*' x}{\coqdocnotation{)}} \coqexternalref{:type scope:x '=' x}{http://coq.inria.fr/distrib/8.6/stdlib/Coq.Init.Logic}{\coqdocnotation{=}} \coqdocvar{\_} ) ] \ensuremath{\Rightarrow}\coqdoceol
\coqdocindent{2.00em}
\coqdoctac{rewrite} \ensuremath{\leftarrow} \coqref{computational.demorgan 9a 2}{\coqdoclemma{demorgan\_9a\_2}}\coqdoceol
\coqdocindent{1.00em}
\coqdockw{end}.\coqdoceol
\coqdocemptyline
\end{coqdoccode}


The following tactic applies reflexivity to trivial goals.


\begin{coqdoccode}
\coqdocemptyline
\coqdocnoindent
\coqdockw{Ltac} \coqdocvar{explicit\_reflexive} :=\coqdoceol
\coqdocindent{1.00em}
\coqdoctac{try} \coqdockw{match} \coqdockw{goal} \coqdockw{with}\coqdoceol
\coqdocindent{3.00em}
\ensuremath{|} [ \ensuremath{\vdash} (\coqref{computational.open}{\coqdocconstructor{open}} \coqexternalref{:type scope:x '=' x}{http://coq.inria.fr/distrib/8.6/stdlib/Coq.Init.Logic}{\coqdocnotation{=}} \coqref{computational.open}{\coqdocconstructor{open}}) ] \ensuremath{\Rightarrow}\coqdoceol
\coqdocindent{4.00em}
\coqdoctac{reflexivity}\coqdoceol
\coqdocindent{3.00em}
\ensuremath{|} [ \ensuremath{\vdash} (\coqref{computational.closed}{\coqdocconstructor{closed}} \coqexternalref{:type scope:x '=' x}{http://coq.inria.fr/distrib/8.6/stdlib/Coq.Init.Logic}{\coqdocnotation{=}} \coqref{computational.closed}{\coqdocconstructor{closed}}) ] \ensuremath{\Rightarrow}\coqdoceol
\coqdocindent{4.00em}
\coqdoctac{reflexivity}\coqdoceol
\coqdocindent{3.00em}
\coqdockw{end}.\coqdoceol
\coqdocemptyline
\end{coqdoccode}


The following tactic leverages potential contradictions in the
hypotheses in the context.


 \begin{coqdoccode}
\coqdocemptyline
\coqdocnoindent
\coqdockw{Ltac} \coqdocvar{contra} :=\coqdoceol
\coqdocindent{1.00em}
\coqdockw{match} \coqdockw{goal} \coqdockw{with}\coqdoceol
\coqdocindent{1.00em}
\ensuremath{|} [ \coqdocvar{Hx} : (\coqref{computational.open}{\coqdocconstructor{open}} \coqref{computational.::x '*' x}{\coqdocnotation{*}} \coqref{computational.::x '*' x}{\coqdocnotation{(}}\coqref{computational.negation}{\coqdocdefinition{negation}} \coqref{computational.closed}{\coqdocconstructor{closed}}\coqref{computational.::x '*' x}{\coqdocnotation{)}} \coqexternalref{:type scope:x '=' x}{http://coq.inria.fr/distrib/8.6/stdlib/Coq.Init.Logic}{\coqdocnotation{=}} \coqref{computational.closed}{\coqdocconstructor{closed}}) \ensuremath{\vdash}\coqdoceol
\coqdocindent{3.00em}
(\coqref{computational.closed}{\coqdocconstructor{closed}} \coqexternalref{:type scope:x '=' x}{http://coq.inria.fr/distrib/8.6/stdlib/Coq.Init.Logic}{\coqdocnotation{=}} \coqref{computational.open}{\coqdocconstructor{open}}) ] \ensuremath{\Rightarrow}\coqdoceol
\coqdocindent{2.00em}
( \coqdoctac{rewrite} \ensuremath{\leftarrow} \coqref{computational.closed neg}{\coqdoclemma{closed\_neg}} \coqdoctac{in} \coqdocvar{Hx};\coqdoceol
\coqdocindent{3.00em}
\coqdoctac{simpl} \coqdoctac{in} \coqdocvar{Hx};\coqdoceol
\coqdocindent{3.00em}
\coqdoctac{rewrite} \ensuremath{\rightarrow} \coqdocvar{Hx};\coqdoceol
\coqdocindent{3.00em}
\coqdoctac{reflexivity} )\coqdoceol
\coqdocindent{1.00em}
\ensuremath{|} [ \coqdocvar{Hx} : (\coqref{computational.open}{\coqdocconstructor{open}} \coqref{computational.::x '*' x}{\coqdocnotation{*}} \coqref{computational.::x '*' x}{\coqdocnotation{(}}\coqref{computational.negation}{\coqdocdefinition{negation}} \coqref{computational.closed}{\coqdocconstructor{closed}}\coqref{computational.::x '*' x}{\coqdocnotation{)}} \coqexternalref{:type scope:x '=' x}{http://coq.inria.fr/distrib/8.6/stdlib/Coq.Init.Logic}{\coqdocnotation{=}} \coqref{computational.closed}{\coqdocconstructor{closed}}) \ensuremath{\vdash}\coqdoceol
\coqdocindent{3.00em}
(\coqref{computational.open}{\coqdocconstructor{open}} \coqexternalref{:type scope:x '=' x}{http://coq.inria.fr/distrib/8.6/stdlib/Coq.Init.Logic}{\coqdocnotation{=}} \coqref{computational.closed}{\coqdocconstructor{closed}}) ] \ensuremath{\Rightarrow}\coqdoceol
\coqdocindent{2.00em}
( \coqdoctac{rewrite} \ensuremath{\leftarrow} \coqref{computational.closed neg}{\coqdoclemma{closed\_neg}} \coqdoctac{in} \coqdocvar{Hx};\coqdoceol
\coqdocindent{3.00em}
\coqdoctac{simpl} \coqdoctac{in} \coqdocvar{Hx};\coqdoceol
\coqdocindent{3.00em}
\coqdoctac{rewrite} \ensuremath{\rightarrow} \coqdocvar{Hx};\coqdoceol
\coqdocindent{3.00em}
\coqdoctac{reflexivity} )\coqdoceol
\coqdocindent{1.00em}
\ensuremath{|} [ \coqdocvar{Hx} : (\coqref{computational.open}{\coqdocconstructor{open}} \coqexternalref{:type scope:x '=' x}{http://coq.inria.fr/distrib/8.6/stdlib/Coq.Init.Logic}{\coqdocnotation{=}} \coqref{computational.closed}{\coqdocconstructor{closed}}) \ensuremath{\vdash}\coqdoceol
\coqdocindent{3.00em}
\coqdocvar{\_} ] \ensuremath{\Rightarrow}\coqdoceol
\coqdocindent{2.00em}
( \coqdoctac{simpl};\coqdoceol
\coqdocindent{3.00em}
\coqdoctac{rewrite} \ensuremath{\rightarrow} \coqdocvar{Hx};\coqdoceol
\coqdocindent{3.00em}
\coqdoctac{reflexivity} )\coqdoceol
\coqdocindent{1.00em}
\ensuremath{|} [ \coqdocvar{Hx} : (\coqref{computational.closed}{\coqdocconstructor{closed}} \coqexternalref{:type scope:x '=' x}{http://coq.inria.fr/distrib/8.6/stdlib/Coq.Init.Logic}{\coqdocnotation{=}} \coqref{computational.open}{\coqdocconstructor{open}}) \ensuremath{\vdash}\coqdoceol
\coqdocindent{3.00em}
\coqdocvar{\_} ] \ensuremath{\Rightarrow}\coqdoceol
\coqdocindent{2.00em}
( \coqdoctac{simpl};\coqdoceol
\coqdocindent{3.00em}
\coqdoctac{rewrite} \ensuremath{\rightarrow} \coqdocvar{Hx};\coqdoceol
\coqdocindent{3.00em}
\coqdoctac{reflexivity} )\coqdoceol
\coqdocindent{1.00em}
\coqdockw{end}.\coqdoceol
\coqdocemptyline
\end{coqdoccode}


The following nearly trivial tactic just simplifies a hypothesis in
the context.


\begin{coqdoccode}
\coqdocemptyline
\coqdocnoindent
\coqdockw{Ltac} \coqdocvar{simpl\_h} :=\coqdoceol
\coqdocindent{1.00em}
\coqdockw{match} \coqdockw{goal} \coqdockw{with}\coqdoceol
\coqdocindent{1.00em}
\ensuremath{|} [ \coqdocvar{Hx} : \coqdocvar{\_} \ensuremath{\vdash} \coqdocvar{\_} ] \ensuremath{\Rightarrow} \coqdoctac{simpl} \coqdoctac{in} \coqdocvar{Hx}\coqdoceol
\coqdocindent{1.00em}
\coqdockw{end}.\coqdoceol
\coqdocemptyline
\end{coqdoccode}


Now we assemble the above Ltac tactics into more powerful tools.


\begin{coqdoccode}
\coqdocnoindent
\coqdockw{Ltac} \coqdocvar{pow} :=\coqdoceol
\coqdocindent{1.00em}
\coqdoctac{try} \coqdoctac{repeat} ( \coqdocvar{demorgan\_2} ||\coqdoceol
\coqdocindent{7.50em}
\coqdocvar{explicit\_reflexive} ||\coqdoceol
\coqdocindent{7.50em}
\coqdocvar{contra} ||\coqdoceol
\coqdocindent{7.50em}
\coqdocvar{simpl\_h} ).\coqdoceol
\coqdocemptyline
\coqdocnoindent
\coqdockw{Ltac} \coqdocvar{blammo\_2} \coqdocvar{X} \coqdocvar{Y} :=\coqdoceol
\coqdocindent{1.00em}
\coqdoctac{try} \coqdoctac{repeat} (\coqdocvar{pow};\coqdoceol
\coqdocindent{7.00em}
\coqdoctac{destruct} \coqdocvar{X}, \coqdocvar{Y};\coqdoceol
\coqdocindent{7.00em}
\coqdoctac{repeat} (\coqdocvar{wham};\coqdoceol
\coqdocindent{11.00em}
\coqdocvar{bam});\coqdoceol
\coqdocindent{7.00em}
\coqdoctac{repeat} (\coqdoctac{simpl};\coqdoceol
\coqdocindent{11.00em}
\coqdoctac{reflexivity})).\coqdoceol
\coqdocemptyline
\coqdocnoindent
\coqdockw{Ltac} \coqdocvar{blammo\_3} \coqdocvar{X} \coqdocvar{Y} \coqdocvar{Z} :=\coqdoceol
\coqdocindent{1.00em}
\coqdoctac{try} \coqdoctac{repeat} (\coqdocvar{pow};\coqdoceol
\coqdocindent{7.00em}
\coqdoctac{destruct} \coqdocvar{X}, \coqdocvar{Y}, \coqdocvar{Z};\coqdoceol
\coqdocindent{7.00em}
\coqdoctac{repeat} (\coqdocvar{wham};\coqdoceol
\coqdocindent{11.00em}
\coqdocvar{bam});\coqdoceol
\coqdocindent{7.00em}
\coqdoctac{repeat} (\coqdoctac{simpl};\coqdoceol
\coqdocindent{11.00em}
\coqdoctac{reflexivity})).\coqdoceol
\coqdocemptyline
\end{coqdoccode}


Now that we have our new Ltac machinery, we can tackle the implication
on page 25.


\begin{coqdoccode}
\coqdocnoindent
\coqdockw{Theorem} \coqdef{computational.page 25 implication}{page\_25\_implication}{\coqdoclemma{page\_25\_implication}} :\coqdoceol
\coqdocindent{1.00em}
\coqdockw{\ensuremath{\forall}} (\coqdocvar{a} \coqdocvar{b} : \coqref{computational.circuit}{\coqdocinductive{circuit}}),\coqdoceol
\coqdocindent{2.00em}
\coqdocvariable{a} \coqref{computational.::x '*' x}{\coqdocnotation{*}} \coqref{computational.::x '*' x}{\coqdocnotation{(}}\coqref{computational.negation}{\coqdocdefinition{negation}} \coqdocvariable{b}\coqref{computational.::x '*' x}{\coqdocnotation{)}} \coqexternalref{:type scope:x '=' x}{http://coq.inria.fr/distrib/8.6/stdlib/Coq.Init.Logic}{\coqdocnotation{=}} \coqref{computational.closed}{\coqdocconstructor{closed}} \coqexternalref{:type scope:x '->' x}{http://coq.inria.fr/distrib/8.6/stdlib/Coq.Init.Logic}{\coqdocnotation{\ensuremath{\rightarrow}}}\coqdoceol
\coqdocindent{2.00em}
\coqref{computational.::x '*' x}{\coqdocnotation{(}}\coqref{computational.negation}{\coqdocdefinition{negation}} \coqdocvariable{a}\coqref{computational.::x '*' x}{\coqdocnotation{)}} \coqref{computational.::x '*' x}{\coqdocnotation{*}} \coqref{computational.::x '*' x}{\coqdocnotation{(}}\coqref{computational.negation}{\coqdocdefinition{negation}} \coqdocvariable{b}\coqref{computational.::x '*' x}{\coqdocnotation{)}} \coqexternalref{:type scope:x '=' x}{http://coq.inria.fr/distrib/8.6/stdlib/Coq.Init.Logic}{\coqdocnotation{=}} \coqexternalref{:type scope:x '=' x}{http://coq.inria.fr/distrib/8.6/stdlib/Coq.Init.Logic}{\coqdocnotation{(}}\coqref{computational.negation}{\coqdocdefinition{negation}} \coqdocvariable{b}\coqexternalref{:type scope:x '=' x}{http://coq.inria.fr/distrib/8.6/stdlib/Coq.Init.Logic}{\coqdocnotation{)}}.\coqdoceol
\coqdocnoindent
\coqdockw{Proof}.\coqdoceol
\coqdocindent{1.00em}
\coqdoctac{intros} \coqdocvar{A} \coqdocvar{B} \coqdocvar{H}.\coqdoceol
\coqdocindent{1.00em}
\coqdocvar{blammo\_2} \coqdocvar{A} \coqdocvar{B}.\coqdoceol
\coqdocnoindent
\coqdockw{Qed}.\coqdoceol
\coqdocemptyline
\coqdocnoindent
\coqdockw{Theorem} \coqdef{computational.page 25 implication 2}{page\_25\_implication\_2}{\coqdoclemma{page\_25\_implication\_2}} :\coqdoceol
\coqdocindent{1.00em}
\coqdockw{\ensuremath{\forall}} (\coqdocvar{a} \coqdocvar{b} : \coqref{computational.circuit}{\coqdocinductive{circuit}}),\coqdoceol
\coqdocindent{2.00em}
\coqdocvariable{a} \coqref{computational.::x '*' x}{\coqdocnotation{*}} \coqref{computational.::x '*' x}{\coqdocnotation{(}}\coqref{computational.negation}{\coqdocdefinition{negation}} \coqdocvariable{b}\coqref{computational.::x '*' x}{\coqdocnotation{)}} \coqexternalref{:type scope:x '=' x}{http://coq.inria.fr/distrib/8.6/stdlib/Coq.Init.Logic}{\coqdocnotation{=}} \coqref{computational.closed}{\coqdocconstructor{closed}} \coqexternalref{:type scope:x '->' x}{http://coq.inria.fr/distrib/8.6/stdlib/Coq.Init.Logic}{\coqdocnotation{\ensuremath{\rightarrow}}}\coqdoceol
\coqdocindent{2.00em}
\coqexternalref{:type scope:x '=' x}{http://coq.inria.fr/distrib/8.6/stdlib/Coq.Init.Logic}{\coqdocnotation{(}}\coqdocvariable{a} \coqref{computational.::x '*' x}{\coqdocnotation{*}} \coqdocvariable{b}\coqexternalref{:type scope:x '=' x}{http://coq.inria.fr/distrib/8.6/stdlib/Coq.Init.Logic}{\coqdocnotation{)}} \coqexternalref{:type scope:x '=' x}{http://coq.inria.fr/distrib/8.6/stdlib/Coq.Init.Logic}{\coqdocnotation{=}} \coqdocvariable{a}.\coqdoceol
\coqdocnoindent
\coqdockw{Proof}.\coqdoceol
\coqdocindent{1.00em}
\coqdoctac{intros} \coqdocvar{A} \coqdocvar{B} \coqdocvar{H}.\coqdoceol
\coqdocindent{1.00em}
\coqdocvar{blammo\_2} \coqdocvar{A} \coqdocvar{B}.\coqdoceol
\coqdocnoindent
\coqdockw{Qed}.\coqdoceol
\coqdocemptyline
\coqdocnoindent
\coqdockw{Theorem} \coqdef{computational.page 25 implication 3}{page\_25\_implication\_3}{\coqdoclemma{page\_25\_implication\_3}} :\coqdoceol
\coqdocindent{1.00em}
\coqdockw{\ensuremath{\forall}} (\coqdocvar{a} \coqdocvar{b} : \coqref{computational.circuit}{\coqdocinductive{circuit}}),\coqdoceol
\coqdocindent{2.00em}
\coqdocvariable{a} \coqref{computational.::x '*' x}{\coqdocnotation{*}} \coqref{computational.::x '*' x}{\coqdocnotation{(}}\coqref{computational.negation}{\coqdocdefinition{negation}} \coqdocvariable{b}\coqref{computational.::x '*' x}{\coqdocnotation{)}} \coqexternalref{:type scope:x '=' x}{http://coq.inria.fr/distrib/8.6/stdlib/Coq.Init.Logic}{\coqdocnotation{=}} \coqref{computational.closed}{\coqdocconstructor{closed}} \coqexternalref{:type scope:x '->' x}{http://coq.inria.fr/distrib/8.6/stdlib/Coq.Init.Logic}{\coqdocnotation{\ensuremath{\rightarrow}}}\coqdoceol
\coqdocindent{2.00em}
\coqref{computational.::x '+' x}{\coqdocnotation{(}}\coqref{computational.negation}{\coqdocdefinition{negation}} \coqdocvariable{a}\coqref{computational.::x '+' x}{\coqdocnotation{)}} \coqref{computational.::x '+' x}{\coqdocnotation{+}} \coqdocvariable{b} \coqexternalref{:type scope:x '=' x}{http://coq.inria.fr/distrib/8.6/stdlib/Coq.Init.Logic}{\coqdocnotation{=}} \coqref{computational.open}{\coqdocconstructor{open}}.\coqdoceol
\coqdocnoindent
\coqdockw{Proof}.\coqdoceol
\coqdocindent{1.00em}
\coqdoctac{intros} \coqdocvar{A} \coqdocvar{B} \coqdocvar{H}.\coqdoceol
\coqdocindent{1.00em}
\coqdocvar{blammo\_2} \coqdocvar{A} \coqdocvar{B}.\coqdoceol
\coqdocnoindent
\coqdockw{Qed}.\coqdoceol
\coqdocemptyline
\coqdocnoindent
\coqdockw{Theorem} \coqdef{computational.page 25 implication 4}{page\_25\_implication\_4}{\coqdoclemma{page\_25\_implication\_4}} :\coqdoceol
\coqdocindent{1.00em}
\coqdockw{\ensuremath{\forall}} (\coqdocvar{a} \coqdocvar{b} : \coqref{computational.circuit}{\coqdocinductive{circuit}}),\coqdoceol
\coqdocindent{2.00em}
\coqdocvariable{a} \coqref{computational.::x '*' x}{\coqdocnotation{*}} \coqref{computational.::x '*' x}{\coqdocnotation{(}}\coqref{computational.negation}{\coqdocdefinition{negation}} \coqdocvariable{b}\coqref{computational.::x '*' x}{\coqdocnotation{)}} \coqexternalref{:type scope:x '=' x}{http://coq.inria.fr/distrib/8.6/stdlib/Coq.Init.Logic}{\coqdocnotation{=}} \coqref{computational.closed}{\coqdocconstructor{closed}} \coqexternalref{:type scope:x '->' x}{http://coq.inria.fr/distrib/8.6/stdlib/Coq.Init.Logic}{\coqdocnotation{\ensuremath{\rightarrow}}}\coqdoceol
\coqdocindent{2.00em}
\coqref{computational.::x '+' x}{\coqdocnotation{(}}\coqref{computational.negation}{\coqdocdefinition{negation}} \coqdocvariable{a}\coqref{computational.::x '+' x}{\coqdocnotation{)}} \coqref{computational.::x '+' x}{\coqdocnotation{+}} \coqref{computational.::x '+' x}{\coqdocnotation{(}}\coqref{computational.negation}{\coqdocdefinition{negation}} \coqdocvariable{b}\coqref{computational.::x '+' x}{\coqdocnotation{)}} \coqexternalref{:type scope:x '=' x}{http://coq.inria.fr/distrib/8.6/stdlib/Coq.Init.Logic}{\coqdocnotation{=}} \coqexternalref{:type scope:x '=' x}{http://coq.inria.fr/distrib/8.6/stdlib/Coq.Init.Logic}{\coqdocnotation{(}}\coqref{computational.negation}{\coqdocdefinition{negation}} \coqdocvariable{a}\coqexternalref{:type scope:x '=' x}{http://coq.inria.fr/distrib/8.6/stdlib/Coq.Init.Logic}{\coqdocnotation{)}}.\coqdoceol
\coqdocnoindent
\coqdockw{Proof}.\coqdoceol
\coqdocindent{1.00em}
\coqdoctac{intros} \coqdocvar{A} \coqdocvar{B} \coqdocvar{H}.\coqdoceol
\coqdocindent{1.00em}
\coqdocvar{blammo\_2} \coqdocvar{A} \coqdocvar{B}.\coqdoceol
\coqdocnoindent
\coqdockw{Qed}.\coqdoceol
\coqdocemptyline
\coqdocnoindent
\coqdockw{Theorem} \coqdef{computational.page 25 implication 5}{page\_25\_implication\_5}{\coqdoclemma{page\_25\_implication\_5}} :\coqdoceol
\coqdocindent{1.00em}
\coqdockw{\ensuremath{\forall}} (\coqdocvar{a} \coqdocvar{b} : \coqref{computational.circuit}{\coqdocinductive{circuit}}),\coqdoceol
\coqdocindent{2.00em}
\coqdocvariable{a} \coqref{computational.::x '*' x}{\coqdocnotation{*}} \coqref{computational.::x '*' x}{\coqdocnotation{(}}\coqref{computational.negation}{\coqdocdefinition{negation}} \coqdocvariable{b}\coqref{computational.::x '*' x}{\coqdocnotation{)}} \coqexternalref{:type scope:x '=' x}{http://coq.inria.fr/distrib/8.6/stdlib/Coq.Init.Logic}{\coqdocnotation{=}} \coqref{computational.closed}{\coqdocconstructor{closed}} \coqexternalref{:type scope:x '->' x}{http://coq.inria.fr/distrib/8.6/stdlib/Coq.Init.Logic}{\coqdocnotation{\ensuremath{\rightarrow}}}\coqdoceol
\coqdocindent{2.00em}
\coqexternalref{:type scope:x '=' x}{http://coq.inria.fr/distrib/8.6/stdlib/Coq.Init.Logic}{\coqdocnotation{(}}\coqdocvariable{a} \coqref{computational.::x '+' x}{\coqdocnotation{+}} \coqdocvariable{b}\coqexternalref{:type scope:x '=' x}{http://coq.inria.fr/distrib/8.6/stdlib/Coq.Init.Logic}{\coqdocnotation{)}} \coqexternalref{:type scope:x '=' x}{http://coq.inria.fr/distrib/8.6/stdlib/Coq.Init.Logic}{\coqdocnotation{=}} \coqdocvariable{b}.\coqdoceol
\coqdocnoindent
\coqdockw{Proof}.\coqdoceol
\coqdocindent{1.00em}
\coqdoctac{intros} \coqdocvar{A} \coqdocvar{B} \coqdocvar{H}.\coqdoceol
\coqdocindent{1.00em}
\coqdocvar{blammo\_2} \coqdocvar{A} \coqdocvar{B}.\coqdoceol
\coqdocnoindent
\coqdockw{Qed}.\coqdoceol
\coqdocemptyline
\end{coqdoccode}
\section{Matrix Methods and Special Methods}




We leave the matrix methods formalization to future work, but prove
the implication on page 30.


\begin{coqdoccode}
\coqdocemptyline
\coqdocnoindent
\coqdockw{Theorem} \coqdef{computational.page 30 implication}{page\_30\_implication}{\coqdoclemma{page\_30\_implication}} : \coqdockw{\ensuremath{\forall}} (\coqdocvar{r} \coqdocvar{s} \coqdocvar{x} : \coqref{computational.circuit}{\coqdocinductive{circuit}}),\coqdoceol
\coqdocindent{2.00em}
\coqexternalref{:type scope:x '=' x}{http://coq.inria.fr/distrib/8.6/stdlib/Coq.Init.Logic}{\coqdocnotation{(}}\coqref{computational.negation}{\coqdocdefinition{negation}} \coqdocvariable{x}\coqexternalref{:type scope:x '=' x}{http://coq.inria.fr/distrib/8.6/stdlib/Coq.Init.Logic}{\coqdocnotation{)}} \coqexternalref{:type scope:x '=' x}{http://coq.inria.fr/distrib/8.6/stdlib/Coq.Init.Logic}{\coqdocnotation{=}} \coqref{computational.::x '+' x}{\coqdocnotation{(}}\coqdocvariable{r} \coqref{computational.::x '*' x}{\coqdocnotation{*}} \coqref{computational.::x '*' x}{\coqdocnotation{(}}\coqref{computational.negation}{\coqdocdefinition{negation}} \coqdocvariable{x}\coqref{computational.::x '*' x}{\coqdocnotation{)}}\coqref{computational.::x '+' x}{\coqdocnotation{)}} \coqref{computational.::x '+' x}{\coqdocnotation{+}} \coqdocvariable{s} \coqexternalref{:type scope:x '->' x}{http://coq.inria.fr/distrib/8.6/stdlib/Coq.Init.Logic}{\coqdocnotation{\ensuremath{\rightarrow}}}\coqdoceol
\coqdocindent{2.00em}
\coqdocvariable{x} \coqexternalref{:type scope:x '=' x}{http://coq.inria.fr/distrib/8.6/stdlib/Coq.Init.Logic}{\coqdocnotation{=}} \coqref{computational.::x '*' x}{\coqdocnotation{(}}\coqref{computational.::x '+' x}{\coqdocnotation{(}}\coqref{computational.negation}{\coqdocdefinition{negation}} \coqdocvariable{r}\coqref{computational.::x '+' x}{\coqdocnotation{)}} \coqref{computational.::x '+' x}{\coqdocnotation{+}} \coqdocvariable{x}\coqref{computational.::x '*' x}{\coqdocnotation{)}} \coqref{computational.::x '*' x}{\coqdocnotation{*}} \coqref{computational.::x '*' x}{\coqdocnotation{(}}\coqref{computational.negation}{\coqdocdefinition{negation}} \coqdocvariable{s}\coqref{computational.::x '*' x}{\coqdocnotation{)}}.\coqdoceol
\coqdocnoindent
\coqdockw{Proof}.\coqdoceol
\coqdocindent{1.00em}
\coqdoctac{intros} \coqdocvar{R} \coqdocvar{S} \coqdocvar{X}.\coqdoceol
\coqdocindent{1.00em}
\coqdoctac{intros} \coqdocvar{H1}.\coqdoceol
\coqdocindent{1.00em}
\coqdocvar{blammo\_3} \coqdocvar{R} \coqdocvar{S} \coqdocvar{X}.\coqdoceol
\coqdocnoindent
\coqdockw{Qed}.\coqdoceol
\coqdocemptyline
\end{coqdoccode}
\section{Synthesis of Networks}




We now move to formalization of synthesis techniques.  We first define
the disjunct operator on page 32.


\begin{coqdoccode}
\coqdocemptyline
\coqdocnoindent
\coqdockw{Definition} \coqdef{computational.disjunct}{disjunct}{\coqdocdefinition{disjunct}} (\coqdocvar{v1} \coqdocvar{v2} : \coqref{computational.circuit}{\coqdocinductive{circuit}}) :\coqdoceol
\coqdocindent{1.00em}
\coqref{computational.circuit}{\coqdocinductive{circuit}} :=\coqdoceol
\coqdocindent{1.00em}
\coqref{computational.::x '+' x}{\coqdocnotation{(}}\coqdocvariable{v1} \coqref{computational.::x '*' x}{\coqdocnotation{*}} \coqref{computational.::x '*' x}{\coqdocnotation{(}}\coqref{computational.negation}{\coqdocdefinition{negation}} \coqdocvariable{v2}\coqref{computational.::x '*' x}{\coqdocnotation{)}}\coqref{computational.::x '+' x}{\coqdocnotation{)}} \coqref{computational.::x '+' x}{\coqdocnotation{+}}\coqdoceol
\coqdocindent{1.00em}
\coqref{computational.::x '+' x}{\coqdocnotation{(}}\coqref{computational.::x '*' x}{\coqdocnotation{(}}\coqref{computational.negation}{\coqdocdefinition{negation}} \coqdocvariable{v1}\coqref{computational.::x '*' x}{\coqdocnotation{)}} \coqref{computational.::x '*' x}{\coqdocnotation{*}} \coqdocvariable{v2}\coqref{computational.::x '+' x}{\coqdocnotation{)}}.\coqdoceol
\coqdocemptyline
\end{coqdoccode}


We provide a bit of notation that aids our development.


\begin{coqdoccode}
\coqdocemptyline
\coqdocnoindent
\coqdockw{Notation} \coqdef{computational.::x '@' x}{"}{"}x @ y" :=\coqdoceol
\coqdocindent{1.00em}
(\coqref{computational.disjunct}{\coqdocdefinition{disjunct}} \coqdocvar{x} \coqdocvar{y})\coqdoceol
\coqdocindent{2.00em}
(\coqdoctac{at} \coqdockw{level} 50,\coqdoceol
\coqdocindent{2.50em}
\coqdoctac{left} \coqdockw{associativity}).\coqdoceol
\coqdocemptyline
\end{coqdoccode}


We create some new tactics that allow us to automate the use of the
disjunct definition.


\begin{coqdoccode}
\coqdocemptyline
\coqdocnoindent
\coqdockw{Ltac} \coqdocvar{disjunctor} :=\coqdoceol
\coqdocindent{1.00em}
\coqdockw{match} \coqdockw{goal} \coqdockw{with}\coqdoceol
\coqdocindent{1.00em}
\ensuremath{|} [ \ensuremath{\vdash} \coqdocvar{\_} \coqref{computational.::x '@' x}{\coqdocnotation{@}} \coqdocvar{\_} \coqexternalref{:type scope:x '=' x}{http://coq.inria.fr/distrib/8.6/stdlib/Coq.Init.Logic}{\coqdocnotation{=}} \coqdocvar{\_} ] \ensuremath{\Rightarrow} \coqdoctac{unfold} \coqref{computational.disjunct}{\coqdocdefinition{disjunct}}\coqdoceol
\coqdocindent{1.00em}
\ensuremath{|} [ \ensuremath{\vdash} \coqdocvar{\_} \coqref{computational.::x '*' x}{\coqdocnotation{*}} \coqref{computational.::x '*' x}{\coqdocnotation{(}}\coqdocvar{\_} \coqref{computational.::x '@' x}{\coqdocnotation{@}} \coqdocvar{\_}\coqref{computational.::x '*' x}{\coqdocnotation{)}} \coqexternalref{:type scope:x '=' x}{http://coq.inria.fr/distrib/8.6/stdlib/Coq.Init.Logic}{\coqdocnotation{=}} \coqdocvar{\_} ] \ensuremath{\Rightarrow} \coqdoctac{unfold} \coqref{computational.disjunct}{\coqdocdefinition{disjunct}}\coqdoceol
\coqdocindent{1.00em}
\ensuremath{|} [ \ensuremath{\vdash} \coqdocvar{\_} \coqref{computational.::x '+' x}{\coqdocnotation{+}} \coqref{computational.::x '+' x}{\coqdocnotation{(}}\coqdocvar{\_} \coqref{computational.::x '@' x}{\coqdocnotation{@}} \coqdocvar{\_}\coqref{computational.::x '+' x}{\coqdocnotation{)}} \coqexternalref{:type scope:x '=' x}{http://coq.inria.fr/distrib/8.6/stdlib/Coq.Init.Logic}{\coqdocnotation{=}} \coqdocvar{\_} ] \ensuremath{\Rightarrow} \coqdoctac{unfold} \coqref{computational.disjunct}{\coqdocdefinition{disjunct}}\coqdoceol
\coqdocindent{1.00em}
\ensuremath{|} [ \ensuremath{\vdash} \coqref{computational.negation}{\coqdocdefinition{negation}} (\coqdocvar{\_} \coqref{computational.::x '@' x}{\coqdocnotation{@}} \coqdocvar{\_}) \coqexternalref{:type scope:x '=' x}{http://coq.inria.fr/distrib/8.6/stdlib/Coq.Init.Logic}{\coqdocnotation{=}} \coqdocvar{\_} ] \ensuremath{\Rightarrow} \coqdoctac{unfold} \coqref{computational.disjunct}{\coqdocdefinition{disjunct}}\coqdoceol
\coqdocindent{1.00em}
\coqdockw{end}.\coqdoceol
\coqdocemptyline
\coqdocnoindent
\coqdockw{Ltac} \coqdocvar{kapow\_1} \coqdocvar{X} :=\coqdoceol
\coqdocindent{1.00em}
\coqdoctac{try} (\coqdocvar{disjunctor};\coqdoceol
\coqdocindent{3.50em}
\coqdocvar{wham\_bam\_1} \coqdocvar{X}).\coqdoceol
\coqdocemptyline
\coqdocnoindent
\coqdockw{Ltac} \coqdocvar{kapow\_2} \coqdocvar{X} \coqdocvar{Y} :=\coqdoceol
\coqdocindent{1.00em}
\coqdoctac{try} (\coqdocvar{disjunctor};\coqdoceol
\coqdocindent{3.50em}
\coqdocvar{wham\_bam\_2} \coqdocvar{X} \coqdocvar{Y}).\coqdoceol
\coqdocemptyline
\coqdocnoindent
\coqdockw{Ltac} \coqdocvar{kapow\_3} \coqdocvar{X} \coqdocvar{Y} \coqdocvar{Z} :=\coqdoceol
\coqdocindent{1.00em}
\coqdoctac{try} (\coqdocvar{disjunctor};\coqdoceol
\coqdocindent{3.50em}
\coqdocvar{wham\_bam\_3} \coqdocvar{X} \coqdocvar{Y} \coqdocvar{Z}).\coqdoceol
\coqdocemptyline
\end{coqdoccode}
\section{Properties of Disjuncts}




Now we can proceed to page 33 and prove that the disjunct operator is
commutative, associative and distributive.  We also prove the property
of negation of a disjunction.


\begin{coqdoccode}
\coqdocemptyline
\coqdocnoindent
\coqdockw{Theorem} \coqdef{computational.disjunct comm}{disjunct\_comm}{\coqdoclemma{disjunct\_comm}} : \coqdockw{\ensuremath{\forall}} (\coqdocvar{a} \coqdocvar{b} : \coqref{computational.circuit}{\coqdocinductive{circuit}}),\coqdoceol
\coqdocindent{2.00em}
\coqdocvariable{a} \coqref{computational.::x '@' x}{\coqdocnotation{@}} \coqdocvariable{b} \coqexternalref{:type scope:x '=' x}{http://coq.inria.fr/distrib/8.6/stdlib/Coq.Init.Logic}{\coqdocnotation{=}} \coqdocvariable{b} \coqref{computational.::x '@' x}{\coqdocnotation{@}} \coqdocvariable{a}.\coqdoceol
\coqdocnoindent
\coqdockw{Proof}.\coqdoceol
\coqdocindent{1.00em}
\coqdoctac{intros} \coqdocvar{A} \coqdocvar{B}.\coqdoceol
\coqdocindent{1.00em}
\coqdocvar{kapow\_2} \coqdocvar{A} \coqdocvar{B}.\coqdoceol
\coqdocnoindent
\coqdockw{Qed}.\coqdoceol
\coqdocemptyline
\coqdocnoindent
\coqdockw{Theorem} \coqdef{computational.disjunct assoc}{disjunct\_assoc}{\coqdoclemma{disjunct\_assoc}} : \coqdockw{\ensuremath{\forall}} (\coqdocvar{a} \coqdocvar{b} \coqdocvar{c} : \coqref{computational.circuit}{\coqdocinductive{circuit}}),\coqdoceol
\coqdocindent{2.00em}
\coqref{computational.::x '@' x}{\coqdocnotation{(}}\coqdocvariable{a} \coqref{computational.::x '@' x}{\coqdocnotation{@}} \coqdocvariable{b}\coqref{computational.::x '@' x}{\coqdocnotation{)}} \coqref{computational.::x '@' x}{\coqdocnotation{@}} \coqdocvariable{c} \coqexternalref{:type scope:x '=' x}{http://coq.inria.fr/distrib/8.6/stdlib/Coq.Init.Logic}{\coqdocnotation{=}} \coqdocvariable{a} \coqref{computational.::x '@' x}{\coqdocnotation{@}} \coqref{computational.::x '@' x}{\coqdocnotation{(}}\coqdocvariable{b} \coqref{computational.::x '@' x}{\coqdocnotation{@}} \coqdocvariable{c}\coqref{computational.::x '@' x}{\coqdocnotation{)}}.\coqdoceol
\coqdocnoindent
\coqdockw{Proof}.\coqdoceol
\coqdocindent{1.00em}
\coqdoctac{intros} \coqdocvar{A} \coqdocvar{B} \coqdocvar{C}.\coqdoceol
\coqdocindent{1.00em}
\coqdocvar{kapow\_3} \coqdocvar{A} \coqdocvar{B} \coqdocvar{C}.\coqdoceol
\coqdocnoindent
\coqdockw{Qed}.\coqdoceol
\coqdocemptyline
\coqdocnoindent
\coqdockw{Theorem} \coqdef{computational.disjunct distrib}{disjunct\_distrib}{\coqdoclemma{disjunct\_distrib}} : \coqdockw{\ensuremath{\forall}} (\coqdocvar{a} \coqdocvar{b} \coqdocvar{c} : \coqref{computational.circuit}{\coqdocinductive{circuit}}),\coqdoceol
\coqdocindent{2.00em}
\coqdocvariable{a} \coqref{computational.::x '*' x}{\coqdocnotation{*}} \coqref{computational.::x '*' x}{\coqdocnotation{(}}\coqdocvariable{b} \coqref{computational.::x '@' x}{\coqdocnotation{@}} \coqdocvariable{c}\coqref{computational.::x '*' x}{\coqdocnotation{)}} \coqexternalref{:type scope:x '=' x}{http://coq.inria.fr/distrib/8.6/stdlib/Coq.Init.Logic}{\coqdocnotation{=}} \coqref{computational.::x '@' x}{\coqdocnotation{(}}\coqdocvariable{a} \coqref{computational.::x '*' x}{\coqdocnotation{*}} \coqdocvariable{b}\coqref{computational.::x '@' x}{\coqdocnotation{)}} \coqref{computational.::x '@' x}{\coqdocnotation{@}} \coqref{computational.::x '@' x}{\coqdocnotation{(}}\coqdocvariable{a} \coqref{computational.::x '*' x}{\coqdocnotation{*}} \coqdocvariable{c}\coqref{computational.::x '@' x}{\coqdocnotation{)}}.\coqdoceol
\coqdocnoindent
\coqdockw{Proof}.\coqdoceol
\coqdocindent{1.00em}
\coqdoctac{intros} \coqdocvar{A} \coqdocvar{B} \coqdocvar{C}.\coqdoceol
\coqdocindent{1.00em}
\coqdocvar{kapow\_3} \coqdocvar{A} \coqdocvar{B} \coqdocvar{C}.\coqdoceol
\coqdocnoindent
\coqdockw{Qed}.\coqdoceol
\coqdocemptyline
\coqdocnoindent
\coqdockw{Theorem} \coqdef{computational.disjunct neg}{disjunct\_neg}{\coqdoclemma{disjunct\_neg}} : \coqdockw{\ensuremath{\forall}} (\coqdocvar{a} \coqdocvar{b} : \coqref{computational.circuit}{\coqdocinductive{circuit}}),\coqdoceol
\coqdocindent{2.00em}
\coqref{computational.negation}{\coqdocdefinition{negation}} (\coqdocvariable{a} \coqref{computational.::x '@' x}{\coqdocnotation{@}} \coqdocvariable{b}) \coqexternalref{:type scope:x '=' x}{http://coq.inria.fr/distrib/8.6/stdlib/Coq.Init.Logic}{\coqdocnotation{=}} \coqdocvariable{a} \coqref{computational.::x '@' x}{\coqdocnotation{@}} \coqref{computational.::x '@' x}{\coqdocnotation{(}}\coqref{computational.negation}{\coqdocdefinition{negation}} \coqdocvariable{b}\coqref{computational.::x '@' x}{\coqdocnotation{)}}.\coqdoceol
\coqdocnoindent
\coqdockw{Proof}.\coqdoceol
\coqdocindent{1.00em}
\coqdoctac{intros} \coqdocvar{A} \coqdocvar{B}.\coqdoceol
\coqdocindent{1.00em}
\coqdocvar{kapow\_2} \coqdocvar{A} \coqdocvar{B}.\coqdoceol
\coqdocnoindent
\coqdockw{Qed}.\coqdoceol
\coqdocemptyline
\coqdocnoindent
\coqdockw{Theorem} \coqdef{computational.disjunct closed}{disjunct\_closed}{\coqdoclemma{disjunct\_closed}} : \coqdockw{\ensuremath{\forall}} (\coqdocvar{a} : \coqref{computational.circuit}{\coqdocinductive{circuit}}),\coqdoceol
\coqdocindent{2.00em}
\coqdocvariable{a} \coqref{computational.::x '@' x}{\coqdocnotation{@}} \coqref{computational.closed}{\coqdocconstructor{closed}} \coqexternalref{:type scope:x '=' x}{http://coq.inria.fr/distrib/8.6/stdlib/Coq.Init.Logic}{\coqdocnotation{=}} \coqdocvariable{a}.\coqdoceol
\coqdocnoindent
\coqdockw{Proof}.\coqdoceol
\coqdocindent{1.00em}
\coqdoctac{intros} \coqdocvar{A}.\coqdoceol
\coqdocindent{1.00em}
\coqdocvar{kapow\_1} \coqdocvar{A}.\coqdoceol
\coqdocnoindent
\coqdockw{Qed}.\coqdoceol
\coqdocemptyline
\coqdocnoindent
\coqdockw{Theorem} \coqdef{computational.disjunct open}{disjunct\_open}{\coqdoclemma{disjunct\_open}} : \coqdockw{\ensuremath{\forall}} (\coqdocvar{a} : \coqref{computational.circuit}{\coqdocinductive{circuit}}),\coqdoceol
\coqdocindent{2.00em}
\coqdocvariable{a} \coqref{computational.::x '@' x}{\coqdocnotation{@}} \coqref{computational.open}{\coqdocconstructor{open}} \coqexternalref{:type scope:x '=' x}{http://coq.inria.fr/distrib/8.6/stdlib/Coq.Init.Logic}{\coqdocnotation{=}} \coqref{computational.negation}{\coqdocdefinition{negation}} \coqdocvariable{a}.\coqdoceol
\coqdocnoindent
\coqdockw{Proof}.\coqdoceol
\coqdocindent{1.00em}
\coqdoctac{intros} \coqdocvar{A}.\coqdoceol
\coqdocindent{1.00em}
\coqdocvar{kapow\_1} \coqdocvar{A}.\coqdoceol
\coqdocnoindent
\coqdockw{Qed}.\coqdoceol
\coqdocemptyline
\coqdocemptyline
\coqdocemptyline
\coqdocemptyline
\end{coqdoccode}
\section{Synthesis of Symmetric Functions}




We prove the assertion at the top of page 40 and leave the remainder
of the section to future work.  We first create two Ltac tactics that
will be helpful.


\begin{coqdoccode}
\coqdocemptyline
\coqdocnoindent
\coqdockw{Ltac} \coqdocvar{hypothesis\_app} :=\coqdoceol
\coqdocindent{1.00em}
\coqdockw{match} \coqdockw{goal} \coqdockw{with}\coqdoceol
\coqdocindent{1.00em}
\ensuremath{|} [\coqdocvar{Hx} : (\coqdocvar{\_} \coqexternalref{:type scope:x '=' x}{http://coq.inria.fr/distrib/8.6/stdlib/Coq.Init.Logic}{\coqdocnotation{=}} \coqref{computational.closed}{\coqdocconstructor{closed}}) \ensuremath{\vdash} \coqdocvar{\_} ] \ensuremath{\Rightarrow} \coqdoctac{rewrite} \ensuremath{\rightarrow} \coqdocvar{Hx}\coqdoceol
\coqdocindent{1.00em}
\ensuremath{|} [\coqdocvar{Hx} : (\coqdocvar{\_} \coqexternalref{:type scope:x '=' x}{http://coq.inria.fr/distrib/8.6/stdlib/Coq.Init.Logic}{\coqdocnotation{=}} \coqref{computational.open}{\coqdocconstructor{open}}) \ensuremath{\vdash} \coqdocvar{\_} ] \ensuremath{\Rightarrow} \coqdoctac{rewrite} \ensuremath{\rightarrow} \coqdocvar{Hx}\coqdoceol
\coqdocindent{1.00em}
\coqdockw{end}.\coqdoceol
\coqdocemptyline
\coqdocnoindent
\coqdockw{Ltac} \coqdocvar{zap\_1} \coqdocvar{X} :=\coqdoceol
\coqdocindent{1.00em}
\coqdoctac{try} \coqdoctac{repeat} (\coqdocvar{hypothesis\_app};\coqdoceol
\coqdocindent{7.00em}
\coqdocvar{wham\_bam\_1} \coqdocvar{X}).\coqdoceol
\coqdocemptyline
\end{coqdoccode}


And then we can proceed with the symmetry example on page 40.


\begin{coqdoccode}
\coqdocemptyline
\coqdocnoindent
\coqdockw{Theorem} \coqdef{computational.symmetry example}{symmetry\_example}{\coqdoclemma{symmetry\_example}} : \coqdockw{\ensuremath{\forall}} (\coqdocvar{x} \coqdocvar{y} \coqdocvar{z} : \coqref{computational.circuit}{\coqdocinductive{circuit}}),\coqdoceol
\coqdocindent{2.00em}
\coqdocvariable{x} \coqexternalref{:type scope:x '=' x}{http://coq.inria.fr/distrib/8.6/stdlib/Coq.Init.Logic}{\coqdocnotation{=}} \coqref{computational.closed}{\coqdocconstructor{closed}} \coqexternalref{:type scope:x '->' x}{http://coq.inria.fr/distrib/8.6/stdlib/Coq.Init.Logic}{\coqdocnotation{\ensuremath{\rightarrow}}}\coqdoceol
\coqdocindent{2.00em}
\coqdocvariable{y} \coqexternalref{:type scope:x '=' x}{http://coq.inria.fr/distrib/8.6/stdlib/Coq.Init.Logic}{\coqdocnotation{=}} \coqref{computational.closed}{\coqdocconstructor{closed}} \coqexternalref{:type scope:x '->' x}{http://coq.inria.fr/distrib/8.6/stdlib/Coq.Init.Logic}{\coqdocnotation{\ensuremath{\rightarrow}}}\coqdoceol
\coqdocindent{2.00em}
\coqdocvariable{x} \coqref{computational.::x '*' x}{\coqdocnotation{*}} \coqdocvariable{y} \coqref{computational.::x '+' x}{\coqdocnotation{+}} \coqdocvariable{x} \coqref{computational.::x '*' x}{\coqdocnotation{*}} \coqdocvariable{z} \coqref{computational.::x '+' x}{\coqdocnotation{+}} \coqdocvariable{y} \coqref{computational.::x '*' x}{\coqdocnotation{*}} \coqdocvariable{z} \coqexternalref{:type scope:x '=' x}{http://coq.inria.fr/distrib/8.6/stdlib/Coq.Init.Logic}{\coqdocnotation{=}} \coqref{computational.closed}{\coqdocconstructor{closed}}.\coqdoceol
\coqdocnoindent
\coqdockw{Proof}.\coqdoceol
\coqdocindent{1.00em}
\coqdoctac{intros} \coqdocvar{X} \coqdocvar{Y} \coqdocvar{Z} \coqdocvar{xc} \coqdocvar{yc}.\coqdoceol
\coqdocindent{1.00em}
\coqdocvar{zap\_1} \coqdocvar{X}.\coqdoceol
\coqdocnoindent
\coqdockw{Qed}.\coqdoceol
\coqdocemptyline
\end{coqdoccode}


We relegate pages 41 to 50 as future work.




\section{A Selective Circuit}




In this section we formalize the example starting on page 51.  We
verify the reduction to the simplest serial-parallel form.


\begin{coqdoccode}
\coqdocemptyline
\coqdocemptyline
\coqdocemptyline
\coqdocnoindent
\coqdockw{Theorem} \coqdef{computational.selective circuit}{selective\_circuit}{\coqdoclemma{selective\_circuit}} : \coqdockw{\ensuremath{\forall}} (\coqdocvar{w} \coqdocvar{x} \coqdocvar{y} \coqdocvar{z} : \coqref{computational.circuit}{\coqdocinductive{circuit}}),\coqdoceol
\coqdocindent{2.00em}
\coqref{computational.::x '+' x}{\coqdocnotation{(}} \coqdocvariable{w} \coqref{computational.::x '*' x}{\coqdocnotation{*}} \coqdocvariable{x} \coqref{computational.::x '*' x}{\coqdocnotation{*}} \coqdocvariable{y} \coqref{computational.::x '*' x}{\coqdocnotation{*}} \coqdocvariable{z} \coqref{computational.::x '+' x}{\coqdocnotation{)}} \coqref{computational.::x '+' x}{\coqdocnotation{+}}\coqdoceol
\coqdocindent{2.00em}
\coqref{computational.::x '+' x}{\coqdocnotation{(}} \coqref{computational.::x '*' x}{\coqdocnotation{(}}\coqref{computational.negation}{\coqdocdefinition{negation}} \coqdocvariable{w}\coqref{computational.::x '*' x}{\coqdocnotation{)}} \coqref{computational.::x '*' x}{\coqdocnotation{*}} \coqref{computational.::x '*' x}{\coqdocnotation{(}}\coqref{computational.negation}{\coqdocdefinition{negation}} \coqdocvariable{x}\coqref{computational.::x '*' x}{\coqdocnotation{)}} \coqref{computational.::x '*' x}{\coqdocnotation{*}} \coqdocvariable{y} \coqref{computational.::x '*' x}{\coqdocnotation{*}} \coqdocvariable{z}\coqref{computational.::x '+' x}{\coqdocnotation{)}} \coqref{computational.::x '+' x}{\coqdocnotation{+}}\coqdoceol
\coqdocindent{2.00em}
\coqref{computational.::x '+' x}{\coqdocnotation{(}} \coqref{computational.::x '*' x}{\coqdocnotation{(}}\coqref{computational.negation}{\coqdocdefinition{negation}} \coqdocvariable{w}\coqref{computational.::x '*' x}{\coqdocnotation{)}} \coqref{computational.::x '*' x}{\coqdocnotation{*}} \coqdocvariable{x} \coqref{computational.::x '*' x}{\coqdocnotation{*}} \coqref{computational.::x '*' x}{\coqdocnotation{(}}\coqref{computational.negation}{\coqdocdefinition{negation}} \coqdocvariable{y}\coqref{computational.::x '*' x}{\coqdocnotation{)}} \coqref{computational.::x '*' x}{\coqdocnotation{*}} \coqdocvariable{z}\coqref{computational.::x '+' x}{\coqdocnotation{)}} \coqref{computational.::x '+' x}{\coqdocnotation{+}}\coqdoceol
\coqdocindent{2.00em}
\coqref{computational.::x '+' x}{\coqdocnotation{(}} \coqref{computational.::x '*' x}{\coqdocnotation{(}}\coqref{computational.negation}{\coqdocdefinition{negation}} \coqdocvariable{w}\coqref{computational.::x '*' x}{\coqdocnotation{)}} \coqref{computational.::x '*' x}{\coqdocnotation{*}} \coqdocvariable{x} \coqref{computational.::x '*' x}{\coqdocnotation{*}} \coqdocvariable{y} \coqref{computational.::x '*' x}{\coqdocnotation{*}} \coqref{computational.::x '*' x}{\coqdocnotation{(}}\coqref{computational.negation}{\coqdocdefinition{negation}} \coqdocvariable{z}\coqref{computational.::x '*' x}{\coqdocnotation{)}} \coqref{computational.::x '+' x}{\coqdocnotation{)}} \coqref{computational.::x '+' x}{\coqdocnotation{+}}\coqdoceol
\coqdocindent{2.00em}
\coqref{computational.::x '+' x}{\coqdocnotation{(}} \coqdocvariable{w} \coqref{computational.::x '*' x}{\coqdocnotation{*}} \coqref{computational.::x '*' x}{\coqdocnotation{(}}\coqref{computational.negation}{\coqdocdefinition{negation}} \coqdocvariable{x}\coqref{computational.::x '*' x}{\coqdocnotation{)}} \coqref{computational.::x '*' x}{\coqdocnotation{*}} \coqref{computational.::x '*' x}{\coqdocnotation{(}}\coqref{computational.negation}{\coqdocdefinition{negation}} \coqdocvariable{y}\coqref{computational.::x '*' x}{\coqdocnotation{)}} \coqref{computational.::x '*' x}{\coqdocnotation{*}} \coqdocvariable{z} \coqref{computational.::x '+' x}{\coqdocnotation{)}} \coqref{computational.::x '+' x}{\coqdocnotation{+}}\coqdoceol
\coqdocindent{2.00em}
\coqref{computational.::x '+' x}{\coqdocnotation{(}} \coqdocvariable{w} \coqref{computational.::x '*' x}{\coqdocnotation{*}} \coqref{computational.::x '*' x}{\coqdocnotation{(}}\coqref{computational.negation}{\coqdocdefinition{negation}} \coqdocvariable{x}\coqref{computational.::x '*' x}{\coqdocnotation{)}} \coqref{computational.::x '*' x}{\coqdocnotation{*}} \coqdocvariable{y} \coqref{computational.::x '*' x}{\coqdocnotation{*}} \coqref{computational.::x '*' x}{\coqdocnotation{(}}\coqref{computational.negation}{\coqdocdefinition{negation}} \coqdocvariable{z}\coqref{computational.::x '*' x}{\coqdocnotation{)}} \coqref{computational.::x '+' x}{\coqdocnotation{)}} \coqref{computational.::x '+' x}{\coqdocnotation{+}}\coqdoceol
\coqdocindent{2.00em}
\coqref{computational.::x '+' x}{\coqdocnotation{(}} \coqdocvariable{w} \coqref{computational.::x '*' x}{\coqdocnotation{*}} \coqdocvariable{x} \coqref{computational.::x '*' x}{\coqdocnotation{*}} \coqref{computational.::x '*' x}{\coqdocnotation{(}}\coqref{computational.negation}{\coqdocdefinition{negation}} \coqdocvariable{y}\coqref{computational.::x '*' x}{\coqdocnotation{)}} \coqref{computational.::x '*' x}{\coqdocnotation{*}} \coqref{computational.::x '*' x}{\coqdocnotation{(}}\coqref{computational.negation}{\coqdocdefinition{negation}} \coqdocvariable{z}\coqref{computational.::x '*' x}{\coqdocnotation{)}} \coqref{computational.::x '+' x}{\coqdocnotation{)}}\coqdoceol
\coqdocindent{2.00em}
\coqexternalref{:type scope:x '=' x}{http://coq.inria.fr/distrib/8.6/stdlib/Coq.Init.Logic}{\coqdocnotation{=}}\coqdoceol
\coqdocindent{2.00em}
\coqdocvariable{w} \coqref{computational.::x '*' x}{\coqdocnotation{*}} \coqref{computational.::x '*' x}{\coqdocnotation{(}} \coqdocvariable{x} \coqref{computational.::x '*' x}{\coqdocnotation{*}}\coqdoceol
\coqdocindent{5.00em}
\coqref{computational.::x '*' x}{\coqdocnotation{(}} \coqref{computational.::x '+' x}{\coqdocnotation{(}}\coqdocvariable{y} \coqref{computational.::x '*' x}{\coqdocnotation{*}} \coqdocvariable{z}\coqref{computational.::x '+' x}{\coqdocnotation{)}} \coqref{computational.::x '+' x}{\coqdocnotation{+}}\coqdoceol
\coqdocindent{6.00em}
\coqref{computational.::x '+' x}{\coqdocnotation{(}} \coqref{computational.::x '*' x}{\coqdocnotation{(}}\coqref{computational.negation}{\coqdocdefinition{negation}} \coqdocvariable{y}\coqref{computational.::x '*' x}{\coqdocnotation{)}} \coqref{computational.::x '*' x}{\coqdocnotation{*}}\coqdoceol
\coqdocindent{7.00em}
\coqref{computational.::x '*' x}{\coqdocnotation{(}}\coqref{computational.negation}{\coqdocdefinition{negation}} \coqdocvariable{z}\coqref{computational.::x '*' x}{\coqdocnotation{)}} \coqref{computational.::x '+' x}{\coqdocnotation{)}} \coqref{computational.::x '*' x}{\coqdocnotation{)}} \coqref{computational.::x '+' x}{\coqdocnotation{+}}\coqdoceol
\coqdocindent{5.00em}
\coqref{computational.::x '*' x}{\coqdocnotation{(}}\coqref{computational.negation}{\coqdocdefinition{negation}} \coqdocvariable{x}\coqref{computational.::x '*' x}{\coqdocnotation{)}} \coqref{computational.::x '*' x}{\coqdocnotation{*}}\coqdoceol
\coqdocindent{5.00em}
\coqref{computational.::x '*' x}{\coqdocnotation{(}} \coqref{computational.::x '+' x}{\coqdocnotation{(}}\coqref{computational.::x '*' x}{\coqdocnotation{(}}\coqref{computational.negation}{\coqdocdefinition{negation}} \coqdocvariable{y}\coqref{computational.::x '*' x}{\coqdocnotation{)}} \coqref{computational.::x '*' x}{\coqdocnotation{*}} \coqdocvariable{z}\coqref{computational.::x '+' x}{\coqdocnotation{)}} \coqref{computational.::x '+' x}{\coqdocnotation{+}}\coqdoceol
\coqdocindent{6.00em}
\coqref{computational.::x '+' x}{\coqdocnotation{(}}\coqdocvariable{y} \coqref{computational.::x '*' x}{\coqdocnotation{*}} \coqref{computational.::x '*' x}{\coqdocnotation{(}}\coqref{computational.negation}{\coqdocdefinition{negation}} \coqdocvariable{z}\coqref{computational.::x '*' x}{\coqdocnotation{)}}\coqref{computational.::x '+' x}{\coqdocnotation{)}} \coqref{computational.::x '*' x}{\coqdocnotation{)}} \coqref{computational.::x '*' x}{\coqdocnotation{)}} \coqref{computational.::x '+' x}{\coqdocnotation{+}}\coqdoceol
\coqdocindent{2.00em}
\coqref{computational.::x '*' x}{\coqdocnotation{(}}\coqref{computational.negation}{\coqdocdefinition{negation}} \coqdocvariable{w}\coqref{computational.::x '*' x}{\coqdocnotation{)}} \coqref{computational.::x '*' x}{\coqdocnotation{*}}\coqdoceol
\coqdocindent{2.00em}
\coqref{computational.::x '*' x}{\coqdocnotation{(}} \coqref{computational.::x '+' x}{\coqdocnotation{(}} \coqdocvariable{x} \coqref{computational.::x '*' x}{\coqdocnotation{*}} \coqref{computational.::x '*' x}{\coqdocnotation{(}} \coqref{computational.::x '+' x}{\coqdocnotation{(}}\coqref{computational.::x '*' x}{\coqdocnotation{(}}\coqref{computational.negation}{\coqdocdefinition{negation}} \coqdocvariable{y}\coqref{computational.::x '*' x}{\coqdocnotation{)}} \coqref{computational.::x '*' x}{\coqdocnotation{*}} \coqdocvariable{z}\coqref{computational.::x '+' x}{\coqdocnotation{)}} \coqref{computational.::x '+' x}{\coqdocnotation{+}}\coqdoceol
\coqdocindent{7.00em}
\coqref{computational.::x '+' x}{\coqdocnotation{(}}\coqdocvariable{y} \coqref{computational.::x '*' x}{\coqdocnotation{*}} \coqref{computational.::x '*' x}{\coqdocnotation{(}}\coqref{computational.negation}{\coqdocdefinition{negation}} \coqdocvariable{z}\coqref{computational.::x '*' x}{\coqdocnotation{)}}\coqref{computational.::x '+' x}{\coqdocnotation{)}} \coqref{computational.::x '*' x}{\coqdocnotation{)}} \coqref{computational.::x '+' x}{\coqdocnotation{)}} \coqref{computational.::x '+' x}{\coqdocnotation{+}}\coqdoceol
\coqdocindent{3.00em}
\coqref{computational.::x '+' x}{\coqdocnotation{(}} \coqref{computational.::x '*' x}{\coqdocnotation{(}}\coqref{computational.negation}{\coqdocdefinition{negation}} \coqdocvariable{x}\coqref{computational.::x '*' x}{\coqdocnotation{)}} \coqref{computational.::x '*' x}{\coqdocnotation{*}} \coqdocvariable{y} \coqref{computational.::x '*' x}{\coqdocnotation{*}} \coqdocvariable{z} \coqref{computational.::x '+' x}{\coqdocnotation{)}} \coqref{computational.::x '*' x}{\coqdocnotation{)}}.\coqdoceol
\coqdocnoindent
\coqdockw{Proof}.\coqdoceol
\coqdocindent{1.00em}
\coqdoctac{intros} \coqdocvar{W} \coqdocvar{X} \coqdocvar{Y} \coqdocvar{Z}.\coqdoceol
\coqdocindent{1.00em}
\coqdocvar{wham\_bam\_4} \coqdocvar{W} \coqdocvar{X} \coqdocvar{Y} \coqdocvar{Z}.\coqdoceol
\coqdocnoindent
\coqdockw{Qed}.\coqdoceol
\coqdocemptyline
\end{coqdoccode}
\section{Future Work}




There are a significant number of claims and assertions that have been
proven in this paper.  However, there are still a significant number
of claims not explicitly proven which have been relegated to future
work.


Additional future work is to convert the postulates into a set of
relations.  These might allow the more elegant encoding of the
non-serial and non-parallel transformations such as wye and delta
transformations so that those transformations could be explicitly used
in proofs of more complicated hindrance functions.  In the current
work, we can only prove ``slices'' of these topologies because of the
lack of ability to precisely define the delta and wye transformations.


In the event that a complete formalization of Claude Shannon's thesis
were completed, we would have a solid foundation upon which to build
electromechanical relay circuits of the future.


 

\section{Conclusion}




This paper has provided proofs of many of the claims and assertions
made in Claude Shannon's masters thesis.  In some sense these proofs
can serve as an additional reading companion -- helping readers stay
on the topic of the interesting ideas in the thesis without getting
distracted about whether a particular claim or assertion is true.


\begin{coqdoccode}
\coqdocemptyline
\end{coqdoccode}

\bibliographystyle{plain}
\bibliography{shannon} 
% Note the lack of whitespace between the commas and the next bib file.

\end{document}
